\documentclass{article}
\usepackage[
  a4paper,
  margin=1in,
  headsep=4pt, % separation between header rule and text
]{geometry}

\usepackage[english]{babel}
\usepackage[utf8]{inputenc}
\usepackage{xcolor}
\usepackage{fancyhdr}
\usepackage{tgschola}
\usepackage{lastpage}
\usepackage{amsmath}




\newcommand{\soptitle}{First order logic of sparse hypergraphs}

\newcommand{\yourname}{Lázaro Alberto Larrauri Borroto}
\newcommand{\youremail}{email@address.edu}
\newcommand{\yourweb}{https://www.abcd.com/}

\newcommand{\statement}[1]{\par\medskip
  \textcolor{blue}{\textbf{#1:}}\space
}

\usepackage[
  breaklinks,
  pdftitle={\yourname - \soptitle},
  pdfauthor={\yourname},
  unicode
]{hyperref}

\begin{document}

\begin{large}
\underline{\textbf{Personal statement}}
\end{large}

\bigskip





From a very young age the aspect of studying that 
I enjoyed the most was solving problems. Because 
of my high motivation and the support from my parents
and teachers, I was able to finish elementary school a 
year earlier than usual. Since I was thirteen
I have been participating in math and science competitions with
various degrees of success. At the time I finished my 
secondary studies, it was said that 
studying math would lead to fewer career possibilities
than other sciences or engineering. In spite of that,
I chose to pursue a math degree because it was what 
interested me the most. \par

Currently I am mostly interested in the connection between
logic, discrete mathematics and complexity theory, but my 
first interests were in the foundations of mathematics. 
Shortly after my math studies in university began, I discovered
the fact that some important theorems require the Axiom 
of Choice to be proven while this same axiom allows for counter-
intuitive results like the Banach-Tarski paradox. This was puzzling 
to me. You even need weak forms of this axiom if you want to show
that various different -and seemingly reasonable- definitions of 
finiteness are equivalent to each other. From there on I started 
studying mathematical logic and stumbled upon computability 
theory. These topics, as well as others that also interested me 
to a lesser extent, like non standard analysis or categorical logic, 
were not taught to me during my Bachelor's degree, so what I learnt 
about them was mostly through self-studying. \par
It was not until my master's thesis when I could do research related
to logic. My supervisor, Marc Noy, introduced me to the field of finite
model theory, which is of a fairly different nature from that of the logic 
I had studied up until that point but was equally interesting to me. 
The goal of my thesis was to extend a theorem that showed a certain 
convergence law  

% Also, there were people 
%that rejected this axiom and others that are usually taken 
%for granted, like the Principle of the Excluded Middle, and 
%did mathematics in a different way. It was a shocking discovery 
%to me that mathematical truths were 







%\bibliography{references}
%\bibliographystyle{acm}

\end{document}
