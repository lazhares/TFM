\documentclass{article}
\usepackage[
  a4paper,
  margin=1in,
  headsep=4pt, % separation between header rule and text
]{geometry}

\usepackage[spanish]{babel}
\usepackage[utf8]{inputenc}
\usepackage{xcolor}
\usepackage{fancyhdr}
\usepackage{tgschola}
\usepackage{lastpage}
\usepackage{amsmath}
\pagestyle{fancy}
\fancyhf{}


\newcommand{\soptitle}{First order logic of sparse hypergraphs}

\newcommand{\yourname}{Lázaro Alberto Larrauri Borroto}
\newcommand{\youremail}{email@address.edu}
\newcommand{\yourweb}{https://www.abcd.com/}

\newcommand{\statement}[1]{\par\medskip
  \textcolor{blue}{\textbf{#1:}}\space
}

\usepackage[
  breaklinks,
  pdftitle={\yourname - \soptitle},
  pdfauthor={\yourname},
  unicode
]{hyperref}

\begin{document}

\begin{center}\LARGE\soptitle\\ 
\vspace{0.5cm}
\Large{Master's thesis proposal for Fall-2019} \\
\yourname
\end{center}

\hrule
\vspace{1pt}
\hrule height 1pt

\bigskip

The goal of this project is to generalize, if possible, the results obtained by 
James F. Lynch in \cite{lynch1992probabilities} about the first-order logic of random graphs to 
the framework of uniform random hypergraphs. A motivation for that field of research is the possibility
of studying very large relational structures, as graphs, by means of analyzing their `typical' properties. 
These structures are of interest because they appear in a great variety of places, ranging from social networks to 
biology, but the analysis of very large individual graphs, for example, is a difficult task. 
Instead, the aforementioned approach has proven to be a useful tool. \par

Going into further detail, in \cite{lynch1992probabilities} are studied the limit properties of a certain model
of random graphs under first-order logic. This model $\mathcal{G}(n,p)$ is one where a graph with $n$ vertices is chosen
randomly in such a way that each pair of vertices is joined by an edge with probability $p$ independently. In other 
words, for any given graph $G$ with $n$ vertices and $m$ edges, the probability of obtaining $G$ in $\mathcal{G}(n,p)$
is $p^m(1-p)^{\binom{n}{2} - m}$. To describe these graphs one can use the first order language which is provided with 
only one binary relation. This way, the variables in this language represent vertices of the graph, while the binary relation 
represents the existence or non-existence of an edge between two vertices. A sentence like $\forall x_1, x_2 (\lnot(x_1=x_2)\implies \exists x_3 (\lnot(x_3=x_1) \wedge \lnot(x_3=x_2)\wedge R(x_1,x_3) \wedge R(x_2,x_3))$ can be interpreted as `for any pair of different vertices $x_1, x_2$ there exists third one $x_3$ that is adjacent to $x_1$ and $x_2$ '. Thus, given a sentence  $\phi$ in this language one can talk about the probability of $\phi$ being satisfied in $\mathcal{G}(n,p)$, i.e. $Pr(\mathcal{G}(n,p) \models \phi)$, or even about the limit of this probability when $n$ tends to infinity- which may or may not exist. 
One of the first important results in this area is the Zero-One Law of random graphs, which states that for any constant $p$, and any
first-order sentence $\phi$, the limit $\lim\limits_{n\rightarrow \infty} Pr(\mathcal{G}(n,p) \models \phi)$ always exists
and it is either zero or one. \par
If one wants the graphs in $\mathcal{G}(n,p)$ to typically contain fewer edges one can consider $p$ as a decreasing function
on $n$. For the particular case of $p=\frac{\beta}{n}$, James F. Lynch proved in 1992 \cite{lynch1992probabilities} that the limit $\lim\limits_{n\rightarrow \infty} Pr(\mathcal{G}(n,\frac{\beta}{n}) \models \phi)$ always existed, even though one can no longer assure that it is zero or one. He also showed that this limit has good properties when seen as a function on $\beta$, proving that it is an analytic function, and that its values are elementary functions associated to the parameters of certain Poisson distributions. \par
In this Master's thesis project the aim is to generalize these results to the model of random uniform hypergraphs 
$\mathcal{H}^k(n,p)$. In this model one consider hypergraphs- graph-like structures where each edge can join an arbitrary number 
of vertices- where each edge contains exactly $k$ vertices and the probability of any $k$ of them being joined by some edge is $p$ independently. Analogously to the case of $\mathcal{G}(n,p)$, one can describe these hypergraphs using the first order language provided only with one $k$-ary relation. We will also try to determine the closure of the set of limiting probabilities as in \cite{heinig2018logical} in a related problem.
\par
To reach the goal of this project a good understanding of Lynch's techniques, as well as of the model $\mathcal{H}^k(n,p)$
will be required. Among those techniques there are tools from finite model theory, namely the Ehrenfeucht-Fraissé games 
(see chapter 3, \cite{libkin2013elements}), and involved combinatorial arguments which need to be translated to this model of random hypergraphs. 




\bibliography{references}
\bibliographystyle{acm}

\end{document}
