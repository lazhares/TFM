
\documentclass{article}
\usepackage[
a4paper,
margin=1in,
headsep=4pt, % separation between header rule and text
]{geometry}
\usepackage{amssymb,mathtools, amsmath, amsfonts, amsthm}
\begin{document}
	
Title: On the evolution of the set of limiting probabilities of first order properties for sparse random graphs. \par

Abstract: This is joint work with Marc Noy and Tobias M\"uller. It is known that for any first order property of graphs P,
the limit probability that the random graph $G(n,c/n)$ satisfies $P$ as $n$ tends to infinity exists and varies in a smooth 
way with $c$. An immediate consequence of this is that first order properties cannot individually ``capture'' the phase
transition that occurs in $G(n,c/n)$ when c=1. \par
We consider the set of limiting probabilities
\[L_c:= \{ \lim\limits_{n->\infty} \mathrm{Pr}(G(n,c/n) \text{ satisfies } P \, |
\, P \text{ first order property } \}.\]
We ask the question of whether we can ``detect" the phase transition in $G(n,c/n)$ through the changes in $L_c$. We 
arrive at a negative answer and show that there is a constant $c_0\simeq 0.93...$ such that the closure $\overline{L_c}$ 
of $L_c$ is the whole interval $[0,1]$ when $c\geq c_0$ and $\overline{L_c}$ is a finite union of disjoint intervals when 
$c< c_0$. Moreover, the number of intervals forming  $\overline{L_c}$ tends to infinity as $c$ tends to zero. The same
question can be asked in the setting of random uniform hypergraphs and similar results are obtained. 
\end{document}