\documentclass[12pt,notitlepage,a4paper]{article}

\usepackage[left=2.5cm,right=2.5cm,top=2.5cm,bottom=2.5cm]{geometry}
\usepackage{graphicx}
\usepackage{amssymb,mathtools, amsmath, amsfonts, amsthm}
%\usepackage{color}
\usepackage{float}
\usepackage{hyperref}
\usepackage{enumerate}
\usepackage{enumitem}
\usepackage{chngcntr}
\usepackage{cleveref}
\usepackage{pdfpages}

\usepackage{mathptmx}

\counterwithout{equation}{section}

\newlength{\margen}
\setlength{\margen}{\paperwidth}
\addtolength{\margen}{-\textwidth}
\addtolength{\skip\footins}{0.7 cm}
\setlength{\margen}{0.5\margen}
\addtolength{\margen}{-1in}
\setlength{\oddsidemargin}{\margen}
\setlength{\evensidemargin}{\margen}
\setlength{\abovedisplayskip}{3pt}
\setlength{\belowdisplayskip}{3pt}
%%%% Small setup %%%%
\hypersetup{
	colorlinks=false,
	pdfborder={1 1 0.0005},
}
\setlength{\parskip}{0.2cm}
%%%%%%%%%%%%%%%
\usepackage{tikz-cd}
\usetikzlibrary{cd}
\usepackage[english]{babel}
\usepackage{todonotes}
\usepackage{cleveref}
\usepackage{caption}
\usepackage{subcaption}
\usepackage{bbding}
\usepackage{tcolorbox}
\usepackage{natbib}


\newtheorem{proposition}{Proposition}[section]
\newtheorem{fact}{Fact}[section]
\newtheorem{theorem}{Theorem}[section]
\newtheorem{obs}{Observation}[section]
\newtheorem{lemma}{Lemma}[section]
\newtheorem{corollary}{Corollary}[section]
\theoremstyle{definition}
\newtheorem{definition}{Definition}[section]
\newtheorem{propdef}{Proposition / Definition}[section]
\newtheorem{remark}{Remark}[section]


\newcommand{\cc}{\mathfrak{c}}
\newcommand{\Z}{\mathbb{Z}}
\newcommand{\CC}{\mathbb{C}}
\newcommand{\Q}{\mathbb{Q}}
\newcommand{\R}{\mathbb{R}}
\newcommand{\N}{\mathbb{N}}
\newcommand{\Hc}{\mathcal{H}}
\newcommand{\Lan}{\mathcal{L}}
\newcommand{\Ln}{\lim\limits_{n\to \infty}}
\newcommand{\clist}{\mathfrak{c}_{1}, \cdots, \mathfrak{c}_m}
\newcommand{\morph}[1]{\simeq_#1}
\newcommand{\vlst}[2]{#1_1,\dots, #1_{#2}}
\newcommand{\gnp}{G(n,\beta_1/n^{a_1-1}, \dots,\beta_l/n^{a_l-1})}
\newcommand{\overbar}[1]{\mkern 1.5mu\overline{\mkern-1.5mu#1\mkern-1.5mu}\mkern 1.5mu}
\newcommand{\ehr}{\textsc{Ehr}}
\newcommand{\PR}[1]{\mathrm{Pr}\big(#1\big)}
\newcommand{\Tr}{\mathrm{Tr}}

\begin{document}
	
	Let $\overline{w} \in (\N)_*$ be a finite set of fixed vertices.
	Given a vertex $v\in V$ we define the set $E([n], \overline{w};v)$
	as the set of loop-free edges $e\in E([n])$ that satisfy $e\cap \overline{w}=\{v\}$.
	Consider a pattern $\epsilon$. We define the set
	$Copies(\epsilon, [n], V;v)$ as the set of $\Sigma$-edges 
	$(e,\chi)\in Copies(\epsilon, [n])$ satisfying that 
	$e\in E([n], V;v)$ and that $\chi(v)=\tau$.	\par

	\begin{theorem}
		Let $\overline{w}\in (\N)_*$ be a finite list
		of vertices and let $\phi(\overline{x})$ be a consistent
		edge sentence with $len(\overline{w})=len(\overline{x})$.
		Let	$\overline{v}\subset \overline{w}$. For each 
		$v\in \overline{v}$ let $r(v)\in \N$
		and $T_{n,v}=Tr(G_n(\overline{w}),v;r(v))$.
		For each $v\in \overline{v}$ let 
		$E_v\subset E(\N,\overline{w},v)$ be a finite set. Suppose
		that each $u\in \N \setminus \overline{w}$
		belongs at most to one edge $e\in 
		\bigcup_{v\in \overline{v} } E_v$.
		Then
		\[
		\Ln \Pr\big( \bigwedge_{\substack{v\in \overline{v}\\
		e\in E_v}} e\in E(T_{n,v}) \, \big| \,
		\phi(\overline{w})
		\bigwedge_{\substack{v\in \overline{v}\\
		e\in E_v}} e\in E(G_n)	\big)=1.	
		\]
		
	\end{theorem}
	\begin{proof}
		Let $r=\max\limits_{v\in \overline{v}}(r(v))$. Suppose that
		for some $v$, some $e\in E_v$ satisfies that
		$e\in E(G_n)$ but $e\notin E(T_{n,v})$. Let
		$\overline{u}\in (\N)_*$ be a list that contains precisely 
		the vertices in $\overline{w}$ and the ones belonging to any
		edge $e^\prime\in \bigcup_{v\in \overline{v}}$. Then
		in $G_n\setminus E(\overline{u})$ there must be path
		whose length is bounded by $2r$ joining
		a vertex in $e$ with either another vertex $w\in \overline{u}$
		or with a cycle $C \subset G_n\setminus E(\overline{u})$ whose
		diameter is bounded by $r$. That is, $SIMPLE_n(\overline{u},r)$
		is not satisfied. Notice that the event $\phi(\overline{w}) 
		\bigwedge_{\substack{v\in \overline{v}\\e\in E_v}} e\in E(G_n)$ 
		can be described by an open sentence whose variables are interpreted as $\overline{u}$. 
		In consequence, using REF we obtain
		\[
		\Ln \Pr\big( SIMPLE_n(\overline{u},r) \, \big|
		\phi(\overline{w}) \bigwedge_{\substack{v\in \overline{v}\\
		e\in E_v}} e\in E(G_n) \big)=1,
		\]
		and the result follows. 
	\end{proof}
	Given
\end{document}