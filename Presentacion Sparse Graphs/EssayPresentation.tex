\documentclass[handout, 11pt]{beamer}

\usepackage[english] {babel}                 % Idioma (és important a l'hora de separar una paraula a final de línia)
%\ProvidesPackage{myStyle}

% Plantilla versió 2.6.1                                4 / 1 / 2017

 \usepackage{bbm}                                       % Permet fer \1
            %%% Nota: Si la plantilla no compila, comenta els dos packages anteriors.
 \usepackage[utf8]{inputenc}                            % Accents i símbols extranys
 \usepackage{amsfonts,amssymb}                          % Símbols matemàtics
 \usepackage{mathrsfs}                                  % Lletra cal·ligràfica (millor que \mathcal)
 \usepackage{amsmath}                                   % Mode matemàtic
\usepackage{amsthm}										% TheoremStyle
\usepackage{bookman}                                   % BarreroTimes
 \usepackage{verbatim}                                  % Verbatim + comentaris multilínia
 \usepackage{graphicx,subcaption}                       % Fotos i subfotos
 \usepackage[free-standing-units]{siunitx}              % Sistema internacional
 \usepackage[siunitx, american]{circuitikz}             % Circuits
 \usepackage{multicol}
 \usepackage{float}                                     % Posa les figures on toca
 \usepackage{anysize}                                   % Posa els marges
 \marginsize{2cm}{2cm}{1cm}{4cm}     % {L}{R}{U}{D}
 \usepackage{tikz-cd}                                   % Diagrames commutatius
 \usetikzlibrary{babel}                                 % Evita interferència entre tikz-cd i babel
%\usepackage{helvet}                                     % Estas dos lineas
%\renewcommand{\familydefault}{\sfdefault}               % ponen Arial
\usepackage[toc,page]{appendix}                         % for appendices
\usepackage{enumerate}					% enumerate options
\usepackage{hyperref}					%for hyperrefferences  (needed in bibliography
\usepackage[square,comma,numbers,sort&compress]{natbib} % bibliography stuff (see http://www.colorado.edu/physics/phys4610/phys4610_sp12/bibtex_guide.pdf)
\usepackage{xcolor} %Texto colores
%\usepackage{bm}						%for bold math symbols \pmb  PROBLEMATIC FOR BEAMER
\usepackage[nobreak=true]{mdframed}					% this is to frame theorems and definitions and figures
\usepackage{lmodern}
\usepackage{pst-func}
\usepackage{pst-ode,pst-plot}

\usepackage{pstricks}
%\usepackage{authblk}			% author affiliation PROBLEMATIC FOR BEAMER
%\usepackage{enumitem}
\usepackage{moreenum}
\usepackage{cleveref} % Clever referencing http://ctan.math.washington.edu/tex-archive/macros/latex/contrib/cleveref/cleveref.pdf

\setcounter{secnumdepth}{5}

\numberwithin{equation}{section}			% equation numbering
\setlength{\parskip}{1em}
\setlength\parindent{0cm}


\renewcommand\emph[1]{\textit{\textbf{#1}}}

%Conjunts importants
\def\NN{\mathbb N}   % Naturals
\def\ZZ{\mathbb Z}   % Enters
\def\QQ{\mathbb Q}   % Racionals
\def\RR{\mathbb R}   % Reals
\def\CC{\mathbb C}   % Complexos
\def\HH{\mathbb H}   % Quaternions, semiplà superior complex
\def\AA{\mathbb A}   % Espai afí en general
\def\EE{\mathbb E}   % Extensió d'un cos, esperança
\def\FF{\mathbb F}   % Cos primer, cos finit
\def\KK{\mathbb K}   % Cos en general
\def\PP{\mathbb P}   % Espai projectiu, nombres primers
\def\DD{\mathbb D}   % Disc unitat complex
\def\SS{\mathbb S}   % Esfera
\def\TT{\mathbb T}   % Tor (n-dimensional)
\def\XX{\mathbb X}   

\def\C{\mathcal C}                                      % Funcions contínues, derivables amb continuïtat
\def\D{\mathcal D}	%Deep networks
\def\F{\mathcal F}	%Deep networks
\def\G{\mathcal G}	% Graph
\def\S{\mathcal S}	% Shallow "

% Per escriure menys
\def\oo{\infty}                                         % Infinit
\def\P{\mathscr P}                                      % Conjunt potència
\def\A{\mathscr A}                                      % Àlgebra, sigma-àlgebra
\def\B{\mathscr B}                                      % Base
\def\T{\mathscr T}                                      % Topologia
%\def\L{\mathscr L}                                      % Espai d'homomorfismes, derivada de Lie
\def\O{\mathcal O}										% big O notation
\def\D{\mathscr D}                                      % Funcions derivables
\def\mm{\mathfrak M}                                    % Matrius
%\def\ss{\mathfrak S}                                    % Grup simètric
\def\aa{\mathfrak A}                                    % Grup alternat
\def\Re#1{\mathrm {Re}\left(#1\right)}              % Part real
\def\Im#1{\mathrm {Im}\left(#1\right)}              % Part imaginària
\def\1{\mathbbm{1}}                                     % Funció indicadora
\def\<{\langle}                                         % subespai/ideal/subgrup -
\def\>{\rangle}%                                          generat per, producte escalar
\def\sect{\mathsection}                                 % Secció
\def\pgph{\mathparagraph}                               % Fi de paràgraf
\def\qed{\hfill\square}                                 % Quadrat blanc de Q.E.D.
\def\defs{\stackrel{\tiny{\mbox{def}}}{=}}		% For definitions

%Calia
\def\phi{\varphi}       %%% Aquesta comanda inhabilita el "\phi" lleig
\def\eps{\varepsilon}   %\epsilon

\DeclareMathOperator*{\argmin}{arg\,min}                                 % Punt on s'assoleix el mínim
\DeclareMathOperator*{\argmax}{arg\,max}                                 % Punt on s'assoleix el màxim
\DeclareMathOperator*{\mex}{mex}                                         % Mínim ordinal exclòs
\DeclareMathOperator*{\sgn}{sgn}                                         % Signe
\DeclareMathOperator*{\im}{Im}                                           % Imatge
\DeclareMathOperator*{\Tr}{Tr}                                           % Traça
\DeclareMathOperator*{\Id}{Id}                                           % Identitat
\DeclareMathOperator*{\supp}{supp}                                       % Suport
\DeclareMathOperator*{\esup}{ess\,sup}                                   % Essential support
\DeclareMathOperator*{\Span}{span}                                       % Span
\DeclareMathOperator*{\prolim}{\underleftarrow{\rm{proj\,lim}}}          % Límit projectiu
\DeclareMathOperator*{\indlim}{\underrightarrow{\rm{ind\,lim}}}          % Límit inductiu

%Trigonometria bàsica
\DeclareMathOperator*{\tg}{tg}                          % tg(·)          = sin(·)/cos(·)
\DeclareMathOperator*{\cosec}{cosec}                    % cosec(·)       = 1/sin(·)
\DeclareMathOperator*{\cotg}{cotg}                      % cotg(·)        = cos(·)/sin(·)

\DeclareMathOperator*{\arctg}{arctg}
\DeclareMathOperator*{\arcsec}{arcsec}
\DeclareMathOperator*{\arccosec}{arccosec}
\DeclareMathOperator*{\arccotg}{arccotg}

\DeclareMathOperator*{\tgh}{tgh}                        % tgh(·)         = sinh(·)/cosh(·)
\DeclareMathOperator*{\sech}{sech}                      % sech(·)        = 1/cosh(·)
\DeclareMathOperator*{\cosech}{cosech}                  % cosech(·)      = 1/sinh(·)
\DeclareMathOperator*{\cotgh}{cotgh}                    % cotgh(·)       = cosh(·)/sinh(·)

\DeclareMathOperator*{\arcsinh}{arcsinh}
\DeclareMathOperator*{\arccosh}{arccosh}
\DeclareMathOperator*{\arctgh}{arctgh}
\DeclareMathOperator*{\arcsech}{arcsech}
\DeclareMathOperator*{\arccosech}{arccosech}
\DeclareMathOperator*{\arccotgh}{arccotgh}

% Operadors grans
\DeclareMathOperator*{\bigcomma}{\raisebox{0.9ex}{\Huge ,}}  % Comatori                    %% e.g. $\RR = \left\{\bigcomma\limits_{x\in\RR} x \right\}$
\DeclareMathOperator*{\bigtimes}{\text{\Large $\times$}}     % Cartesionatori              %% e.g. $\bigtimes\limits_{i\in B} A = A^B :=\{f:B\to A\}$
\DeclareMathOperator*{\bigvoid}{\text{\Large $\O$}}          % Concatenatori               %% e.g. $\bigvoid\limits_{i=1}^n \left(\sum\limits_{j_i=1}^n\right) 1 = n^n$
\DeclareMathOperator*{\bigequals}{\text{\Large $=$}}         % Igualatori                  %% e.g. $\bigequals\limits_{n=1}^\oo \sum\limits_{i=1}^n \dfrac1n$
\DeclareMathOperator*{\bigle}{\text{\Large $\le$}}           % Creixentatori               %% e.g. $0<\bigle\limits_{n=1}^\oo a_n<k\implies\exists\lim\limits_{n\to\oo}a_n\leq k$
\DeclareMathOperator*{\bigge}{\text{\Large $\ge$}}           % Decreixentatori
\DeclareMathOperator*{\bigless}{\text{\Large $<$}}           % Creixentatori estricte
\DeclareMathOperator*{\biggreater}{\text{\Large $>$}}        % Decreixentatori estricte
\DeclareMathOperator*{\bigsse}{\text{\Large $\sse$}}         % Inclusionatori              %% e.g. $\O\sse\bigsse\limits_{n=1}^\oo (-n,n)\sse\R$
\DeclareMathOperator*{\bigspse}{\text{\Large $\spse$}}       % Antiinclusionatori
\DeclareMathOperator*{\bigsss}{\text{\Large $\sss$}}         % Inclusionatori estricte
\DeclareMathOperator*{\bigssne}{\text{\Large $\ssne$}}       % Inclusionatori estricte
\DeclareMathOperator*{\bigssps}{\text{\Large $\ssps$}}       % Antiinclusionatori estricte
\DeclareMathOperator*{\bigspsne}{\text{\Large $\spsne$}}     % Antiinclusionatori estricte
\DeclareMathOperator*{\bigni}{\text{\Large $\ni$}}           % Contenatori                 %% e.g. $\nexists\bigni\limits_{i=0}^\oo A_i$
\DeclareMathOperator*{\bigin}{\text{\Large $\in$}}           % Pertanyatori
\DeclareMathOperator*{\bigo}{\bigcirc}                       % Compositori                 %% e.g. $\bigequals\limits_{x\in\R}\left(\bigo\limits_{i=1}^\oo \cos\right) (x)$
\DeclareMathOperator*{\bigfrac}{\raisebox{-.5ex}{\Large K}}  % K-atori                     %% e.g. $\bigfrac\limits_{i=1}^\oo(b_i:c_i):=\cfrac{b_1}{c_1+\cfrac{b_2}{c_2+\ddots}}$
            %%% Nota: El sumadirectori (\bigoplus), els dos uniodisjuntatoris (\bigsqcup, \biguplus) i el tensionatori (\bigotimes) ja estan implementats per defecte.
\newcommand\restr[2]{{% we make the whole thing an ordinary symbol    		%restriccions de funcions
  \left.\kern-\nulldelimiterspace % automatically resize the bar with \right
  #1 % the function
  \vphantom{\big|} % pretend it's a little taller at normal size
  \right|_{#2} % this is the delimiter
  }}




\usetheme{CambridgeUS}
\usepackage{color}
\usepackage{amssymb,mathtools, amsmath, amsfonts, amsthm}


\setbeamertemplate{itemize items}[square]
\newtheorem{inneraxiom}{Axiom}
\newenvironment{axiom}[1]
{\renewcommand\theinneraxiom{#1}\inneraxiom}
{\endinneraxiom}
\definecolor{MyBackground}{RGB}{0.149    0.218   0.443}

\title[An Introduction to Synthetic Differential Geometry]{An Introduction to Synthetic Differential Geometry}


\author[Differential Geometry]{Lázaro Alberto Larrauri Borroto}
\date\today




\begin{document}
	\frame{\titlepage}
	\frame{\tableofcontents}
	
	
	
	
\section{Introduction}
\begin{frame}{What is Synthetic Differential Geometry?}
		\begin{columns}
			\begin{column}{0.55\textwidth}
				\begin{itemize}
					\item Given two different points there exists only one line incident to both
					of them.
					\\ ~\\
					\item Through a point not in a line, only one parallel line to the given one can be drawn. 
				
				\end{itemize}
			\end{column}
			\begin{column}{0.45\textwidth}  %%<--- here
				\begin{center}
					\includegraphics[width=0.45\textwidth]{Euklid.jpg}
				\end{center}
			\end{column}
		\end{columns}
		\begin{center}
			But what are points and lines?
		\end{center}	
\end{frame}

\begin{frame}{What is Synthetic Differential Geometry?}
	\begin{columns}
		\begin{column}{0.55\textwidth}
			\begin{itemize}
				\item It is an axiomatic theory that deals with space forms in terms of their structure.
				\\~\\
				\item It allows for rigorous reasoning with nilpotent infinitesimals.
			\end{itemize}
		\end{column}
		\begin{column}{0.45\textwidth}
			\begin{center}				
				\includegraphics[width=0.45\textwidth]{Lie.jpg}
			\end{center}
		\end{column}
	\end{columns}
\end{frame}

\begin{frame}{Where does it take place?}
	We work in an ambient category $\mathcal{E}$ composed of ``smooth spaces and morphisms". \\~\\
	Synthetic differential geometry has no models in the category of sets. It has to be interpreted
	over a \textbf{topos}.
	\begin{itemize}
		\item Cartesian closed category with sub-object classifier. \\
		$A\times B$, $A^B$, $A\cup B$, $A\cap B$, $P(A)$ \dots
		
	\end{itemize}
\end{frame}

\section{Basic structure of the geometric line. The Kock-Lawvere axiom.}
\begin{frame}{Basic structure of the geometric line.}
	The geometric line $R$ satisfies:  \\~\\
	\begin{axiom}{0} $R$ is a non-trivial $\mathbb{Q}$-algebra.
	\end{axiom}
	\begin{center}
		\includegraphics[width=0.6\textwidth]{Compass.png}
	\end{center}
\end{frame}

\begin{frame}{The Kock-Lawvere axiom.}
\[ f(x) \simeq f(0) + f^\prime(0)x \]
\\~\\
\begin{itemize}
	\item Is there any $x$ such that we can substitute $\simeq$ with $=$ for all $f$'s? 
	\\~\\
	\pause
	\item No, we would need $x$ ``so small" that $x^2=0$.  
	
\end{itemize}
\end{frame}

\begin{frame}{The Kock-Lawvere axiom.}
Let 
\[
D:=\{ \, d\in R  \, | \, d^2=0 \,\}\] \\~\\

\begin{axiom}{1}[Kock-Lawvere axiom]
	For any $f:D\rightarrow R$ there exists a unique $b\in R$ such that
	\[\forall d\in D: \quad f(d)= f(0) + db. \]
\end{axiom}
\end{frame}

\begin{frame}{Wait, are we safe?}
Intuitionistically speaking, yes. \\~\\
Under classical assumptions, not so much. 
\end{frame}

\begin{frame}{Wait, are we safe?}
The Kock-Lawvere axiom is not consistent with the Principle of the Excluded Middle: \\~\\
\[ P\vee \neg P \]
\\~\\
We must use ``intuitionistic" logic. 
\end{frame}

\begin{frame}{Derivatives and Taylor series.}
	Derivatives are defined in a natural way. \\~\\~\\
	\begin{definition} 
		Let $f\in R^R$. The derivative of $f$ at the point $x\in R$ is the unique 
		constant $f^\prime(x)\in R$ such that
		\[\forall d\in D: \quad f(x+d)= f(x)+ f^\prime(x)d.\]
	\end{definition}
	And functions ``locally" coincide with their linear approximations. 
\end{frame}

\begin{frame}{Derivatives and Taylor series.} 
	D is not an ideal.\\ ~ \\
	What can we say  about
	\[ D_2:= \{ \, d\in R \, | \, d^3=0 \,  \}?\]
\end{frame}

\begin{frame}{Derivatives and Taylor series.} 
	Not much. 
	We need an additional axiom:
	\\~\\
	\begin{axiom}{$1_2$} For any $f\in R^{D_2}$ there exist unique $c_1,c_2\in R$ such that
		\[ \forall d\in D_2: \quad f(d) = f(0) + c_1d + c_2d^2  \]
	\end{axiom}  
\end{frame}

\begin{frame}{Derivatives and Taylor series.} 
	This way we have:
	\\~\\ 
	\begin{theorem} For any $f\in R^R$
		\[ \forall d \in D_2, \forall x\in R : \quad f(x + d)=f(0) + f^\prime(x)d + \frac{f^{\prime \prime}(x)}{2}
		d^2. \]
	\end{theorem}
\end{frame}

\begin{frame}{Derivatives and Taylor series.} 
	Similarly, we would need an additional axiom for any of
	\[ D_k := \{ \, d\in R \, | d^{k+1}=0 \, \} \text{ for } k=1,2,\dots  	\]
	\\~\\
	We can state them all together: \\~\\
	
	\begin{axiom}{$1^\prime$} Let $f\in R^{D_k}$ for some $k\in \N$. Then there exists 
		a unique $k$-tuple of constants $c_1,\dots, c_k\in R$ such that
		\[ \forall d\in D_k: \quad f(d)= f(0) + \sum_{i=1}^{k}c_k d^k \]
	\end{axiom}

\end{frame}

\begin{frame}{Derivatives and Taylor series.} 
	If we define $D_\infty=\bigcup_{i=0}^\infty D_i$ it follows\\~\\~\\
	\begin{theorem}[Taylor series] 
		For all $f\in R^R$ and $x\in R$ there exists a unique formal power series $\Phi(X)$
		such that
		\[ \forall d\in D_\infty:  \quad f(x+d) = \Phi(d).\]
		Namely, 
		\[ \Phi(X) = \sum_{i=0}^\infty \frac{f^{(k)}}{k!} X^k.	\] 		
		
	\end{theorem}
\end{frame}

\begin{frame}{Don't let the axiom-party stop.} 
	There are still families of infinitesimals we cannot deal with.
	\[ D(2):=\{\, (d_1,d_2)\in D^2 \, | \, d_1d_2=0 \, \} \]
	\\~\\ 
	Our current axioms state that
	\[ W_k:=R[X]/\langle X^{k+1} \simeq R^{D_k}, \text{ where } ``D_k=Spec_R(W_k)" \]
 
\end{frame}

\begin{frame}{Don't let the axiom-party stop.} 	
	In general, if $W=R[X_1,\dots,X_n]$ satisfies some technical condition (it is a \textbf{Weil algebra}) 
	we can state:
	\begin{axiom}{w}~
	\[W\simeq R^{Spec_R(W)} \]
	\end{axiom}
\end{frame}

\section{Vector fields}
\begin{frame}{Tangent Vectors}

	\begin{definition}
		A \textbf{tangent vector} to $M$ at the point $p\in M$ is a map 
		$t\in M^D$ such that $t(0)=p$.
	\end{definition}
	Thus, $M^D$ is the \text{tangent bundle} of $M$. 
	\\~\\ To give $(M^D)_p$ a tangent space structure we need $M$ to be ``infinitesimally linear".
	 

\end{frame}

\begin{frame}{Tangent Vectors}
	\begin{definition} \label{def:inflinear}
		An object $M$ is said to be \textbf{infinitesimally linear} if for any $p\in M$ and 
		any	$n$-tuple of maps $t_1,\dots, t_n \in (M^D)_p$ there is a unique map $l\in M^{D(n)}$ satisfying 
		$l\circ incl_i=t_i$ for all $i=1,\dots,n$. 
	\end{definition} 
\end{frame}

\begin{frame}{Differentials}

	\begin{theorem} \label{prof:difflinear}
		Let $M$ and $N$ be infinitesimally linear, and $f\in N^M$. Then, for
		any $p\in M$ the map $f^D(t)=f\circ t$ restricts to a linear map from $(M^D)_p$
		to $(N^D)_{f(p)}$.
		
	\end{theorem}

\end{frame}
\begin{frame}{Vector Fields}
\begin{definition} 
	A vector field $X$ over $M$ is any of the following: 
	\begin{itemize}
		\item A ``section of the tangent bundle",  $\hat{X}:M\rightarrow M^D$.
		\item An ``infinitesimal flow
		of the aditive group $R$" $X: M\times D \rightarrow M$
		\item An ``infinitesimal
		deformation of the identity map" $\check{X}:D\rightarrow M^M$.
	\end{itemize}
\end{definition}
\end{frame}


\begin{frame}{Directional Derivatives}
	\begin{definition} 
		The directional derivative of $f$ in the direction of $X$
		is the unique map $X(f)\in M^R$ such that, for any $p\in M$
		\[\forall d\in D: \quad f(X(p,d))= f(p) + dX(f)(p) .\]			
	\end{definition}
\end{frame}

\begin{frame}
Under some additional hypotheses on $M$ we can define:
\\~\\

\begin{definition}
	Let $X,Y\in Vect(M)$. The \textbf{Lie bracket} $[X,Y]$ is the unique vector 
	field such that
	\[\forall d_1, d_2 \in D: \quad[X,Y]^\vee (d_1d_2)= 
	\check{Y}(-d_2)\circ \check{X}(-d_1)\circ
	\check{Y}(d_2)\circ \check{X}(d_1) \]
\end{definition}
~\\
And it is satisfied 
\[ [X,Y](f)= X(f) - Y(f) \]
\end{frame}

\begin{frame}

\begin{center}
	\Large{Questions?}\\~\\~\\~\\~\\
	\tiny{Let them be easy please}
\end{center}

\end{frame}

\end{document}