\documentclass[12pt,notitlepage,a4paper]{article}

\usepackage[left=2.5cm,right=2.5cm,top=2.5cm,bottom=2.5cm]{geometry}
\usepackage{graphicx}
\usepackage{amssymb,mathtools, amsmath, amsfonts, amsthm}
%\usepackage{color}
\usepackage{float}
\usepackage{hyperref}
\usepackage{enumerate}
\usepackage{enumitem}
\usepackage{chngcntr}
\usepackage{cleveref}
\usepackage{pdfpages}

\usepackage{mathptmx}

\counterwithout{equation}{section}

\newlength{\margen}
\setlength{\margen}{\paperwidth}
\addtolength{\margen}{-\textwidth}
\addtolength{\skip\footins}{0.7 cm}
\setlength{\margen}{0.5\margen}
\addtolength{\margen}{-1in}
\setlength{\oddsidemargin}{\margen}
\setlength{\evensidemargin}{\margen}
\setlength{\abovedisplayskip}{3pt}
\setlength{\belowdisplayskip}{3pt}
%%%% Small setup %%%%
\hypersetup{
	colorlinks=false,
	pdfborder={1 1 0.0005},
}
\setlength{\parskip}{0.2cm}
%%%%%%%%%%%%%%%
\usepackage{tikz-cd}
\usetikzlibrary{cd}
\usepackage[english]{babel}
\usepackage{todonotes}
\usepackage{cleveref}
\usepackage{caption}
\usepackage{subcaption}
\usepackage{bbding}
\usepackage{tcolorbox}
\usepackage{natbib}


\theoremstyle{definition}
\newtheorem{proposition}{Proposition}[section]
\newtheorem{fact}{Fact}[section]
\newtheorem{theorem}{Theorem}[section]
\newtheorem{obs}{Observation}[section]
\newtheorem{lemma}{Lemma}[section]
\newtheorem{corollary}{Corollary}[section]
\newtheorem{definition}{Definition}[section]
\newtheorem{propdef}{Proposition / Definition}[section]
\newtheorem{remark}{Remark}[section]


\newcommand{\cc}{\mathfrak{c}}
\newcommand{\Z}{\mathbb{Z}}
\newcommand{\CC}{\mathbb{C}}
\newcommand{\Q}{\mathbb{Q}}
\newcommand{\R}{\mathbb{R}}
\newcommand{\N}{\mathbb{N}}
\newcommand{\Hc}{\mathcal{H}}
\newcommand{\Lan}{\mathcal{L}}
\newcommand{\Ln}{\lim\limits_{n\to \infty}}
\newcommand{\clist}{\mathfrak{c}_{1}, \cdots, \mathfrak{c}_m}
\newcommand{\morph}[1]{\sim_#1}
\newcommand{\vlst}[2]{#1_1,\dots, #1_{#2}}
\newcommand{\gnp}{G(n,\beta_1/n^{a_1-1}, \dots,\beta_l/n^{a_l-1})}
\newcommand{\overbar}[1]{\mkern 1.5mu\overline{\mkern-1.5mu#1\mkern-1.5mu}\mkern 1.5mu}
\newcommand{\ehr}{\textsc{Ehr}}
\newcommand{\PR}[1]{\mathrm{Pr}\big(#1\big)}
\newcommand{\Tr}{\mathrm{Tr}}
\newcommand{\InR}[1]{\left\{ #1_R \right\}_{R\in \sigma}}
\newcommand{\sep}{\noindent\rule{2cm}{0.4pt}}
\newcommand{\aut}{\mathrm{aut}}

\DeclarePairedDelimiter\floor{\lfloor}{\rfloor}


\begin{document}
\begin{abstract}
	We extend the convergence law for sparse random graphs proven
	by Lynch to arbitrary relational languages.
	We consider a finite relational vocabulary $\sigma$
	and a first order theory $T$ for $\sigma$ 
    composed of symmetry and 
    anti-reflexivity axioms. We define a binomial random model of finite 
    $\sigma$-structures that satisfy $T$ and show that first order properties have 
    well defined asymptotic probabilities when the expected number of tuples satisfying
    each relation in $\sigma$ is linear.
    It is also shown that those limit probabilities are well-behaved with
    respect to some parameters that represent the density of tuples satisfying $R$
    for each relation $R$ in the vocabulary $\sigma$. 
    An application of these results to the problem of random Boolean 
    satisfiability is presented afterwards. 
    We show that in a random $k$-CNF formula over $n$ variables where
    each possible clause occurs with probability $\sim c/n^{k-1}$ independently any 
    first order property of $k$-CNF formulas that
    implies unsatisfiability does almost surely not hold as $n$ tends to infinity.
   
\end{abstract}
\clearpage

\section*{Introduction}

\todo[inline]{INCOMPLETA, PENDIENTE DE SER REVISADA}
Since the work of Erd\"os and R\`enyi on the evolution of random graphs
\cite{erdHos1960evolution} the study of the asymptotic properties of random
structures has played a relevant role in combinatorics and computer science.
A central theme in this topic is, given a succession $(G_n)_n$ of random
structures of some sort and a property $P$, to determine the limit probability
that $G_n$ satisfies $P$ or to determine whether that limit exists. \par
One approach that has proven to be useful is to classify the properties $P$
according to the logical languages they can be defined in. We say that the 
succession $(G_n)_n$ obeys a convergence law with respect to some logical language
$\mathcal{L}$ if for any given property $P$ expressible in $\mathcal{L}$ the 
probability that $G_n$ satisfies $P$ tends to some limit as $n$ grows to infinity.
We say that $(G_n)_n$ obeys a zero-one law with respect to $\mathcal{L}$ if
that limit is always either zero or one. 
The seminal theorem on this topic, due to Fagin \cite{fagin1976probabilities}
and Glebskii et al. 
\cite{glebskii1969range} independently, states that if $G_n$ denotes a labeled
graph with $n$ vertices picked uniformly at random among all $2^{\binom{n}{2}}$ 
possible then $(G_n)_n$ satisfies a zero-one law with respect to the first order
(FO) language of graphs. \par
Originally this result was proven in the broader context of relational
structures but it was in the theory of random graphs were the study of 
other zero-one and convergence laws became more prominent. In particular,
the asymptotic behavior of FO logic in the binomial model of random graphs 
$G(n,p)$ has been extensively studied. In this model, introduced by Gilbert 
\cite{gilbert1959random}, a random graph is obtained from $n$ labeled vertices
by adding each possible edge with probability $p$ independently. When $p=1/2$
this distribution of random graphs coincides with the uniform one, mentioned 
above. In general, for the case where $p$ is a constant probability a slight 
generalization of the proofs in \cite{fagin1976probabilities} and 
\cite{glebskii1969range} works and $G(n,p)$ satisfies a zero-one law 
for FO logic. If we consider $p(n)$ a decreasing function of the form
$n^{-\alpha}$ we can ask the question of what are the values of $\alpha$
for which $G(n,p(n))$ obeys a zero-one or a convergence law for FO logic. 
In \cite{shelah1988zero} Shelah and Spencer gave a complete answer 
for the range $\alpha\in (0,1)$. Among other results, they proved that
if $\alpha$ is an irrational number in this interval then
$G(n,p(n))$ obeys a zero-one law for FO logic, while if $\alpha$
is a rational number in the same range then $G(n,p(n))$ does not
even satisfy a convergence law for FO logic. The case $\alpha=1$
was later solved by Lynch in \cite{lynch1992probabilities}. A weaker
form of the main theorem in that article states the following:
\begin{theorem} 
	For any FO sentence $\phi$, the function
	$F_\phi: (0,\infty)\rightarrow [0,1]$ given by 
	\[ F_\phi(\beta) = \Ln \mathrm{Pr}\big( \mathrm{G}(n,\beta/n)
	\text{ satisfies } \phi   \big) \]
	is well defined and analytic. In particular, for any
	$\beta \geq 0$ the model $\mathrm{G}(n,\beta/n)$
	obeys a convergence law for FO logic. 
\end{theorem}

The analyticity of these asymptotic probabilities with respect to 
the parameter $\beta$ implies that FO properties cannot "capture" sudden
changes that occur in the random graph $\mathrm{G}(n,\beta/n)$
as $\beta$ changes. Given $p(n)$ a probability, $P$ a property of graphs,
and $Q$ a sufficient condition for $P$ - i.e., a property that 
implies $P$ -, we say that $Q$ explains $P$ if $\mathrm{G}(n,p(n))$ satisfies 
the converse implication $P \implies Q$ asymptotically almost surely
(a.a.s.). A notable example of this phenomenon happens in the range 
$p(n)= \log(n)/n + \beta/n$ with $\beta$ constant. Erd\"os and R\`enyi
\cite{erdHos1960evolution} 
showed that for probabilities of this form $\mathrm{G}(n,p(n))$ 
a.a.s. is disconnected only if it contains an isolated vertex. 
An observation by Albert Atserias is the following:

\begin{theorem}
Let $c$ be a real constant such that 
$\Ln \mathrm{Pr}\big( G(n, c/n) 
\text{ is not }	3\text{-colorable }\big)> 0$.  
Then there is no FO graph property that explains 
non-$3$-colorability for $G(n, c/n)$.
\end{theorem}

The short proof of this theorem is as follows: 
It is a known fact that there are positive constants
$c_0\leq c_1$ such that $G(n, c/n)$ is a.a.s $3$-colorable
if $c<c_0$ and it is a.a.s non $3$-colorable if $c>c_1$ 
REFERENCES NEEDED.
Suppose $P$ is a FO graph property that implies
non-$3$-colorability. Then, because of this implication,
for all values of $c$
\[\Ln \mathrm{Pr}\big( G(n, c/n) 
\text{ satisfies } P \big) \leq \Ln \mathrm{Pr}\big( G(n, c/n) 
\text{ is not }	3\text{-colorable}\big).\]
In consequence the asymptotic probability that $G(n,c/n)$ 
satisfies $P$ is zero when $c<c_0$. By Lynch's theorem, if 
$P$ is definable in FO logic then this asymptotic probability
varies analytically with $c$. Using the fact that any analytic 
function that takes value zero in a non-empty interval must 
equal zero everywhere, we obtain that $G(n,c/n)$ a.a.s does 
not satisfy $P$ for any value of $c$. 
As a consequence the theorem follows.  \par

The aim of this work is to extend Lynch's result to arbitrary 
relational structures were the relations are subject to some
predetermined symmetry and anti-reflexivity axioms. This was 
originally motivated by an application to the study of random
$k$-CNF formulas. Since \cite{chvatal1992mick} it is known that 
for each $k$ there are constants $c_0,c_1$ such that a random 
$k$-CNF formula with $cn$ clauses over $n$ variables 

%Besides this 
% 
%This question was originally posed in \cite{atserias2005definability}
%by Albert Atserias where the answer was given for various probability





\setcounter{section}{0}

\section{Preliminaries}

\subsection{General notation}\label{subsect:notation}

Given a positive natural number $n$, we will write
$[n]$ to denote the set ${1,2,\dots,n}$.\par
Given a set $S$ and a natural number $k\in \N$
we will use $\binom{S}{k}$ to denote the set of 
subsets of $S$ whose size is $k$. \par
Given numbers, $n,m\in \N$ with $m\leq n$ we denote by
$(n)_m$ the $m$-th falling factorial of $n$. \par
We will use the convention that over-lined variables
, like $\overline{x}$, denote ordered tuples of arbitrary length.
Given an ordered tuple $\overline{x}$
we define the number $len(\overline{x})$ as its length. 
Given a tuple $\overline{x}$ and an element $x$ the expression
$x\in \overline{x}$ means that $x$ appears as some coordinate
in $\overline{x}$. 
Given a map $f:X\rightarrow Y$ between two sets $X, Y$ and 
an ordered tuple $\overline{x}:=(x_1,\dots,x_a)\in X^*$ 
we define $f(\overline{x})\in Y^*$ as the tuple 
$(f(x_1),\dots,f(x_a))$.
Given two tuples $\overline{x},\overline{y}$
we write $\overline{x}^\smallfrown \overline{y}$ to denote their 
concatenation.
\par

Given a set
$S$ and a number $n\in \N$ with $n\leq |S|$ we define
$(S)_n$ as the subset of $S^n$ consisting of the $n$-tuples
whose coordinates are all different. 
We also define $S^*:=\bigcup_{n=0}^\infty S^n$ and
$(S)_*:=\bigcup_{n\leq |S|} (S)_n$. \par

Let $S$ be a set, $a$ a positive natural number, 
and $\Phi$ a group of permutations over 
$[a]$. Then $\Phi$ acts naturally over
$S^a$ in the following way: Given $g\in \Phi$ and
$\overline{x}:=(x_1,\dots,x_a)\in S^a$ we define 
$g\cdot \overline{x}$ 
as the tuple $(x_{g(1)},\dots,x_{g(a)})$. 
We will denote by $S^a/\Phi$ to the quotient
of the set $S^a$ by this action. Given an element
$\overline{x}:=(x_1,\dots, x_a)\in S^a$ we will denote its equivalence
class in $S^a/\Phi$ by $[x_1,\dots,x_a]$ or $[\overline{x}]$.
Thus, for any
$g\in \Phi$, by definition $[x_1,\dots,x_a]=[x_{g(1)}
,\dots,x_{g(a)}]$. \par
The notations $\overline{x}$ and
$(x_1,\dots, x_a)$ 
will be reserved 
to ordered tuples while 
$[\overline{x}]$ and
$[x_1,\dots,x_a]$ will denote ordered tuples modulo the
action of some arbitrary group of permutations. Which group is
this will depend on the ambient set where $[x_1,\dots,x_a]$ belongs
and it should either be clear from context or not be relevant.\par

Given two real functions over the natural numbers 
$f,g:\N \rightarrow \R$ we will write $f=O(g)$ to 
mean that there exists some constant $C\in \R$
such that $f(n)\leq Cg(n)$ for $n$ sufficiently large, 
as usual. We will write $f=\Theta(g)$ if both $f=O(g)$ and
$g=O(f)$. If $g(n)\neq 0$ for $n$ large enough then
we will write $f\sim g$ when $\Ln \frac{f(n)}{g(n)}=1$. 

%\todo[inline]{Añadir referencias y trabajo relacionado con random SAT}



\subsection{Logical preliminaries}
We assume familiarity with first order logic (FO). We follow 
the convention that first order logic contains the equality symbol. 
Given a vocabulary $\sigma$ we will denote by $FO[\sigma]$ the set of 
first order formulas of vocabulary $\sigma$. We define the set of 
\textbf{free variables} of a formula as usual. 
Given a relation symbol $R\in \sigma$ we denote by $ar(R)$ the arity of $R$. 
Given a formula $\phi\in FO[\sigma]$ we will use the notation $\phi(\overline{y})$ 
to denote that $\overline{y}$ is a tuple of 
(different) variables that contains all free variables in $\phi$ and
none of its bounded variables, although it may contain variables
which not appear in $\phi$.
Formulas with no free variables are called \textbf{sentences} and 
formulas with no quantifiers are called \textbf{open formulas}. 


\subsection{Structures as multi-hypergraphs} \label{sect:structures}

For the rest of the article consider fixed:
\begin{itemize}
	\item A relational vocabulary $\sigma$ such 
	that all the relations $R\in\sigma$ satisfy $ar(R)\geq 2$. 
	\item 
	Groups $\{ \Phi_R \}_{R\in \sigma}$
	such that each $\Phi_R$ is consists of 
	permutations on $[ar(R)]$ with the usual 
	composition as its operation.	
	\item 
	Sets $\{P_R\}_{R\in \sigma}$ satisfying that for 
	all $R\in \sigma$, 
	$P_R\subseteq \binom{[ar(R)]}{2}$
	
	
\end{itemize}
%
%We will only consider relational structures where the relations
%are of arity at least two. We claim that it is also possible to prove
%our results for structures that include unary relations but we will refrain from
%doing so because it would add unnecessary complexity to our proofs. \par

%We will denote by $t_u$ the, possibly zero,
%amount of unary
%relation symbols in $\sigma$. 
%For the sake of convenience we may assume that $R_i$'s are
%ordered in such a way that the unary relation symbols, if any, 
%are the first ones $R_1,\dots, R_{t_u}$. This will
%become relevant later. \par 


We define the class $\mathcal{C}$ as the 
class of $\sigma$-structures that
satisfy the 
following axioms: 
\begin{itemize}
	\item \textit{Symmetry axioms}: For each $R\in \sigma$ and
	each $g\in \Phi_R$:
	\[ \forall \overline{x}:=x_1,\dots, x_{ar(R)} \big(  R(\overline{x})
	\iff R(g\cdot\overline{x}) \big)    \]
	\item \textit{Anti-reflexivity axioms}: For each 
	$R\in \sigma$ and $\{i,j\}\in P_R$
	\[ \forall x_1,\dots, x_{ar(R)} 
	\big( (x_i=x_j) \implies \neg R(x_1,\dots, x_{a_s})
	\big)\]
\end{itemize}

Structures in $\mathcal{C}$ generalize the usual notion of hypergraph
in the sense that they contain multiple ``adjacency'' relations with arbitrary 
symmetry and anti-reflexivity axioms. \par
We will use the usual graph theory nomenclature and notation with some minor changes. 
In the scope of this article we will call \textbf{hypergraphs}
to structures in $\mathcal{C}$.
Given an hypergraph $G$ we will call its \textbf{vertex set}, denoted by $V(G)$,
to its universe.\par
In order to define the edge sets of $G$ we need the following auxiliary definition
\begin{definition} 
	Let $V$ be a set, and let $R\in \sigma$.
	We define the \textbf{set of possible edges over $V$
	given by $R$} as
	\[ E_R[V]= (V^{ar(R)}/\Phi_R) \quad \setminus 
	\quad X, \]
	where
	\[
	X=
	\Big\{ [v_1,\dots,v_{ar(R)}]  
	\quad \Big| \quad
	v_1,\dots,v_{ar(R)}\in V, \,
	\text{ and } 
	 \, v_i=v_j \text{ for some } 
	\{i,j\}\in P_R \Big\}.
	\]
	We will call \textbf{edges} to the elements of
	$E_R[V]$ and we will say that the \textbf{sort} of any edge $e\in E_R[V]$
	is $R$.	In the case where $V=[n]$ we will write simply $E_R[n]$ instead
	of $E_R[[n]]$
\end{definition}

That is, $E_R[V]$ contains all the ``$ar(R)$-tuples of elements
in $V$ modulo the permutations
in $\phi_R$" 
excluding those that contain some repetition of elements in
the positions given by $P_R$.\par
Let $G$ be an hypergraph whose set of vertices is $V$ and let $R\in \sigma$
be a relation. 
We define the \textbf{edge set of $G$ given by $R$}, denoted by $E_R(G)$, 
as the set of edges $[\overline{v}]\in E_R[V]$ such that $\overline{v}\in R^G$. 
We define \textbf{the total edge set of $G$} as the set $E(G):=\cup_{R\in \sigma} E_R(G)$.
Given an edge, $e\in E(G)$ we will denote by $V(e)$
the set of all vertices that participate in $e$. 
\par
Clearly an hypergraph $G$ is completely given by its vertex set $V(G)$ and its edge 
set $ E(G)$. Notice that edges $e\in E(G)$ are sorted according to the relation they represent.
\par

Given two hypergraphs $H$ and $G$ we say that $H$
is a \textbf{sub-hypergraph} of $G$, which we write as $H\subset G$,
if $V(H)\subset V(G)$ and $E(H)\subset E(G)$ (notice 
that this is equivalent to $E_R(H)\subset E_R(G)$ for all
$R\in \sigma$, since the edges are sorted).\par


Given a set of vertices $U\subseteq V(G)$, 
we will denote by $G[U]$ the \textbf{hypergraph induced
by $G$ on $U$}. That is, $G[U]$ is an hypergraph
$H=(V(H),\{E(H)_R\}_{R\in \sigma})$ such that 
$V(H)=U$ and for any $R\in \sigma$ 
an edge $e\in E_R(G)$ belongs 
to $E_R(H)$ if and only if $V(e)\subset U$.
\par
%
%An homomorphism between two hypergraphs $G$ and $H$ 
%a map $f:V(G)\rightarrow V(H)$ that sends edges 
%from $G$ to edges in $H$ of the same color. That is,
%if vertices $v_1,\dots, v_{a_i}$ form an edge 
%$[v_1,\dots,v_{a_i}]\in E_i(G)$, then
%$[f(v_1),\dots,f(v_{a_i})]\in E_i(H)$.
%If $f$ is injective then it is called a monomorphism. 
%If $f$ is bijective and its inverse is also an homomorphism 
%between $H$ and $G$ then $f$ is called an isomorphism.\par
%
%The group of automorphisms $Aut(G)$ of an hypergraph $G$ is the group
%of isomorphisms between $G$ and itself. 

%
%Given two hypergraphs $G$ and $H$, a copy of $H$ in $G$ is
%a sub-hypergraph $H_2\subseteq G$ isomorphic to $H$. The copy
%is called induced if $H_2$ is an induced sub-hypergraph.
%We will call a labeled copy
%of $H$ in $G$ to a monomorphism $f:H\rightarrow G$. 
%It is satisfied that the number
%of labeled copies of $H$ in $G$ is $\aut(H)$ times the number
%of copies of $H$ in $G$.


We define the \textbf{excess} $ex(G)$ of an hypergraph $G$ as the number
\[
ex(G):= \big(\sum_{R\in \sigma} (ar(R)-1)|E_R(G)|\big) - |V(G)|.  
\] 
That is, the excess of $G$ is its "weighted number of edges"
minus its number of vertices. \par
An hypergraph $G$ is \textbf{connected} if for any two vertices $v,u\in V(G)$
there is a sequence of edges $e_1,\dots, e_m\in E(G)$ such that
$v\in V(e_1), u\in V(e_m)$ and for each $i\in [m-1]$, 
$V(e_i)\cap V(e_{i+1})\neq \emptyset$. It holds that
$ex(G)\geq -1$ for any connected hypergraph.
\par
A connected hypergraph $G$ is a path between two of its 
vertices $v,u\in V(G)$ if $G$ 
does not contain any connected proper sub-hypergraph containing both $v,u$.
\par
A connected hypergraph $G$ is a \textbf{tree} if $ex(G)=-1$ and \textbf{dense} if $ex(G)>0$.
A connected hypergraph $G$ with $ex(G)\geq 0$ is called \textbf{saturated} 
if for any non-empty proper sub-hypergraph
$H\subset G$ it holds $ex(H)<ex(G)$. 
A connected hypergraph $G$ with $ex(G)=0$ is called \textbf{unicycle}. 
A saturated unicycle is called a \textbf{cycle}. \par
Given an hypergraph $G$ we define the following metric, $d$, over $V(G)$:
\[ d^G(u,v)= \min_{\substack{H \subset G\\ 
		H \text{ connected }\\
		u,v\in V(H)}} |E(H)| .\]
That is, the \textbf{distance} between $v$ and $u$ is the minimum number of
edges necessary to connect $v$ and $u$. 
If such number does not exist we define $d^G(u,v)=\infty$. 
When $G$ 
is understood or not relevant we will usually simply denote the 
distance by $d$ instead of $d^G$. Equivalently, the distance $d$
coincides with the usual one defined over the Gaifman graph of the structure 
$G$. The \textbf{diameter} of an hypergraph is the maximum distance between any 
two of its vertices. 
We extend naturally the distance $d$ to sets and tuples of
vertices, as usual. Given a vertex/set/tuple $X$ and a number
$r\in \N$ we define the \textbf{neighborhood}
$N^G(X;r)$, or simply $N(X;r)$ when $G$ is not relevant,
as the set of vertices $v$ such that $d^G(X,v)\leq r$.
\par

\textbf{Isomorphisms} between hypergraphs are defined the same as 
isomorphisms between relational structures. We denote the isomorphism relation
between hypergraphs by $\simeq$. 
Given and hypergraph $H$, an \textbf{automorphism} of $H$
is an isomorphism from $H$ to itself. We will denote by
$\aut(H)$ the number of such automorphisms. \par

Let $H$ be an hypergraph and let $V$ be a set. We define the
set of \textbf{copies of $H$ over $V$}, denoted as $Copies(H,V)$, 
as the set of hypergraphs 
$H^\prime$ such that
$V(H^\prime)\subset V$ and $H\simeq H^\prime$.   \par

In our proofs it will be useful to consider colorings over 
hypergraphs as a way to decorate their 
vertices with extra information. 
Let $\Sigma$ be a set. A \textbf{$\Sigma$-hypergraph}
is a pair $(H, \chi)$ where $H$ is an hypergraph
and $\chi: V(H)\rightarrow \Sigma$ is a map 
called \textbf{$\Sigma$-coloring} of $H$. \par
\textbf{Isomorphisms} between $\Sigma$-hypergraphs are just isomorphisms between the underlying
hypergraphs that also preserve their colorings. We also denote the isomorphism relation between 
$\Sigma$-hypergraphs by $\simeq$.
Given a $\Sigma$-hypergraph $(H,\chi)$,  an 
\textbf{automorphism} of $(H,\chi)$ is an 
isomorphism from it into itself, as before. We will denote
by $\aut(H,\chi)$ the number of such automorphisms. 
\par

Let $(H,\chi)$ be a $\Sigma$-hypergraph and let $V$ be a set.
As before, we define the set $Copies\big(
(H,\chi),\, \, V\big)$ as the set of $\Sigma$-hypergraphs
$(H^\prime,\chi^\prime)$ satisfying $V(H^\prime)\subset V$ and
$(H,\chi)\simeq (H^\prime,\chi^\prime)$. Let $\mathbb{H}$ be 
an isomorphism class of $\Sigma$-hypergraphs. Then the set
$Copies(\mathbb{H}, V)$ is defined as the set of $\Sigma$-hypergraphs
$(H^\prime,\chi^\prime)$ such that 
$V(H^\prime)\subset V$ and
$(H^\prime,\chi^\prime)\in \mathbb{H}$. 
Let $v\in V$ and $s\in \Sigma$. We define the
set $Copies\big(\mathbb{H}, V;\,\, (v,s)\big)$ 
as the set of $\Sigma$-hypergraphs
$(H^\prime,\chi^\prime)\in Copies(\mathbb{H}, V)$
that satisfy $v\in V(H^\prime)$ as well as
$\chi^\prime(v)=s$. \par

Given $\mathbb{H}$ an isomorphism class of hypergraphs or $\Sigma$-hypergraphs
, we define expressions
such as $ex(\mathbb{H})$, $\aut(\mathbb{H})$,
$|V(\mathbb{H})|$, $|E(\mathbb{H})|$ or
$Copies(\mathbb{H},V)$ via representatives of $\mathbb{H}$.\par 

\subsection{The random model} \label{sect:random}

For each $R\in \sigma$ let
$p_R$ be a real number between zero and one.
The random model $G^{\mathcal{C}}\Big(n,\InR{p}\Big)$ 
is the discrete probability space that
assigns to each hypergraph $G$ whose vertex
set $V(G)$ is $[n]$ the following probability:

\[ \mathrm{Pr}(G)=\prod_{R\in \sigma} p_R^{|E_R(G)|}
(1-p_R)^{ \big|E_R[n]\big|-\big|E_R(G)\big|}.	
\]
Equivalently, this is the probability space obtained by 
assigning to each edge $e\in E_R[n]$ probability 
$p_R$ independently for each $R\in \sigma$. \par

%Let $G_n$ denote a random hypergraph in 
%$G^\mathcal{C}(n,\overline{p})$, where 
%$\bar{p}:=p_1,\dots,p_t$, and 
%we allow each $p_i$ to depend on $n$. 
As in the case of Lynch's theorem, we are interested in the
"sparse regime" of $G^\mathcal{C}(n,\overline{p})$, were the 
expected number of edges each sort is linear. 
%In other words, we want that
%$\mathrm{E}[m_i(G_n)]=\Theta(n)$ for all $i$'s.
This is achieved when each of the $p_R$'s are 
of the form $\beta_R/n^{ar(R)-1}$ for some 
positive real numbers $\InR{\beta}$.
We will denote a random sample of 
$G^\mathcal{C}\left(n,\left\{p_R\right\}_{R\in\sigma}\right)$
by $G_n\left(\InR{\beta}\right)$ when the probabilities $p_R$
satisfy $p_R(n)\sim \beta_R/n^{ar(R)-1}$ for all $R\in \sigma$.
When the choice of $\InR{\beta}$ is not relevant
we will write $G_n$ instead of 
$G_n\left(\InR{\beta}\right)$.\par

Our goal is to prove the following theorem:


\begin{theorem} \label{thm:main}
	Let $\phi$ be a sentence in $FO[\sigma]$. Then
	the function
	%$F_\phi: [0,1]^{t_u}\times [0,\infty)^{t-t_u}
	$F_\phi: [0,\infty)^{|\sigma|}
	\rightarrow \R$ given by 
	\[
	\InR{\beta} \mapsto \Ln Pr\big( G_n\left(
	\InR{\beta}\right) \models \phi\big)
	\]
	is well defined and analytic. 
\end{theorem}



\subsection{Ehrenfeucht-Fraisse Games}

We assume familiarity with Ehrenfeucht-Fraisse (EF) games.
An introduction to the subject can be found in
\cite[Section 2]{finitemodeltheory1}, for example. Given
hypergraphs $H_1$ and $H_2$ we denote the $k$-round EF game played 
on $H_1$ and
$H_2$ by $\ehr_k(H_1;H_2)$.
The following is satisfied:

\begin{theorem}
	[Ehrenfeut, \citealp{ehrenfeucht1961application}] Let
	$H_1$ and $H_2$ be hypergraphs.
	Then Duplicator wins $\ehr_k(H_1;H_2)$
	if and only if $H_1$ and $H_2$ satisfy the same 
	sentences $\phi\in FO[\sigma]$ with $qr(\phi)\leq k$.		
\end{theorem}

Given lists $\overline{v}\in V(H_1)^*$, and 
$\overline{u}\in V(H_2)^*$ of the same length, 
we denote the $k$ round 
Ehrenfeucht-Fraisse game on $H_1$ and $H_2$ with initial position given
by $\overline{v}$ and $\overline{u}$ by $\ehr_k(H_1,\overline{v};H_2,\overline{u})$.\par

We also define the $k$-round distance Ehrenfeucht-Fraisse game on 
$H_1$ and $H_2$, denoted by $d\ehr_k(H_1;H_2)$, the same way as
$\ehr_k(H_1;H_2)$, but now in order for Duplicator to win the
game the following additional condition has to be satisfied 
at the end: For any $i,j\in [k]$, $d^{H_1}(v_i,v_j)=d^{H_2}(u_i,u_j)$,
where $v_s$ and $u_s$ denote the vertex played on $H_1$, resp. $H_2$ in the 
$s$-th round of the game. 
Given $\overline{v}\in V(H_1)^*$, and $\overline{u}\in V(H_2)^*$
lists of vertices of the same length,
we define the game 
$d\ehr_k(H_1,\overline{v};H_2,\overline{u})$ analogously to 
$\ehr_k(H_1,\overline{v};H_2,\overline{u})$.



\subsection{Outline of the proof}

We show now an outline of the proof of \cref{thm:main}. \par 
The arguments mirror the ones in the proof of theorem 2.1
in \cite{lynch1992probabilities}, adapted to fit our context. Fix $r\in \N$.
The following facts hold:
\begin{itemize}
	\item A.a.s $G_n$ does not contain any dense hypergraph of diameter at most 
	$2r+1$ (\cref{thm:sparse}). In consequence, all $r$-neighborhoods in $G_n$
	a.a.s are either trees or unicycles.
	\item Given any fixed vertices $v_1,\dots, v_m\in\N$, a.a.s 
	$d(v_i,v_j)>2r+1$ for any two $v_i,v_j$ and a.a.s all the $N(v_i;r)$'s 
	are trees (\cref{lem:disjointtrees}).
\end{itemize} 
For any given $k\in \N$, we define an equivalence relation $\sim_k$ for 
hypergraphs (\cref{sect:equivtrees}, and \cref{sect:equivunicycles})
in a way that $H_1\sim_k H_2$ implies that Duplicator wins $d\ehr_{k}(H_1;H_2)$.\par
Given $r\in \N$, we define the \textbf{$r$-core} of an hypergraph $H$, 
written as $Core(H;r)$, the union of the $r$-neighborhoods of
all saturated sub-hypergraphs of $H$ whose diameter is at most $2r+1$.
We say that $H$ is $r$-simple if all connected components of
$Core(H;r)$ are unicycles (\cref{sec:Core}) \par
For any given $k,r\in \N$ we say that $H_1\approx_{k,r} H_2$ for two
hypergraphs if $Core(H_1;r)$ and $Core(H_2;r)$ contain ``the same number
up to $k$" of connected components of each $\sim_k$ class (\cref{def:agreeability}).\par
Given $k,r\in \N$ we say that an hypergraph $H$ is $(k,r)$-rich if, informally, 
it has enough ``$r$-neighborhoods that are trees" of each $\sim_k$ class which
are sufficiently far from each other and sufficiently far from any small saturated
graph (\cref{def:rich}). \par
We prove the following facts:
\begin{itemize}
	\item[(1)] Let $k\in \N$ and let $r:=(3^k-1)/2$. Let $H_1$ and $H_2$
	be $(k,r)$-rich hypergraphs satisfying $H_1\approx_{k,r} H_2$. Then
	Duplicator wins $\ehr_{k}(H_1; H_2)$ (\cref{thm:Duplicatorwins}).
	\item[(2)] Let $r\in \N$. Then a.a.s $G_n$ is $r$-simple (\cref{cor:simple}).
	\item[(3)] Let $k,r\in \N$. Then a.a.s $G_n$ is $(k,r)$-rich (\cref{thm:rich}).
	\item[(4)] Let $k,r \in \N$. Let $\mathcal{O}$ be a $\approx_{k,r}$ class
	of $r$-simple hypergraphs. Then 
	\[
	\Ln \Pr\left(G_n\left(\{\beta_R\}_{R\in \sigma}\right)\in \mathcal{O}\right)
	\]
	exists
	and is an analytic expression in $\{\beta_R\}_{R\in \sigma}$ 
	(\cref{thm:agreeabilityprobabilities}).
\end{itemize}
Then an sketch of the proof of the main theorem \cref{thm:main}, given in \cref{sect:main},
 is the following:
Let $\Phi\in FO[\sigma]$ be a sentence and let $k:=qr(\Phi)$, $r:=(3^k-1)/2$.
Because of (1) and (3) it holds that for any $\approx_{k,r}$ class $\mathcal{O}$
\[
\Ln \Pr\left(G_n \models \Phi\, \big| \, G_n\in \mathcal{O} \right)= \text{ $0$ or $1$ }.
\]
This together with (3) and the fact that there are a finite number of
$\approx_{k,r}$-classes of $r$-simple hypergraphs imply that $\Ln \PR{G_n \models \Phi}$
equals a finite sum of limits of the form $\Ln \PR{G_n \in \mathcal{O}}$
where $\mathcal{O}$ is some $\approx_{k,r}$-class of $r$-simple hypergraphs.
Finally, using (4) we get that $\Ln \PR{G_n \models \Phi}$ exists and 
is an analytic expression in $\{\beta_R\}_{R\in \sigma}$, as we wanted. 
%We show that for any quantifier rank $k$ there are some classes of
%hypergraphs 
%$C^k_1,\dots, C^k_{n_k}$ such that
%\begin{itemize}[noitemsep, topsep=0pt]
%	\item[(1)] a.a.s the rank $k$ type of any two graphs in the same class coincide, 
%	\item[(2)] a.a.s. any random graph belongs to some of them, and
%	\item[(3)] the limit probability of random graph belonging to 
%	any of them is an analytic expression on the parameters $\overline{\beta}$. 
%\end{itemize}
%
%After this is archived the theorem follows easily. 
%
%The objective of next sections will be to define the classes 
%$C_1,\dots, C_{n_k}$ and to show that they satisfy properties (1), (2) and (3).
%
%Fix a quantifier rank $k$. It is known (\cite{finitemodeltheory1}, Section 2.4)
%that given an structure $G$, its 
%first order logic properties definable by sentences
%whose quantifier rank is bounded by $k$
%depend only on the $r_k$-neighborhoods in $G$ for some 
%constant $r_k$ that only
%depends on $k$.

\section{Model theoretic results}

\subsection{Some winning strategies for Duplicator}

During this section $H_1$ and $H_2$ stand for
hypergraphs and $V_1:=V(H_1)$, $V_2:=V(H_2)$.

\begin{definition} \label{def:similar}
Let $\overline{v} \in V_1^*, 
\overline{u} \in V_2^*$ be tuples of the same length.
We write $(H_1,\overline{v})\simeq_{k,r}(H_2 \overline{u})$, if Duplicator wins
$d\ehr_k\big(N(\overline{v};r),
\overline{v};\, N(\overline{u};r),\overline{u}\big)$.
Given $X\subseteq V_1$ and $Y\subseteq V_2$
we write $(H_1,X)\simeq_{k,r} (H_2,Y)$, if we can order $X$, resp $Y$ to form
lists $\overline{v}$, resp. $\overline{u}$ such that 
$(H_1,\overline{v})\simeq_{k,r}(H_2,\overline{u})$. 
Given $X\in V_1$, $Y\in V_2$ and 
tuples of the same length $\overline{v}\in V_1^*$ and
$\overline{u}\in V_2^*$ we write
 $\big(H_1, (X,\overline{v})  \big)
\simeq_{k,r} \big(H_2, (Y,\overline{u})  \big)$, if
$X$ and $Y$ can be ordered to form lists
$\overline{w}$, resp. $\overline{z}$ such that
$(H_1,\overline{w}^\smallfrown \overline{v})
\simeq_{k,r} (H_2,\overline{z}^\smallfrown \overline{u})$. 
\end{definition}

\begin{definition} \label{def:analogous}
Fix $r\in \N$. Suppose $X\subseteq V_1$ and 
$Y\subseteq V_2$ can
be partitioned into sets $X=X_1\cup \dots \cup X_a$
and $Y=Y_1\cup \dots \cup Y_b$ 
such that $N(X_i;r)$'s, and the
$N(Y_i;r)$'s, are connected and disjoint. 
We write $(H_1,X)\cong_{k,r} (H_2,Y)$, 
if for any set $Z\subset V_\delta$, with $\delta\in \{1,2\}$,
among the $X_i$'s or the $Y_i$'s
it is satisfied that ``the number of $X_i$'s  
such that $(H_\delta, Z) \simeq_{k,r} (H_1,X_i)$" 
and ``the number of $Y_i$'s such that
$(H_\delta,Z)\simeq_{k,r} (H_2,Y_i)$"
are both equal or are both greater than $k-1$.
\end{definition}

The main theorem of this section, which
is a slight strengthening of \cite[Theorem 
2.6.7]{spencer2013strange}, is the following:

\begin{theorem}\label{thm:DuplicatorAux}
	Set $r=(3^k-1)/2$.
	Suppose there exist
	sets $X\subseteq V_1$, $Y\subseteq V_2$ with the 
	following properties:
	\begin{enumerate}
		\item[(1)] $(H_1,X)\cong_{k,r} (H_2,Y)$.
		\item[(2)]
		\begin{itemize}
			\item Let $r^\prime\leq r$. Let $v\in V_1$ be
			a vertex such that $d(X,v)> 2r^\prime + 1$. Let 
			$\overline{u}\in (V_2)^{k-1}$ be a tuple of vertices. 
			Then there exists $u\in V_2$ such that 
			$d(u,\overline{u})>2r^\prime+1$,
			$d(Y,u)>2r^\prime +1$ and
			$(H_1,v)\simeq_{k,r^\prime} (H_2,u)$.	
			\item Let $r^\prime\leq r$. Let $u\in V_2$ be
			a vertex such that $d(Y,u)> 2r^\prime + 1$. Let 
			$\overline{v}\in (V_1)^{k-1}$ be a tuple of vertices. 
			Then there exists $v\in V_1$ such that 
			$d(v,\overline{v})>2r^\prime+1$,
			$d(X,v)>2r^\prime +1$ and
			$(H_1,v)\simeq_{k,r^\prime} (H_2,u)$
		\end{itemize}
	\end{enumerate}
	Then Duplicator wins $\ehr_k\big(H_1;H_2\big)$.
\end{theorem}

In order to prove this theorem we need to make two observations
and prove a previous lemma. 

\begin{obs} \label{obs1}
	Let
	$\overline{v}\in V(H_1)^*$, $\overline{u}\in V(H_2)^*$ be of equal length. Suppose
	Duplicator wins $d\ehr_k(H_1,\overline{v}; \, H_2,\overline{u})$.
	Then, for any $r\in \N$, $(H_1, \overline{v})\simeq_{k,r} 
	(H_2,\overline{u})$. 
\end{obs}



\begin{obs} \label{obs2}
	Let
	$\overline{v}\in V(H_1)^*$, $\overline{u}\in V(H_2)^*$ be of equal length. Suppose 
	Duplicator wins $d\ehr_k(H_1,\overline{v};\,H_2,\overline{u})$. 
	Let $v\in V(H_1),u\in V(H_2)$ be vertices
	played in the first round of an instance of the game 
	where Duplicator is following a winning strategy. Then 
	Duplicator also wins $d\ehr_{k-1}(H_1,\overline{v_2}; \,
	H_2,\overline{u_2})$, where $\overline{v_2}:=\overline{v}^\smallfrown v$
	and $\overline{u_2}:=\overline{u}^\smallfrown u$.
\end{obs}

\begin{lemma} \label{lemm:Duplicator}
	Let $\overline{v}\in V_1^*$ and
	$\overline{u} \in V_2^*$ be of equal length. Let
	$r\in \N$ be greater than zero. Suppose
	$(H_1,\overline{v})\simeq_{k,3r+1} (H_2,\overline{u})$.
	Let $v \in V_1$ and $u\in V_2$
	be vertices played in the first round of an instance of 
	\[
	d\ehr_k\big(\, N(\overline{v};3r+1),
	\overline{v}; \quad N(\overline{u};3r+1),\overline{u}\, \big)
	\]
	where Duplicator is following a winning strategy. Further suppose
	that $d(\overline{v},v)\leq 2r+1$ (and in consequence
	$d(\overline{u},u)\leq 2r+1$ as well). 
	Let $\overline{v_2}:=\overline{v}^\smallfrown v$
	and $\overline{u_2}:=\overline{u}\smallfrown u$.
	Then $(H_1,\overline{v_2})\simeq_{k-1,r}
	(H_2,\overline{u_2})$.
\end{lemma}

\begin{proof}
	Using \cref{obs2} we get that Duplicator wins 
	\[
	d\ehr_{k-1}\big(\, N(\overline{v};3r+1),
	\overline{v_2}; \quad N(\overline{u};3r+1)
	,\overline{u_2} \, \big)\]
	as well. Call $H^\prime_1=N(\overline{v};3r+1)$,
	$H^\prime_2=N(\overline{u};3r+1)$. Then by \cref{obs2}
	Duplicator wins
	\[
	d\ehr_{k-1}\big(\, N^{H^\prime_1}(
	\overline{v_2};r),\overline{v_2};\quad
	N^{H^\prime_2}(\overline{u_2};r),\overline{u_2}\, \big).
	\]
	Because of this if we prove $N^{H_1}(\overline{v_2};r)
	=N^{H^\prime_1}(\overline{v_2};r)$ and $N^{H_2}(\overline{u_2};r)
	=N^{H^\prime_2}(\overline{u_2};r)$, then we are finished. 
	Let $z\in N^{H_1}(v^\prime;r)$. Then
	$d(z,\overline{v})\leq d(z,v^\prime)+d(v^\prime,\overline{v})=3r+1$.
	In consequence, $N^{H_1}(v;r)\subset H^\prime_1$. Thus,
	$N^{H_1}(\overline{v_2};r)\subseteq H^\prime_1$, and $N^{H_1}(\overline{v_2};r)
	=N^{H^\prime_1}(\overline{v_2};r)$. Analogously we obtain 
	$N^{H_2}(\overline{u_2};r)=N^{H^\prime_2}(\overline{u_2};r)$, as we wanted. 
\end{proof}
\noindent\rule{2cm}{0.4pt}\par

Now we are in conditions to prove \cref{thm:DuplicatorAux}.





\begin{proof}[Proof of \cref{thm:DuplicatorAux}]
	%Set 
	%$\overline{v}[i]:=v_1,\dots,v_i$ and
	%$\overline{u}[i]:=u_1,\dots,u_i$
	%We say that Duplicator can play following the strategy $S$
	%if they can play in a way that
	%at the end of the $s$-th round, $\overline{w}[s]$ and
	%$\overline{z}[s]$ have $(k-s)$-similar $r_{k-s}$-neighborhoods.
	%In particular this means that at the end of the $k$-th
	%round Duplicator will have won $\ehr_k(G^1,H_2)$.	
	Let $X_1,\dots,X_a$ and 
	$Y_1,\dots,Y_b$ be partitions of 
	$X$ and $Y$ respectively
	as in the definition of $\cong_{k,r}$.
	Define $r_0=(3^k-1)/2$ and $r_i=(r_{i-1}-1)/3$ for each
	$1\leq i \leq k$. 
	Let $v^1_i$ and $v^2_i$ be the vertices played
	in $H_1$ and $H_2$ respectively during the $i$-th
	round of $\ehr_k(H_1,H_2)$. 
	We show a winning strategy for Duplicator
	in $\ehr_{k}\big(H_1;\, H_2\big)$. For each $0\leq i \leq k$,
	Duplicator will	keep track of some marked sets 
	of vertices $T\subset V_1$, $S\subset V_2$. 
	For $\delta=1,2$ each marked set
	$T\subset V_\delta$ will have associated a tuple
	of vertices $\overline{v}(T)\in V_\delta^*$ consisting
	of the vertices played in $H_\delta$ so far that were 
	"appropriately close" to $T$ when chosen,  ordered according
	to the rounds they where played in.
	The game will start with no sets of vertices marked and 
	at the end of the $i$-th
	round Duplicator will perform one of the two
	following operations:
	\begin{itemize}
	\item Mark two sets $S\subset V_1$ and $T\subset V_2$ and
	define $\overline{v}(S):=v^1_i$ and $\overline{v}(T):=v^2_i$.
	\item Given two sets $S\subset V_1$, $T\subset V_2$ that were 
	previously marked during the same round, append $v^1_i$
	and $v^2_i$ to $\overline{v}(S)$ and $\overline{v}(T)$ 
	respectively. 
	\end{itemize}
	We show that Duplicator can play in a way such that at the
	end round the following are satisfied:
	\begin{itemize}
		\item[(i)] For $\delta=1,2$, each vertex played
		so far $v^\delta_j\in V_\delta$ belongs to 
		$\overline{v}(S)$ for a	 unique marked set 
		$S\subset V_\delta$.
		\item[(ii)] Let $S\subset V_1$ and $T\subset V_2$ be sets
		marked during the same round. Then any previously played
		vertex $v^1_j$ occupies a position in $\overline{v}(S)$
		if and only if $v^2_j$ occupies the same position 
		in $\overline{v}(T)$.
		\item[(iii)] 
			\begin{itemize}
			\item  Let $S\subset V_1$ be a marked set. Then 
			for any different marked $S^\prime \subset V_1$ 
			of any different $S^\prime$ among $X_1,\dots,X_a$
			it holds $d(S,S^\prime)>2r_i + 1$. 
			\item Let $T\subset V_2$ be a marked set. Then
			for any different marked $T^\prime \subset V_2$
			or any different $T^\prime$ among
			$Y_1,\dots, Y_b$  it holds $d(T,T^\prime)>2r_i +1$.
			\end{itemize}
		\item[(iv)] Let $S\subset V_1, T\subset V_2$ be sets
		marked during the same round. Then
		\[
		\big(H_1, (S,\overline{v}(S))\big)\simeq_{k-i,r_i}
		\big(H_2, (T,\overline{v}(T))\big).	\]
					
		
	\end{itemize}
	In particular, if conditions (i) to (iv) are satisfied this means
	that if $\overline{v}^1:=(v^1_1,\dots,v^1_i)$ and
	$\overline{v}^2:=(v^2_1,\dots, v^2_i)$ are the vertices played so far
	then Duplicator wins
	\[
	d\ehr_{k-i}\big(\,
	N(\overline{v}^1;\, r_i), \overline{v}^1; \quad
	N(\overline{v}^2;\, r_i),\overline{v}^2 \,
	\big),	
	\]
	And at the end of the $k$-th round Duplicator will have won
	$\ehr(H_1;\, H_2)$. \par
	The game $d\ehr_k(H_1; \, H_2)$ proceeds as follows.
	Clearly properties (i) to (iv) hold at the beginning of 
	the game.  Suppose
	Duplicator can play in such a way that properties (i) to (iv) hold 
	until the beginning of the $i$-th round. Suppose
	during the $i$-th round Spoiler chooses $v^1_i\in V_1$ 
	(the case where they play in $V_2$ is symmetric). There are 
	three possible cases:
	\begin{itemize}[leftmargin=*]
		\item For some unique previously marked set $S\subset V_1$ 
		it holds that $d(S\cup \overline{v},\quad v^1_i)\leq 2r_i +1$. 
		In this case let $T\subset V_2$ be the set in $H_2$ marked in the 
		same round as $T$. By hypothesis 
		\[ \big(H_1,(S,\overline{v}(S))\big)
		\simeq_{k-i+1,3r_i+1}
		\big(H_2,(T,\overline{v}(T))\big). 		
		\]
		Then, by definition,
		for some orderings $\overline{w}$, $\overline{z}$
		of the vertices in $S$ and $T$ respectively it holds that
		Duplicator wins
		\[
		d\ehr_{k-i+1}\big(\,
		N(\overline{w}^\smallfrown \overline{v}(S); \, 3r_i	+ 1)
		, \overline{w}^\smallfrown \overline{v}(S); \quad
		N(\overline{z}^\smallfrown \overline{v}(T); \, 3r_i	+ 1)
		, \overline{z}^\smallfrown \overline{v}(T)\,	
		\big).		
		\]
		Thus Duplicator can choose $v^2_i\in V_2$ according to the 
		winning strategy in that game. After this Duplicator sets 
		$\overline{v}(S):= \overline{v}(S)^\smallfrown v^1_i$, and
		$\overline{v}(T):= \overline{v}(T)^\smallfrown v^2_i$. Notice
		that because of \cref{lemm:Duplicator} now
		\[
		\big(H_1, (S,\overline{v}(S))\big)\simeq_{k-i,r_i}
		\big(H_2, (T,\overline{v}(T))\big).
		\]
		\item For all marked sets $S\subset V_1$ it holds
		$d(S\cup \overline{v}(S), \quad v^1_i)>2r_i +1$, but there is
		a unique $S$ among $X_1,\dots, X_a$ such that
		$d(S,v^1_i)\leq 2r_i+1$. In this case from condition (1)
		of the statement follows that there is some non-marked
		set $T$ among $Y_1,\dots, Y_b$ such that
		\[
		(H_1,S)\simeq_{k-i+1,3r_i+1} (H_2,T).\] 
		Thus, by definition, for some orderings $\overline{w}$, 
		$\overline{z}$ of the vertices in $S$ and $T$ respectively
		it holds that Duplicator wins
		\[
		d\ehr_{k-i+1}\big( \,
		N(\overline{w};3r_i+1), \overline{w};
		\quad
		N(\overline{z};3r_i+1), \overline{z}
		\, \big).		
		\]
		Then Duplicator can choose $v^2_i\in V_2$ according 
		to a winning strategy for this game. After this Duplicator
		marks both $S$ and $T$ and sets 
		$\overline{v}(S):=v^1_i$, and 
		$\overline{v}(T):=v^2_i$. Notice
		that because of \cref{lemm:Duplicator} now
		\[
		\big(H_1, (S,\overline{v}(S))\big)\simeq_{k-i,r_i}
		\big(H_2, (T,\overline{v}(T))\big).
		\]
		\item For all marked sets $S\subset V_1$ it holds
		$d(S\cup \overline{v}(S), \quad v^1_i)>2r_i +1$, and
		for all sets $S$ among $X_1,\dots, X_a$ it also holds
		$d(S,v^1_i)> 2r_i+1$. In this case
		from condition (2) of the statement follows that
		Duplicator can choose
		$v^2_i\in V_2$ such that
		(A)
		$d(T\cup \overline{v}(T),\quad v^2_i)>2r_i+1$ for all
		marked sets $T\subset V_2$, (B)
		$d(T, v^2_i)> 2r_i+1$ for all sets $T$ among
		$Y_1,\dots, Y_b$, and (C) 
		$(H_1,v^1_i)\simeq_{k-i,r_i} (H_2, v^2_i)$.
		After this Duplicator marks both $S=\{
		v^1_i\}$ and $T=\{v^2_i\}$ and sets
		$\overline{v}(S):=v^1_i$, and
		$\overline{v}(T):=v^2_i$. 		
	\end{itemize}	 
	The fact that conditions
	(i) to (iv) still hold at the end of the round
	follows from comparing $r_{i-1}$ and $r_{i}$ as well 
	as applying \cref{obs1} and \cref{obs2}.

	% During the game
%	Duplicator will track of sets of lists $L^1[i]$'s
%	and $L^2[i]$'s. At the end of the $i$-th round
%	Duplicator will obtain $L^1[i]$, resp $L^2[i]$
%	performing one of the following operations on 
%	$L^1[i-1]$, resp $L^2[i-1]$:
%	\begin{itemize}
%		\item Adding a list $\overline{v}\in (V_1)^*$
%		to $L^1[i-1]$ and marking it with the index $i$. 
%		\item Appending a new vertex $v\in V_1$ to an 
%		existing list $\overline{v}\in L^1[i-1]$.
%	\end{itemize}	
%	Let $r_k=1$ and for each $i\in [k-1]$ 
%	let $r_i=3r_{i+1}+1$.
%	We are going to show that Duplicator can 
%	play in a way such that
%	at the end of the $i$-th 
%	round the following conditions
%	are satisfied:
%	\begin{itemize}
%		\item[(1)] 
%		
%		
%		
%	\end{itemize}
%	
%	
%	\[\mathcal{X}[0]=\emptyset,
%	\quad \mathcal{Y}[0]=\emptyset.\]
%	At the end of the $s$-th round $\mathcal{X}[s-1]$,
%	resp. $\mathcal{Y}[s-1]$, will be updated into 
%	$\mathcal{X}[s]$,
%	resp. $\mathcal{Y}[s]$, by performing on it
%	some of the following operations: adding a new list
%	to it, appending one vertex to an existing list, and 
%	marking a list with the index $s$. Duplicator will
%	keep track of the sets $\mathcal{X}[s]$ and $\mathcal{Y}[s]$.\par
%	
%	We show first an strategy for Duplicator and will prove its
%	correctness afterwards. The strategy is as follows: At the 
%	beginning of the $s$-th round suppose Spoiler plays $w_s$ in
%	$G^1$. The case where they play $z_s$ in $G^2$ is symmetric.
%	Call $r=r_{k-s}$. There are three possibilities. 
%	\begin{itemize}
%		\item[Case 1:] The vertex $w_s$ satisfies
%		$d(w_s,\overline{v})>2r+1$ for all 
%		$\overline{v}\in \mathcal{X}[s-1]$. Then Duplicator 
%		can find a vertex $z_s$ in $G^2$ such that
%		$d(z_s,\overline{u})>2r+1$ for all 
%		$\overline{u}\in \mathcal{Y}[s-1]$ satisfying 
%		that $w_s$ and $z_s$ have $(k-s)$-similar
%		$r$-neighborhoods. To form $ \mathcal{X}[s]$
%		and $ \mathcal{Y}[s]$, add to $\mathcal{X}[s-1]$
%		and $ \mathcal{Y}[s-1]$ the lists consisting
%		of only $w_s$ and only $z_s$ respectively, and mark
%		them with the number $s$.
%		
%		\item[Case 2:] The vertex $w_s$ satisfies
%		$d(w_s,\overline{v})\leq 2r+1$ for a unique 
%		$\overline{v}\in \mathcal{X}[s-1]$,and
%		$\overline{v}$ is marked. In this case,
%		find the list $\overline{u}\in \mathcal{Y}[s-1]$
%		with the same mark. Duplicator 
%		then can chose $z_s\in N(\overline{u},2r+1)$
%		in response to $w_s$ according to a winning strategy
%		for
%		\[
%		d\ehr_{k-s}(N(\overline{v},3r+1),\overline{v},
%		N(\overline{u},3r+1),\overline{u}).
%		\]
%		To form $ \mathcal{X}[s]$
%		and $ \mathcal{Y}[s]$, append $w_s$ and $z_s$
%		to $\overline{v}$ and $\overline{u}$ respectively.
%		
%		\item[Case 3:] The vertex $w_s$ satisfies
%		$d(w_s,\overline{v})\leq 2r+1$ for a unique 
%		$\overline{v}\in \mathcal{X}[s-1]$,and
%		$\overline{v}$ is not marked. In this case we 
%		can	find a non-marked 
%		list $\overline{u}\in \mathcal{Y}[s-1]$
%		such that $\overline{v}$ and $\overline{u}$ 
%		have $(k-s)$-similar $(3r+1)$-neighborhoods. 
%		Duplicator then can chose $z_s\in N(\overline{u},2r+1)$
%		in response to $w_s$ according to a winning strategy
%		for
%		\[
%		d\ehr_{k-s}(N(\overline{v},3r+1),\overline{v},
%		N(\overline{u},3r+1),\overline{u}).
%		\]
%		To form $ \mathcal{X}[s]$
%		and $ \mathcal{Y}[s]$, append $w_s$ and $z_s$
%		to $\overline{v}$ and $\overline{u}$ respectively,
%		and mark those lists with the number $s$. 
%	\end{itemize}
%	
%	All that is left now is to prove the correctness of the
%	strategy. We show that at the end the $s$-th round, if 
%	two lists $\overline{v} \in \mathcal{X}[s]$ and 
%	$\overline{u} \in \mathcal{Z}[s]$ have the same mark then
%	$\overline{v}$ and $\overline{u}$ have $(k-s)$-similar
%	$r_{k-s}$-neighborhoods. This happens trivially at
%	the end of the zeroth round -i.e., the beginning 
%	of the game- as there are no marked lists. Assume the
%	statement holds up to the end of the $(s-1)$-th round, 
%	where $s>0$. 
%	
%	\begin{itemize}
%		\item[Case 1:] Notice that the lists
%		in $\mathcal{Y}[s-1]$ only contain the
%		vertices previously played in $G^2$ and 
%		the ones from $Y$. Thus, assumption $(3)$ 
%		of the theorem, (or assumption $(2)$ in the 
%		symmetric
%		case where Spoiler plays in $G^2$) assures us
%		that Duplicator can always find such $z_s$ 
%		sufficiently far away from all the other lists. 
%		In this case, the only new marked lists in $\mathcal{X}[s]$
%		and $\mathcal{Y}[s]$ are the ones
%		consisting of $w_s$ and $z_s$ respectively. By assumption
%		$w_s$ and $z_s$ have $(k-s)$-similar $r_{k-s}$-neighborhoods.
%		
%		\item[Case 2:] Notice that by the induction hypothesis 
%		$\overline{v}$ and $\overline{u}$ have $(k-s+1)$-similar
%		$r_{s-k+1}$-neighborhoods, and in consequence a winning strategy
%		for Duplicator exists. Using \cref{lemm:Duplicator} we obtain 
%		that the extended lists $\overline{v},w_s$ and $\overline{u},z_s$
%		have $(k-s)$-similar $r_{s-k}$-neighborhoods. 	
%		
%		\item[Case 3:] This case is analogous to the previous one.
%		The definition of $k$-agreeability implies that there is 
%		such an unmarked list $\overline{u}$ available. Using \cref{lemm:Duplicator} 
%		we obtain 
%		that the extended lists $\overline{v},w_s$ and $\overline{u},z_s$
%		have $(k-s)$-similar $r_{s-k}$-neighborhoods.	
%	\end{itemize} 
%	
%	In the three cases, if $\overline{v}$ and $\overline{v}$
%	are lists in $\mathcal{X}[s-1]$ and $\mathcal{Y}[s-1]$
%	respectively that share the same mark and remain unmodified
%	in $\mathcal{X}[s]$ and $\mathcal{Y}[s]$, then by the 
%	induction hypothesis $\overline{v}$ and $\overline{v}$
%	have $(k-s+1)$-similar $r_{k-s+1}$-neighborhoods. This
%	easily implies that they also have
%	$(k-s)$-similar $r_{k-s}$-neighborhoods.\par
%	At the end of the game, if $\overline{v}\in \mathcal{X}[k]$ and 
%	$\overline{u}\in \mathcal{Y}[k]$ are lists with the same mark
%	then the natural mapping between $\overline{v}$ and $\overline{u}$
%	defines an isomorphism between $G^1[\overline{v}]$ and $G^2[\overline{u}]$.
\end{proof}



\subsection{k-Equivalence relation}


\subsubsection{k-Equivalent trees} \label{sect:equivtrees}


A \textbf{rooted tree} $(T,v)$ is a tree $T$ with a 
distinguished vertex $v\in V(T)$ called its \textbf{root}.
We will usually omit the root when it is not relevant and 
write just $T$ instead of $(T,v)$. The
\textbf{initial edges} of a rooted tree $(T,v)$ are 
the edges in $T$ that contain $v$. 
We define the radius of a rooted tree
as the maximum distance between its root
and any other vertex. 
\par

%
%Similarly, we define an edge-rooted tree $(T,e,v)$
%as a tree $T$ with a distinguished edge $e\in E(T)$ and
%a distinguished vertex $v\in e$. \par
Given a rooted tree $(T, v)$, and a vertex $u\in V(T)$, 
we define $\mathrm{Tr}(T,v;\,\, u)$ as the tree $T[X]$ induced on the
set $X:=\{ \, w\in V(T) \, | \, d(v,w)=d(v,u)+ d(u,w) \,  \}
$, to which we assign $u$ as the root.
That is, $\mathrm{Tr}(T,v;\,\, u)$ is the tree consisting of those vertices
whose only path to $v$ contains $u$.
\par


\begin{definition}

Fix a natural number $k$. We define the
\textbf{$k$-equivalence} relation over rooted trees, 
written as $\sim_k$, by induction over their radii as follows:

\begin{itemize}[leftmargin=*]
	\item Any two trees with radius zero are $k$-equivalent.
	Notice that those trees
	consist only of one vertex: their respective roots.
	
	\item Let $r>0$.
	Suppose the $k$-equivalence relation has been
	defined for rooted trees with radius at most $r-1$. Let $\Sigma_{k,r-1}$
	be the set consisting of the $\sim_k$ classes of trees
	with radius at most $r-1$. Let $\rho$ be an special symbol called the
	\textbf{root symbol}. Set $\widehat{\Sigma}_{k,r-1}:=\Sigma_{k,r-1}\cup \{\rho\}$.
	Then we call a $(k,r)$-\textbf{pattern} to an isomorphism class
	of $\widehat{\Sigma}_{k,r-1}$-hypergraphs 
	$(e,\tau)$ that consist of only one edge and satisfy 
	$\tau(v)=\rho$ for exactly one vertex $v\in V(e)$. We will
	denote by $P(k,r)$ the set of $(k,r)$-patterns. \par
	Given a rooted tree $(T,v)$ of radius $r$
	we define its \textbf{canonical coloring}
	as the map 
	$\tau_{(T,v)}: V(T)\rightarrow \widehat{\Sigma}_{k,r-1}$ satisfying that
	 $\tau_{(T,v)}(u)$ is the
	$\sim_k$ class of $\mathrm{Tr}(T,u;\, \,v)$
	for any $u\neq v$, and $\tau_{(T,v)}(v)=\tau$. \par
	Let $T_1$ and $T_2$ be rooted trees of radius $r$. 
	We say that $(T_1,v_1)\sim_k (T_2,v_2)$ 
	if for any pattern $\epsilon\in P(k,r)$	the
	``quantity of initial edges $e_1\in E(T_1)$ such that
	$\big(e,\tau_{(T_\delta,v_\delta)}\big) \in \epsilon$" 
	and the
	``quantity of initial edges $e_2\in E(T_2)$ such that
	$\big(e,\tau_{(T_\delta,v_\delta)}\big)\in \epsilon$
	" are equal or
	are both greater than $k-1$.
\end{itemize}

\end{definition}

The following is a way of characterizing $\sim_k$ classes
of rooted trees with radii at most $r$ that will be useful later. 

\begin{obs}\label{obs:equivalenttrees}
Let $\mathcal{T}$ be a $\sim_k$ class of rooted trees with
radii at most $r$. Then there is a partition $E^1_\mathcal{T},
E^2_\mathcal{T}$ of $P(k,r)$ and natural numbers $a_\epsilon<k$
for each $\epsilon\in E^2_\mathcal{T}$ that only depend on 
$\mathcal{T}$ such that any rooted tree $(T,v)$ belongs to
$\mathcal{T}$ if and only if the following hold: (1) For any pattern 
$\epsilon\in E^1_\mathcal{T}$ there are at least $k$ initial edges $e\in E(T)$ such that
$(e,\tau_{(T,v)})\in \epsilon$, and (2) for any pattern 
$\epsilon\in E^2_\mathcal{T}$ there are exactly
$a_\epsilon$ initial edges $e\in E(T)$ such that
$(e,\tau_{(T,v)})\in \epsilon$.	
\end{obs}

\begin{obs}\label{obs:finitetrees}
	Using last characterization of $\sim_k$ classes
	it is easy to show that for any $r\in \N$ the quantity
	of $\sim_k$ classes of trees with radii at most $r$ is finite. 
	We proceed by induction. For $r=0$ there is only one $\sim_k$ class.
	Now let $r>0$ and suppose the statement holds for $r-1$. Then
	the number of $(k,r)$-patterns is finite and so is the number
	of $\sim_k$ classes of trees with radii at most $r$. 
\end{obs}


We want prove the following
\begin{theorem} \label{thm:equivalenttrees} 
	Let $(T_1,v_1)$ and $(T_2,v_2)$ be rooted trees such
	that $(T_1,v_1)\sim_k (T_2,v_2)$.
	Then Duplicator wins
	$d\ehr_{k}(T_1,v_1;\, \, T_2,v_2)$.
\end{theorem}

Before proceeding with the proof we need the following auxiliary
result. Let $(T,v)$ be a rooted tree and $e$ an 
initial edge of $T$. We define $\mathrm{Tr}(T,v;\, e)$ as
the induced tree $T[X]$ on the set
$X:=\{v\} \cup \{\, u\in V(T) \, | \, d(v,u) = 1 + d(e,u) \,\}$,
to which we assign $v$ as the root. In other words, 
$\mathrm{Tr}\big(T,v;\, e\big)$ is the tree formed of 
$v$ and all the vertices
in $T$ whose only path to $v$ contain $e$. 

\begin{lemma} \label{lem:equivalentedges}
	Fix $r>0$. Suppose theorem \ref{thm:equivalenttrees}
	holds for rooted trees with radii at most $r$.
	Let $(T_1,v_1)$ and $(T_2,v_2)$ be rooted trees with radius
	$r+1$. Let $\tau_{(T_1,v_1)}$ and $\tau_{(T_2,v_2)}$
	be colorings over $T_1$ and $T_2$ as in the
	definition of $k$-equivalence. 
	Let $e_1$ and $e_2$ be initial edges 
	of $T_1$ and $T_2$ respectively satisfying 
	$(e_1,\tau_{(T_1,v_1)})\simeq (e_2,\tau_{(T_2,v_2)})$. Name 
	$T^\prime_1:=\mathrm{Tr}(T_1,v_1; \,\,e_1)$ and 
	$T^\prime_2:=\mathrm{Tr}(T_2,v_2;\,\,e_2)$. Then
	Duplicator wins 
	$d\ehr_{k}(T^\prime_1, v_1; \, \, T^\prime_2,v_2)$.
\end{lemma}
\begin{proof}
	We show a winning strategy for Duplicator.
	At the beginning of the game fix 
	$f:V(e_1)\rightarrow V(e_2)$ an isomorphism between 
	$(e_1,\tau_{(T_1,v_1)})$
	and $(e_2,\tau_{(T_2,v_2)})$.
	Suppose in the $i$-th round of the game Spoiler
	plays on $T^\prime_1$. The other case is symmetric. 
	There are two possibilities:
	\begin{itemize}[leftmargin=*]
		\item If Spoiler plays $v_1$ 
		then Duplicator chooses $v_2$. 
		\item Otherwise, Spoiler plays a vertex $v$ that belongs
		to some $\mathrm{Tr}(T^\prime_1,v_1;\,\, u)$ for a unique $u\in V(e_1)$
		different from the root $v_1$. 
		Set $T^{\prime\prime}_1:=
		\mathrm{Tr}\big(T^\prime_1,v_1;\,\, u\big)$
		and
		$T^{\prime\prime}_2:=\mathrm{Tr}\big(T^\prime_2,v_2;\,\, f(u)\big)$
		Then, as $\tau_{(T_1,v_1)}\big(u\big)=\tau_{(T_2,v_2)}\big(f(u)\big)$,
		we obtain
		$\big(T^{\prime\prime}_1,u\big) \sim_k 
		\big(T^{\prime \prime}_2,f(u)\big)$.
		As both these trees have radii at most $r$, 
		by assumption Duplicator has a winning 
		strategy in
		$d\ehr_{k}\big(\,T^{\prime\prime}_1, u;
		\quad T^{\prime \prime}_2, f(u)\,  \big)$		
		and they can follow it considering the previous plays in
		$T^{\prime\prime}_1$ and $T^{\prime\prime}_2$.	
	\end{itemize}
\end{proof}
\noindent\rule{2cm}{0.4pt}

Now we can prove the main theorem of this section:

\begin{proof}[Proof of \cref{thm:equivalenttrees}]~ \par
	Notice that, as $(T_1,v_1)\morph{k} (T_2,v_2)$, both $T_1$ and
	$T_2$ have the same radius $r$.
	We prove the result by induction on $r$.
	If $r=0$ then both $T_1$ and $T_2$ consist
	of only one vertex and we are done. \par
	Now let $r>0$ and assume that the 
	statement is true for all lesser values of $r$.
	Let $\tau_{(T_1,v_1)}$ and $\tau_{(T_2,v_2)}$ 
	be the colorings over $T_1$ and $T_2$ as in 
	the definition of $\sim_k$. 
	We show that there is a winning strategy 
	for Duplicator in
	$d\ehr_k(T_1,v_1;\,\, T_2,v_2)$.
	At the start of the game, set all the initial edges
	in $T_1$ and $T_2$ as non-marked. 
	Suppose in the $i$-th round Spoiler plays in 
	$T_1$. The other case is symmetric. 
	\begin{itemize}[leftmargin=*]
		\item If Spoiler plays $v_1$ then Duplicator plays $v_2$.
		\item Otherwise, the vertex played by Spoiler belongs to
		$\mathrm{Tr}(T_1,v_1;\,\,e_1)$
		for a unique initial edge $e_1$ of $T_1$. 
		There are two possibilities:
		\begin{itemize}[leftmargin=*]
			\item If $e_1$ is not marked yet, mark it. 
			In this case, there is a 
			non-marked initial
			edge $e_2$ in $T_2$ satisfying 
			$\big(e_1,\tau_{(T_1,v_1)}\big)\simeq
			\big(e_2,\tau_{(T_2,v_2)}  \big)$.
			Mark $e_2$ as well. 
			Set $T^\prime_1:=\mathrm{Tr}(T_1,v_1;\,\,e_1)$
			and
			$T^\prime_2:=\mathrm{Tr}(T_2,v_2;\,\,e_2)$
			Because of
			\cref{lem:equivalentedges}, Duplicator
			has a winning strategy in
			$ d\ehr{k}(T^\prime_1, v_1;$ $\,\, T^\prime_2,v_2)$
			and can play according to it.
			\item If $e_1$ is already marked then there is
			a unique initial edge $e_2$ in $T_2$ that was 
			marked during the same round as $e_1$ and it satisfies 
			$\big(e_1,\tau_{(T_1,v_1)}\big)\simeq
			\big(e_2,\tau_{(T_2,v_2)}  \big)$.	
			Again, because of \cref{lem:equivalentedges}, 
			Duplicator has a winning strategy in
			$d\ehr{k}(T^\prime_1, v_1; \,\, T^\prime_2,v_2)$
			and can continue playing according to it taking
			into account the plays made previously in 
			$T^\prime_1$ and $T^\prime_2$.	
		\end{itemize}
	\end{itemize}
\end{proof}

\subsubsection{k-Equivalent hypergraphs} \label{sect:equivunicycles}

The \textbf{center} of an hypergraph $H$, written as
$Center(H)$,  
is the union of all saturated sub- hypergraphs of $H$. 
%	Given $r\in \N$, the $r$-\textbf{center} of $H$,
%	written as $Center(H;r)$, is the union of all saturated 
%	sub-hypergraphs of $H$ with diameter at most $2r+1$.  
Let $H$ be an hypergraph, let $\overline{v}\in V(H)^*$. 
%	and let $r\in \N$
If $H$
is connected then we define the graph
$Center(H,\overline{v})$
%	, resp. $Center(H,\overline{v};\, r)$,
as the minimal connected sub-hypergaph of $H$ that contains
both $Center(H)$
%	, resp. $Center(H;\,r)$,
and the vertices in $\overline{v}$. Otherwise,
if $H$ is a general hypergraph we define $Center(H,\overline{v})$,
%	, resp. $Center(H,\overline{v};\, r)$, 
as the union, for all connected components $H^\prime \subset H$,
of the minimal connected sub-hypergraph of $H^\prime$ that
contains both $Center(H^\prime)$
%	, resp. $Center(H^\prime,\overline{v};\, r)$,
and the vertices in $\overline{v}$
that belong to $V(H^\prime)$.



\begin{definition}\label{def:TrOperator}
	Let $H$ be an hypergraph, let $\overline{v}\in V(H)^*$ and 
	let $v\in H$. We define $\mathrm{Tr}\big(
	H,\overline{v};\,\, v\big)$ in the following way:
	\begin{itemize}[leftmargin=*]
		\item If $d\big(\, Center(H, \overline{v})
		, \, \, v \big)=\infty$ then $v$ belongs to a connected
		component $T$ of $H$ which is a tree and does not contain
		any vertex in $\overline{v}$. In this case
		$\mathrm{Tr}\big(
		H,\overline{v};\,\, v\big)$ is the tree $T$ rooted at $v$.
		\item Otherwise $\mathrm{Tr}\big(
		H,\overline{v};\,\, v\big)$ is the tree $H[X]$ 
		induced on the set 
		\[
		X:=\big\{\, u\in V(H) \,\, \big| \, 
		d\big(\, Center(H, \overline{v})
		, \, \, u \big)= d\big(\, Center(H, \overline{v})
		, \, \, v \big) + d(v,u)
		\big\},
		\]
		to which we assign $v$ as a root. That is,
		$\mathrm{Tr}\big(
		H,\overline{v};\,\, v\big)$ is the tree formed of all 
		vertices whose only path to $Center(H,\overline{v})$ 
		contains $v$. 		
	\end{itemize}
%	Given $r\in\N$ we define  $\mathrm{Tr}\big(
%	H,\overline{v};\,\, v;\,\, r\big)$ analogously to $\mathrm{Tr}\big(
%	H,\overline{v};\,\, v\big)$ in the following way:
%	\begin{itemize}
%		\item If $d\big(\, Center(H, \overline{v};\, r)
%		, \, \, v \big)>r$ then $T:=N(v;r)$ is a tree
%		and	$\mathrm{Tr}\big(
%		H,\overline{v};\,\, v;\,\, r\big)$ is the tree $T$ rooted at $v$.
%		\item Otherwise $\mathrm{Tr}\big(
%		H,\overline{v};\,\, v\big)$ is the tree $H[X]$ 
%		induced on the set 
%		\begin{align*}
%		X:=\Big\{\, u\in V(H) \,\, \Big| \, \,
%		d\big(\, Center(H, \overline{v};\, r)
%		, \, \, u \big)= d\big(\, Center(H, \overline{v};\, r)
%		, \, \, v \big) + d(v,u),&\\  \text{ and } d(v,u)\leq r
%		\Big\},&
%		\end{align*}
%		to which we assign $v$ as a root. That is,
%		$\mathrm{Tr}\big(
%		H,\overline{v};\,\, v;\,\, r\big)$ is the tree formed of all 
%		vertices $u$ whose distance to $v$ is at most $r$ and 
%		whose only path to $Center(H,\overline{v};\, r)$ contains $v$. 
%	\end{itemize}
	In the case that $\overline{v}$ is the empty list we will write
	simply $\mathrm{Tr}(H;\,\, v)$ 
%	or $\mathrm{Tr}(H;\,\, v;\,\,r)$ 
	instead of 
	$\mathrm{Tr}(H,\, \, ;\,\, v)$.
%	 or $\mathrm{Tr}(H,\,\,;\,\, v;\,\,r)$.
	Notice that in the case that
	$(T,u)$ is a rooted tree then the definition of 
	$\mathrm{Tr}(T,u;\, \, v)$ given in \cref{sect:equivtrees}
	coincides with the one we have
	given now,
	so no confusion should arise. \par
\end{definition}

Let $H$ be a non-tree connected hypergraph. We define
the \textbf{canonical coloring} $\tau_{H}$ over $H$
as the one that assigns to each vertex $v\in V(H)$ the $\sim_k$ class of the tree
$\Tr(H,v)$.\par

\begin{definition}
	Let $H_1$ and $H_2$ be connected hypergraphs which are not trees.
	Set $H^\prime_1:= Center(H_1)$ and $H^\prime_2:= Center(H_2)$.
	We say that $H_1$ and $H_2$ are $k$-equivalent,
	written as $H_1\sim_k H_2$, if
	$\big( H^\prime_1,\tau_{H_1}\big) \simeq 
	\big(H^\prime_2,\tau_{H_2}\big)$
\end{definition}

The main theorem of this section is the following
	
	\begin{theorem} \label{thm:strategyaux}
		Let $H_1$ and $H_2$ be non-tree 
		connected hypergraphs satisfying
		$H_1\sim_k H_2$. 
		Set $H^\prime_1:= Center(H_1)$ and 
		$H^\prime_2:= Center(H_2)$. Let $f$ be an isomorphism
		between $\big( H^\prime_1,\tau_{H_1}\big)$ and 
		$\big(H^\prime_2,\tau_{H_2}\big)$. Let $\overline{v}$ 
		be an ordering of the vertices of $H^\prime_1$ and let
		$\overline{u}:=f(\overline{v})$ be the corresponding
		ordering of the vertices of $H^\prime_2$. Then
		Duplicator wins 
		$	d\ehr_{k}\big(\,
		H^\prime_1,\overline{v};\,\,
		H^\prime_2,\overline{u}\,
		\big).
		$
	\end{theorem}
	\begin{proof}
		The winning strategy for Duplicator is as follows. 
		Suppose at the beginning of the $i$-th round
		Spoiler plays in $H_1$ (the case where they play in
		$H_2$ is symmetric). Then Spoiler has chosen a vertex 
		that belongs to $\mathrm{Tr}(H_1;\,\,u)$ for a unique
		$u\in H^\prime_1$. 
		Set $T_1:=\mathrm{Tr}\big(H_1;\,\,u\big)$ and
		$T_2:=\mathrm{Tr}\big(H_2;\,\,f(u)\big)$.
		By hypothesis
		$(T_1,u)\sim_k (T_2,f(u))$. 
		Then because of \cref{thm:equivalenttrees} we have that
		Duplicator has a winning strategy in
		$
		d\ehr_{k}\big(\,
		T_1,u; \, \, T_2, f(u)\,\big),
		$
		and they can follow it taking into account the previous
		plays made in $T_1$ and $T_2$, if any. In particular, 
		if Spoiler has chosen
		$u$ then Duplicator will necessarily choose $f(u)$.
		One can easily check that distances are preserved
		following this strategy. 
	\end{proof}
		
\subsection{$r$-Cores} \label{sec:Core}

Let $H$ be an hypergraph. Let $X$ be the set of vertices $v\in V(H)$
that belong to any saturated sub-hypergraph of $H$
whose diameter is at most $2r+1$. 
We define the \textbf{$r$-core} of $H$, 
written as $Core(H;r)$ as $N(X;r)$. Given 
$\overline{v}\in V(G)^*$, we define $Core(H,\overline{v};r)$
as $N(Y;r)$, where $Y$ is the set of vertices $v\in V(H)$ that either
belong to $\overline{v}$ or belong to any saturated sub-hypergraph of $H$
whose diameter is at most $2r+1$.
We say that $H$ is $r$-\textbf{simple} if all connected components
of $Core(H;r)$ are unicycles.
	

\begin{definition} \label{def:agreeability}
	Let $H_1$ and $H_2$ be hypergraphs and let $r\in \N$.
	Let $H^\prime_1:=Core(H_1;r)$ and $H^\prime_2:=Core(H_2;r)$. 
	We say that $H_1$ and $H_2$ are $(k,r)$-agreeable, written
	as $H_1\approx_{k,r} H_2$ if for any $\sim_k$ class $\mathcal{H}$ 
	``the number of connected
	components in $H^\prime_1$ that belong to $\mathcal{H}$" and
	``the number of connected components in $H^\prime_2$ that belong to 
	$\mathcal{H}$" are the same or are both greater than $k-1$.\par
%	 satisfies that
%	``the number of connected components $H^{\prime\prime}_1$
%	from $H^\prime_1$ such that $ H^{\prime\prime}_1 \sim_k F\, \, \,$'' 
%	and ``the number of
%	connected components $H^{\prime \prime}_2$ of $H^\prime_2$
%	such that
%	$ H^{\prime\prime}_2\sim_k F\, \, \,$" 
%	are both equal or are both greater 
%	or equal than $k$. Then we say that $H_1$ and $H_2$
%	are $(k,r)$-\textbf{agreeable}, written as $H_1\approx_{k,r} H_2$. \par

\end{definition}



\begin{definition}
	Let $\Sigma_{(k,r-1)}$ be the set of $\sim_k$ classes
	of rooted trees with radii at most $r-1$. Then
	we call a $(k,r)$-\textbf{cycle} to an isomorphism class
	of $\Sigma_{(k,r-1)}$-hypergraphs 
	$(H,\tau)$ that are cycles of diameter at most $2r+1$.
	We will denote by $C(k,r)$ the set of $(k,r)$-cycles.
\end{definition}

The following is a way of characterizing $\approx_{k,r}$ classes
of $r$-simple hypergraphs.


\begin{obs}\label{obs:agreeablecores}
	Let $\mathcal{O}$ be a $\approx_{k,r}$ class
	of $r$-simple hypergraphs. 
	Then there is a partition $U^1_\mathcal{O},
	U^2_\mathcal{O}$ of $C(k,r)$ and natural numbers $a_\omega<k$
	for each $\omega\in U^2_\mathcal{O}$ that only depend on 
	$\mathcal{O}$ such that any $r$-simple hypergraph $G$ belongs to
	$\mathcal{O}$ if and only if it holds that (1) for any $\omega\in U^1_\mathcal{O}$ 
	there are at least $k$ connected components $H \subset Core(G;r)$ whose cycle 		
	$H^\prime=Center(H)$ satisfies that	$(H^{\prime},\tau_{H})\in \omega$, and
	(2) for any $\omega\in U^2_\mathcal{O}$ there are exactly $a_\omega$ 
	connected components $H \subset Core(G;r)$ whose cycle 
	$H^\prime=Center(H)$ satisfies that	$(H^{\prime},\tau_{H})\in \omega$.	
\end{obs}


\begin{lemma} \label{lem:aux1}
	Let $r\in \N$ and let $H_1, H_2$ be hypergraphs such that 
	$H_1\approx_{k,r} H_2$. Let $X$ and $Y$ be the
	sets of vertices in $H_1$, resp. $H_2$, that
	belong to any saturated sub-hypergraph of diameter 
	at most $2r+1$. Then $(H_1,X)\cong_{k,r} (H_2,Y)$ in the sense of
	\cref{def:analogous}.
\end{lemma}
\begin{proof}
	Let $X_1,\dots, X_a$ and $Y_1,\dots, Y_b$ be partitions of
	$X$ and $Y$ such that each $N(X_i;r)$ and $N(Y_i;r)$ is 
	a connected component of $Core(H_1;r)$, resp. $Core(H_2;r)$.
	Using the definition of $H_1\approx_{k,r}H_2$ as well 
	as the fact that  because of \cref{thm:strategyaux}
	$N(X_i;r)\sim_k N(Y_j;r)$ implies 
	$(H_1,X_i)\simeq_{k,r} (H_2,Y_j)$
	the result follows.	
\end{proof}
\sep
\begin{definition} \label{def:rich}
	Let $H$ be an hypergraph and let $r\in\N$. Let
	$X\subset V(H)$ be the set of vertices in $H$
	belonging to some saturated sub-hypergraph of diameter
	at most $2r+1$.	We say that $H$ is $(k,r)$-\textbf{rich}
	if for any $r^\prime\leq r$, any vertices $v_1,\dots, v_k$ 
	and any $\sim_k$ class $\mathcal{T}$ of trees with radius
	at most $r^\prime$ it holds that there exists a vertex $v\in V(H)$
	such that $d(v,X)> 2r^\prime+1$, $d(v,v_i)>2r^\prime+1$ for all
	$v_i$'s and that $T:=N(v;r^\prime)$ is a tree such that
	$(T,v)\in \mathcal{T}$ (notice that $T$ is a tree necessarily. 
	Otherwise $T$ contains a saturated sub-hypergraph with diameter 
	lesser or equal than $2r^\prime+1$).	
\end{definition}

\begin{theorem}\label{thm:Duplicatorwins}
	Let $H_1$, $H_2$ be hypergraphs. Let $r:=(3^k-1)/2$. Suppose
	that both $H_1$ and $H_2$ are $(k,r)$-rich and
	$H_1\approx_{k,r} H_2$. Then Duplicator wins $\ehr_{k}(H_1,H_2)$.
\end{theorem}
	\begin{proof}
		Because of the previous lemma we can apply 
		\cref{thm:DuplicatorAux} with $X\subset V(H_1)$ 
		and	$Y\subset V(H_2)$ the sets of vertices that belong
		to some saturated sub-hypergraph of $H_1$ or $H_2$ respectively
		with diameter at most $2r+1$.
	\end{proof}


\section{Probabilistic results}



\subsection{Almost all hypergraphs are simple}


We say that a connected hypergraph $G$ is \textbf{dense} if
$ex(G)>0$. Given $r\in \N$, we say that $G$ is \textbf{$r$-sparse}
if $G$ does not contain any dense subgraph $H$ such that 
$diam(H)\leq r$. The goal of this section is to show that, for any
fixed $r$, a.a.s $G_n$ is $r$-sparse.\par

\begin{lemma}
	Let $H$ be an hypergraph. Then 
	$\mathrm{E}\big[\# \text{ copies of }H \text{ in } G_n\big]\sim
	C\cdot(n^{-ex(H)})$ for some constant $C\in \R$ as $n$ tends to infinity.  
\end{lemma}
\begin{proof}
It holds
\[
 \mathrm{E}\big[\# \text{ copies of }H \text{ in } G_n\big]=
 \sum_{H^\prime \in Copies(H,[n])} \PR{H^\prime \subset G_n}.
\]	
We have that $|Copies(H,[n])|=\frac{(n)_{v(H)}}{\aut(H)}$. Also, 
for any $H^\prime \in Copies(H,[n])$ it is satisfied
\[
\PR{H^\prime \subset G_n}= \prod_{R\in \sigma} \left(\frac{\beta_R}{n^{ar(R)-1}} 
\right)^{|E_R(H)|}.
\]
Substituting in the first equation we get
\[
\mathrm{E}\big[\# \text{ copies of }H \text{ in } G_n\big]=
\frac{(n)_{v(H)}}{\aut(H)}\cdot
\prod_{R\in \sigma} \left(\frac{\beta_R}{n^{ar(R)-1}}\right)^{|E_R(H)|}
\sim
n^{-ex(H)} \cdot \frac{\prod_{R\in \sigma} \beta_R^{
|E_R{H}| }}{\aut(H)}.
\]	



%Given any one-to-one map $f\in [n]_{V(H)}$, let $X^H_{n,f}$ be 
%the indicator variable
%that takes value one when $f$ defines an monomorphism 
%between $H$ and $G_n$ and zero otherwise. 
%The probability that $X^H_{n,f}=1$ is exactly
%$\prod_{i=1}^t \left(\frac{\beta_i}{n^{a_i-1}}\right)^{e_i(H)}$ which can
%be written as $C\cdot n^{-ex(H)-v(H)}$ for some constant
%$C$ that does not depend on $f$ nor $n$. Define 
%$X^H_n=\sum_{f\in [n]_{V(H)}} X^H_{n,f}$. Then, by definition, the number of 
%copies of $H$ in $G_n$ is exactly $X^H_n/\aut(H)$.
%Taking into account that 
%\[
%\mathrm{E}[X^H_n]=(n)_{v(H)} \cdot C\cdot n^{-\sum_{i=1}^t (a_i-1)\cdot e_i(H)}= \Theta(n^{-ex(H)}),
%\]
%the result follows. 
%Indeed, let $n\in \N$ and let $G_n$ be a random hypergraph sampled from
% $\mathrm{G}^\mathcal{C}(n,\overline{p})$. Let
% $\overline{h}$ be an ordering of the vertices
%in $V(H)$. For any list 
% $\overline{v}\in [n]_{|V(H)|}$ (recall that $[n]=V(G)$), 
%let $X_{\overline{v}}$ be the random indicator variable
%that takes value $1$ if the natural map 
% $f:\overline{h}\rightarrow \overline{v}$ defines an homomorphism 
%between $H$ and $G$. That is, $X_{\overline{v}}$ takes value $1$
%with probability
% $ \prod_{i=1}^{t} \big(\beta_i n^{1-a_i} \big)^{|E_i(H)|},
% $
%and zero otherwise. The sum of all $X_{\overline{v}}$'s for 
%all $\overline{v}\in [n]_{|V(H)|}$ counts how many labelled 
%copies of $H$ are in $G_n$. Thus, 
%\[ \mathrm{E}\big[\# \text{labelled copies of }H \text{ in } G_n\big] 
%= \sum_{\overline{v}\in [n]_{|V(G)|}} 
%\prod_{i=1}^{t} \big(\beta_i n^{1-a_i} \big)^{|E_i(H)|}
%= (n)_{|V(H)|} \prod_{i=1}^{t} \big(\beta_i n^{1-a_i} \big)^{|E_i(H)|}.
%\]
%The expected number of copies of $H$ in $G_n$ is then this last
%expectation divided by the number of automorphisms of $H$, $\aut(H)$.
%In consequence,
%\begin{align*}
% &\Ln \mathrm{E}\big[\# \text{copies of }H \text{ in } G_n\big] 
%=\Ln \frac{(n)_{|V(H)|}}{\aut(H)} \prod_{i=1}^{t} 
%\big(\beta_i n^{1-a_i} \big)^{|E_i(H)|}
%=\\ &\Ln \Big[(n)_{|V(H)|}\prod_{i=1}^{t}n^{(1-a_i)|E_i(H)|}\Big]
%\frac{\prod_{i=1}^{t} \beta_i^{|E_i(H)|}}{\aut(H)} \simeq \Ln
%n^{-ex(H)}\frac{\prod_{i=1}^{t} \beta_i^{|E_i(H)|}}{\aut(H)},
%\end{align*}
%and last expression belongs to $\theta(n^{-ex(H)})$, as desired.\par
\end{proof}
\sep

As a corollary of last result we get the following:  
\begin{lemma} \label{lem:nocopiesdense}
	Let $H$ be an hypergraph such that $ex(H)>0$. Then
	a.a.s there are no copies of $H$ in $G_n$. 
\end{lemma}  
\begin{proof}
	Because of the previous fact, 
	$\mathrm{E}\big[\# \text{ copies of }H \text{ in } G_n\big] 
	\xrightarrow[]{n\to \infty} 0$ . An application of the first moment
	method yields the desired result. 
\end{proof} 
\sep


A similar result that will be useful later is the following:

%\subsubsection{Rooted hypergraphs}
%
%A rooted hypergraph $(G,\overline{u})$ is an hypergraph $H$ together with 
%an ordered sequence of distinguished vertices $\overline{u}\in (V(H))_*$.
%An isomorphism between two rooted hypergraphs $(G,\overline{u})$ and
%$(H,\overline{v})$ is a map $f: V(G)\rightarrow V(H)$ such that $f$
%is an isomorphism between $G$ and $H$ that satisfies the additional condition
%$f(\overline{u})=\overline{v}$. An automorphism of $(G,\overline{u})$  is 
%an isomorphism from $(G,\overline{u})$  to itself. 
%We write $Aut(G,\overline{u})$ to denote the group
%of automorphisms of $(G,\overline{u})$. \par
%
%Given a rooted hypergraph $(G,\overline{u})$, a set of vertices
%$V$ and a list $\overline{v}\in (V)_*$ such that $len(\overline{u})
%= len(\overline{v})$ we define the set $Copies((G,\overline{u}),
%(V,\overline{v}))$ as the set of rooted hypergraphs 
%$(H,\overline{v})$ isomorphic to $(G,\overline{u})$ such that
%$V(H)\subset V$. \par

\begin{lemma}\label{lem:nocopiesfixed}
	Let $H$ be an hypergraph. Let
	$\overline{v}\in (\N)_*$ be a list of vertices
	with $len(\overline{v})\leq |V(H)|$.
	For each $n\in \N$ 
	let $X_n$ be the random variable that
	counts the copies of $H$ in $G_n$ that contain the vertices
	in $\overline{v}$. Then
	\[
	\mathrm{E}\big[ X_n \big]=\Theta(n^{-ex(H)-len(\overline{v})}).\]
	In particular, given any fixed $r\in \N$,
	a.a.s the vertices $v\in \overline{v}$ satisfy that
	the $N(v;r)$'s are disjoint trees.	
\end{lemma}
\begin{proof}
	It holds that the number of hypergraphs $H^\prime \in Copies(H,[n])$
	that contain all vertices in $\overline{v}$ is
	 $\Theta\big(n^{|V(H)|-len(\overline{v})}\big)$. Then for
	 some constant $C$,
	\[
	\mathrm{E}\big[ X_n \big]\sim 
	C n^{|V(H)|-len(\overline{v})}
	\cdot\prod_{R\in\tau} \left( \frac{\beta_R}{n^{ar(R)-1}}\right)^{e_R(H)}=
	n^{-ex(H)-len(\overline{v})}\cdot C \cdot
	\prod_{R\in\tau} \left( \beta_R \right)^{e_R(H)}.
	\]
\end{proof}
\sep

%
%\begin{lemma}
%	Let $\overline{v}:=(v_1,\dots,v_j)\in \N*$.
%	Let $H$ be an hypergraph such that $ex(H)>-j$. 
%	Then a.a.s there is no copy of $H$ in $G_n$ that
%	contains all $v_1,\dots,v_j$. 
%\end{lemma} 
%\begin{proof}
%It is sufficient to show that
%\begin{equation}\label{eqn:aux}
%\mathrm{E}\big[\# \text{ copies of }H \text{ in } G_n \text{ containing }
%\overline{v} \big]\xrightarrow[]{n\to \infty} 0.
%\end{equation}
%Then, because of a first moment argument the result follows.\par
%Suppose $v(H)\geq j$. Otherwise the statement is trivial. 
%As before, given any $f\in [n]_{V(H)}$, let $X^H_{n,f}$ be the random
%variable that takes value one if $f$ is a monomorphism from $H$ to $G_n$
%and zero otherwise. The probability that $X^H_{n,f}$ takes value one is
%$C\cdot n^{-ex(H)-v(H)}$ for some constant $C$ independent from $f$ and $n$. 
%Let 
%\[ Y^H_{n,\overline{x}}= \sum_{\substack{f\in [n]_{V(H)}\\ 
%\overline{x}\subset Im(f)}} X^H_{n,f}.\]
%The number of functions $f\in [n]_{V(H)}$ such that $v_1,\dots,v_j\in Im(f)$
%is $\Theta(n^{v(H)-j})$. In consequence 
%$
%\mathrm{E}\big[Y^H_{n,\overline{v}}\big]=\Theta(n^{-ex(H)-j})
%$, and 
%\[\mathrm{E}\big[\# \text{ copies of }H \text{ in } G_n \text{ containing }
%\overline{v} \big]=\Theta(n^{-ex(H)-j}).
%\]
%This, together with $ex(H)>-j$, proves \cref{eqn:aux}. 
%\end{proof}

The main theorem of this section is the following
\begin{theorem} \label{thm:sparse}
	Let $r\in \N$. Then a.a.s $G_n$ is $r$-sparse. 
\end{theorem}

	The first moment method alone is not sufficient
	to prove our claim because the amount of dense 
	hypergraphs	$H$ such that $diam(H)\leq r$ is not finite
	in general. Thus, we need to prove that it suffices to
	prohibit a finite amount of dense sub-hypergraphs in order
	to guarantee that $G_n$ is $r$-sparse.\par
	Given an hypergraph $H$ and an edge $e\in E(H)$ we
	define the operation of \textbf{cutting} the edge 
	$e$ as removing $e$ from $H$ and then removing any isolated
	vertices from the resulting hypergraph. \par

\begin{lemma}
	Let $G$ be a
	dense hypergraph with diameter at most $r$.
	And let $H\subset G$ be a connected 
	sub-hypergraph with $ex(H)<ex(G)$. Then
	there is a connected
	sub-hypergraph $H^\prime \subset G$
	satisfying $H\subset H^\prime$,
	$ex(H)<ex(H^\prime)$ and that 
	$|E(H^\prime)|\leq |E(H)|+2\cdot(r) + 1$, 
\end{lemma}
\begin{proof}
	Suppose there is some edge $e\in E(G)\setminus E(H)$ with
	and $ex(e)\geq 0$. Let $P$ be a path
	of length at most $r$ joining $H$ and $e$ in $G$. 
	Then $H^\prime:=H\cup P \cup e$ satisfies the conditions of 
	the statement. \par
	Otherwise, all edges $e\in E(G)\setminus E(H)$ 
	satisfy $ex(e)=-1$. In this case we successively cut
	edges $e$ from $G$ such that $d(e, H)$ is the maximum possible
	(notice that this always yields a connected hypergraph)
	until we obtain an hypergraph $G^\prime$ with $ex(G^\prime)<ex(G)$.
	Let $e$ be the edge that was cut last. Then $V(G^\prime)\cap V(e)=
	ex(G)-ex(G^\prime)+1 \geq 2$. Let $v_1, v_2\in V(G^\prime)\cap V(e)$,
	and let $P_1$, $P_2$ be paths of length at most $r$ that join $H$
	with $v_1$ and $v_2$ respectively in $G^\prime$. Then the hypergraph
	$H^\prime:=H\cup e \cup P^1 \cup P^2$ satisfies the conditions 
	in the statement. 
\end{proof}
\sep
\begin{lemma}
	Let $G$ be a
	dense hypergraph of diameter at most 
	$r$. Then $G$ contains a connected 
	dense sub-hypergraph $H$ 
	with $|E(H)|\leq 4r+2$. 
\end{lemma}
\begin{proof}
	Apply the previous lemma twice in a row starting with $G$ and taking
	as $H$ a sub-hypergraph of $G$ consisting of a single vertex and 
	no edges. 
\end{proof}
\sep

In particular, if we define $l:=\max\limits_{R\in \sigma} \,\, ar(R)$
then last lemma implies that if $G$ is a dense hypergraph whose
diameter is at most $r$ then $G$ contains a dense sub-hypergraph
$H$ with $|V(H)|\leq l\cdot (4r+2)$.\par

Now we are in conditions to prove \cref{thm:sparse}.

\begin{proof}
	Because of last lemma there is a constant $R$ such that 
	``$G$ does not contain dense hypergraphs of size bounded by $R$" implies
	that ``$G$ is $r$-sparse". Thus,
	\[ \Ln \mathrm{Pr}\big( G_n \text{ is } r \text{-sparse}  \big)
	\geq \Ln \mathrm{Pr} \big( G_n \text{ does not contain dense 
	hypergraphs of size bounded by } R\big).\] 
	Because of	\cref{lem:nocopiesdense}, given any individual dense hypergraph,
	the probability that there are no copies
	of it in $G_n$ tends to $1$ as $n$ goes to infinity. Using that
	there are a finite number of $\sim$ classes of dense hypergraphs whose
	size bounded by	$R$ we deduce that the RHS of last inequality tends to $1$. 
\end{proof}
\sep
\begin{corollary}\label{cor:simple}
	Let $r\in \N$. Then a.a.s $G_n$ is $r$-simple.
\end{corollary}
\begin{proof}
	If some connected component of $Core(G_n;r)$ is not a cycle that means
	that either $G_n$ contains a dense hypergraph of diameter at most $4r+1$
	or that $G_n$ contains two cycles of diameter at most $2r+1$ that are at
	a distance at most $2r+1$. In the second case, considering the two cycles
	and the path joining them, $G_n$ contains a dense hypergraph of diameter
	bounded by $6r+3$. In consequence the fact that $G_n$ is $(6r+3)$-sparse
	implies that $G_n$ is $r$-simple. Because of the previous theorem 
	$G_n$ is a.a.s $(6r+3)$-sparse and the result follows. 
\end{proof}
	
\begin{lemma} \label{lem:disjointtrees}
	Let $\overline{v}\in (\N)_*$ and let $r\in \N$. Then
	a.a.s, for all vertices $v\in \overline{v}$ the neighborhoods 
	$N(v;r)$'s are all trees and they are all disjoint. 
\end{lemma}
\begin{proof}
	An application of the first moment method together with
	\cref{lem:nocopiesfixed} and the fact that there are a finite number
	of $\sim$ classes of paths whose length is at most $2r+1$ implies that
	a.a.s the $N(v;r)$'s are disjoint. \par
	Also, because of \cref{thm:sparse} a.a.s the $N(v;r)$'s are either 
	trees or unicycles. But if any of the $N(v;r)$'s was an unicycle then
	that would mean that in $G_n$ there exists a path $P$ of length at most
	$2r+1$ joining some vertex $v\in \overline{v}$
	with a cycle $C$ of diameter at most $2r+1$. Using \cref{lem:nocopiesfixed}
	again as well as the fact that the number of $\sim$ classes of the possible hypergraphs
	$P\cup C$ is finite we obtain that a.a.s no such $P$ and $C$ exist. 
	In consequence all the $N(v;r)$'s are disjoint trees as we wanted to prove. \par
\end{proof}

\subsection{Convergence to Poisson variables}

Our main tool for computing probabilities in the following sections
will be the following multivariate version of Brun's Sieve 
(Theorem 1.23, \cite{bollobas2001random}).


\begin{theorem} \label{thm:BrunSieve}
	Fix $k\in \N$. For each 
	$n\in \N$, let $X_{n,1},\dots, X_{n,l}$ be non-negative
	random integer variables over the same
	probability space. Let $\lambda_1,\dots,\lambda_l$ 
	be real numbers. Suppose for any $r_1,\dots,r_l \in \N$
	\[ 
	\Ln \mathrm{E}\left[
	\prod_{i=1}^{l} \binom{X_{n,i}}{r_i} \right]
	= \prod_{i=1}^{l} \frac{\lambda_i}{r_i !}.	
	\]
	Then the $X_{n,1},\dots,X_{n,l}$ converge in distribution to
	independent Poisson variables with means $\lambda_1,\dots,\lambda_l$ 
	respectively. 
\end{theorem}

To make use of last theorem we will need to employ the following observation.

\begin{obs} \label{obs:binomialmean} Let $X_1,\dots, X_l$ be non negative
	random integer variables over the same space. 
	Let $r_1,\dots,r_l\in \N$.	Suppose
	each $X_i$ is the sum of various indicator random variables
	(i.e. variables that only take the values $0$ and $1$)
	$X_i=\sum_{j=1}^{a_i} Y_{i,j}$. Define 
	$\Omega:=\prod_{i=1}^l \binom{[a_i]}{r_i}$. That is,
	the elements $\{S_i\}_{i\in[l]}\in \Omega$
	represent all the possible unordered choices of 
	$r_i$ indicator variables $Y_{i,j}$ for each $i\in [l]$.
	Then 
	\[
	\mathrm{E}\left[
	\prod_{i=1}^{l} \binom{X_i}{r_i}\right]=
	\sum_{\{S_i\}_{i\in [l]}} \Pr\left(
	\bigwedge_{\substack{i\in [l]\\ j\in S_i}} Y_{i,j}=1
	\right).	
	\]
\end{obs} 



\subsection{Probabilities of trees}

During this section we want to study
the asymptotic probability that the 
$r$-neighborhood of a given vertex $v\in \N$
in $G_n$ 
is a tree that belongs to a given $k$-equivalence
class of trees $\mathcal{T}$ with radius at most
$r$. That is, we want to know
\[\Ln \mathrm{Pr}\big( 
T:=N^{G_n}(v;r) \text{ is a tree, and } (T,v)\in \mathcal{T} \big). 
\]

Denote this limit by $\mathrm{Pr}[r,\mathcal{T}]$. Notice that the 
definition of  $\mathrm{Pr}[r,\mathcal{T}]$ does not depend by the
choice of $v$.\par

\begin{definition} \label{def:treeprobabilies}
We define $\Lambda$ and $M$ as the minimal families
of expressions with arguments $\{\beta_R\}_{R\in\sigma}$
that satisfy the conditions: \textbf{(1)}
$1\in \Lambda$, \textbf{(2)} 
for any $R\in \sigma$, any positive $b\in \N$,
and $\overline{\lambda} \in \Lambda^*$,
the expression $(\beta_R/b) \prod_{\lambda\in \overline{\lambda}}
\lambda$
belongs to $M$, \textbf{(3)}
for any $\mu\in M$ and any $n\in \N$ both
$\mathrm{Poiss}_{\mu}(n)$ and $\mathrm{Poiss}_\mu(\geq n)$ are in $\Lambda$, 
and  \textbf{(4)} for any $\lambda_1,\lambda_2 \in \Lambda$, the
product $\lambda_1\lambda_2$ belongs to $\Lambda$ as well.
\end{definition}

The goal of this section is to show 
that $\mathrm{Pr}[r,\mathcal{T}]$,
as an expression with parameters
$\{\beta_R\}_{R\in\sigma}$, belongs to $\Lambda$ for any choice of 
$r$ and $\mathcal{T}$. \par

Before we proceed it will be useful to define the following abbreviation
\begin{definition} Let $H$ be an hypergraph, $\overline{v}\in V(H)^*$, $v\in V(H)$ and
	$r\in \N$. Then we define $\mathrm{Tr}(H,\overline{v};\, v;\, r)$ as
	\[
	\mathrm{Tr}\Big( Core(H,\overline{v};\,\,r),\overline{v}; \,\, v     \Big).
	\]
\end{definition}

\begin{lemma}\label{lem:far_away}
	Let $\overline{v} \subset \N*$ be a finite set of fixed vertices and let 
	$\pi(\overline{x})$ be an edge sentence such that
	$len(\overline{x})=len(\overline{v})$. 
	Define $G_n^\prime=G_n \setminus E[\overline{v}]$ (i.e. $G_n$ minus all the
	edges induced over $\overline{v}$). Fix $R\in \N$. 
	\begin{itemize}
		\item Let $A_n$ be the event that $G^\prime_n$ contains a path of 
		length	at most	$R$ between any two vertices $u,w\in \overline{v}$.
		\item Let $B_n$ be the event
		that $G^\prime_n$ contains a cycle of diameter at most $R$
		at distance at most $R$ to some vertex $u\in \overline{v}$.
	\end{itemize}
	Then $\Ln \mathrm{Pr}\big(A_n \, | \, \pi(\overline{v})\big)=0$, and 
	$\Ln \mathrm{Pr}\big(B_n \, | \, \pi(\overline{v})\big)=0$. 
\end{lemma}
\begin{proof}
	Notice that the events $A_n$ and $B_n$ do not concern the possible edges
	induced over $\overline{v}$. In consequence, because edges are independent
	in our random model, 
	$\mathrm{Pr}\big(A_n \, | \, \pi(\overline{v})\big)
	=\mathrm{Pr}(A_n)$ and 
	$\mathrm{Pr}\big(B_n \, | \, \pi(\overline{v})\big)
	=\mathrm{Pr}(B_n)$.\par
	Now both $\Ln \Pr(A_n)=0$ and $\Ln Pr(B_n)=0$ follow from 
	\cref{lem:disjointtrees} using that $G^\prime_n\subset G_n$.  
\end{proof}

\sep
%\begin{corollary}\label{cor:disjointtrees}
%	Let $\overline{v}\in (\N)_*$ be a finite set of vertices 
%	and let $\pi(\overline{x})$ be an edge sentence such that
%	$len(\overline{v})=len(\overline{x})$. 
%	Define $G_n^\prime=G_n \setminus E[\overline{v}]$. 
%	Fix $r\in \N$.
%	Let $U_n$ be the event that for all $v\in \overline{v}$
%	the neighborhoods $N^{G^\prime_n}(v;r)$ are disjoint trees.
%	Then
%	\[
%	\Ln \PR{U_n \, \big| \, \pi(\overline{v})}=1.\]
%	As a consequence, for all $v\in \overline{v}$,
%	\[
%	\Pr\Big(Tr\big(G_n,\overline{v};\, v;\, r)=N^{G^\prime_n}(v;r)\, 
%	\Big| \, \pi(\overline{v}) \Big)=1.
%	\]
%\end{corollary}
%\begin{proof}
%	As in the previous lemma, because edges are independent 
%	in our random model, \[
%	\PR{U_n \, \big| \, \pi(\overline{v})}
%	=\PR{U_n}\]
%	 for all $n$. Because of the previous lemma
%	a.a.s there is no path in $G^\prime_n$ of length at most $2r+1$ joining 
%	any two vertices in $\overline{v}$. In consequence a.a.s
%	the $N^{G^\prime_n}(v;r)$'s are disjoint.\par
%	Also, because of \cref{thm:sparse} a.a.s there are no dense
%	hypergraphs with diameter at most $r$ in $G_n$, 
%	so a.a.s the  $N^{G^\prime_n}(v;r)$'s are not dense.\par
%	Finally, because of the previous lemma, a.a.s 
%	none of the vertices in $\overbar{v}$ is at a distance
%	lesser than $r$ to a cycle of diameter at most $r$.
%	In consequence a.a.s all the $N^{G^\prime_n}(v;r)$'s are trees, 
%	as we wanted to prove. 	 \par
%	The fact that for all $v\in \overline{v}$ a.a.s
%	$
%	Tr\big(G_n,\overline{v};\, v;\, r)=N^{G^\prime_n}(v;r)
%	$
%	is deduced from the fact that if some 
%	$u\in V\Big(N^{G^\prime_n}(v;r) \Big)$ does not belong
%	to $Tr\big(G_n,\overline{v};\, v;\, r)$, then 
%	either (a) $u$ belongs to a path in $G_n^\prime$
%	of length at most $2r+1$ 
%	between $v$ and another vertex $w\in \overline{v}$, 
%	(b) $u$ belongs to a cycle in $G^\prime_n$ of diameter
%	at most $2r+1$ which contains $v$, or
%	(c) $u$ belongs to a path in $G_n^\prime$
%	of length at most $2r+1$ between $v$ and some cycle
%	in $G_n^\prime$ of diameter at most $2r+1$. But as the
%	$N^{G^\prime_n}(v;r)$'s a.a.s are disjoint trees none of 
%	(a),(b),(c) hold with positive probability. 
%\end{proof}
%
%
%\sep

\begin{theorem} \label{thm:BigTrees}
	Fix $r\in \N$. The following are satisfied:
	\begin{itemize}
		\item[(1)] Let $\mathcal{T}$ be a
		$k$-equivalence class for trees with radii at most $r$. Let
		$v\in \N$ be a vertex.	Then 
		\[
		\mathrm{Pr}[r,\mathcal{T}]:=\Ln  \PR{Tr\big(G_n,v;\,\,v;\,\,r\big)\in \mathcal{T}}
		\]
		exists,
		is positive for all choices of 
		$\{\beta_R\}_{R\in \epsilon}\in [0,\infty)^{|\sigma|}$,
		and is an expression
		in $\Lambda$ that depends only on the choice of $r$ and $\mathcal{T}$.
		\item[(2)] Let $\overline{u}\in (\N)_*$ be a list of different fixed 
		vertices, and let $\pi(\overline{x})\in FO[\sigma]$ be a consistent
		edge sentence such that 
		$len(\overline{x})=len(\overline{u})$.
		Let $\overline{v}\in (\N)_*$ be vertices contained
		in $\overline{u}$. For each $v\in \overline{v}$
		let $\mathcal{T}_v$ be a $k$-equivalence class
		of trees with radii	at most $r$. Then
		\[
		\Ln \mathrm{Pr}\big( \bigwedge_{v\in \overline{v}} 
		Tr\big(G_n, \overline{u};\,\,v;\,\,r\big)\in \mathcal{T}_v 
		\, | \, \pi(\overline{u})
		\big)= \prod_{v\in \overline{v}} \mathrm{Pr}[r,\mathcal{T}_v]. \]	 	
	\end{itemize}
\end{theorem}
	We will devote the rest of this section to proving this
	theorem. The proof will be done by induction on $r$. 
	
\begin{lemma}
	Conditions (1) and (2) of \cref{thm:BigTrees} 
	are satisfied for $r=0$.
\end{lemma}
\begin{proof}
	Recall that	all trees with radius zero are $k$-equivalent. Thus,
	the limits
	appearing in conditions (1) and (2) are both equal to $1$.
\end{proof}
\sep

Now, for the induction step we have to prove that 
conditions (1) and (2) of \cref{thm:BigTrees} hold
for any $r>0$, given that they hold for all
$r^\prime < r$. Notice that (1) is, in fact, a particular
case of (2). Proving only (2) would be the ``mathematically"
reasonable way to proceed, but the proof of (2) does not contain
any new ideas that do not appear in the proof of (1) and is, in turn,
more convoluted notation-wise. In consequence we will offer the complete 
proof of (1) and later give indications about what changes in the proof
are necessary in order to show (2). 
	
	\begin{lemma} \label{lem:singletreeprob}
	Let $r\in \N$, $r>0$. Let $\mathcal{T}$ be a 
	$\sim_k$ class of trees with radii at most $r$ and 
	let $v\in \N$ be any vertex. 
	Suppose \cref{thm:BigTrees} holds
	for $r-1$. Then \[
	\Pr\big[r, \mathcal{T}\big]:=
	\Ln  \PR{Tr\big(G_n,v;\,\,v;\,\,r\big)\in \mathcal{T}}
	\] 
	exists, is positive for all choices of 
	$\{\beta_R\}_{R\in \epsilon}\in [0,\infty)^{|\sigma|}$,
	and is and expression in $\Lambda$ that depends
	only on the choice of $r$ and $\mathcal{T}$.   
	\end{lemma}
	\begin{proof}
		For any $(k,r)$-pattern  $\epsilon$  let 
		$X_{n,\epsilon}$ be the random variable that counts
		the initial edges $e\in E(T_n)$ whose pattern
		is $\epsilon$ (i.e. $(e,\tau_{(T_n,v)})\in \epsilon$). 
		Fix a pattern $\epsilon\in P(k,r)$.
		we define the expressions $\lambda_{r,\epsilon}$ and
		 $\mu_{r,\epsilon}$ 
		as follows: let $(e,\tau)$ be a representative of $\epsilon$
		whose root is $v$. Then
		\[
		\lambda_{r,\epsilon}=\prod_{\substack{u\in V(e)\\ u\neq v} } \Pr\big[
		r-1, \tau(u)\big], \quad \text{ and } \quad
		\mu_{r,\epsilon}=\frac{\beta_{R(e)}}{\aut(\epsilon)} 
		\cdot \lambda_{r,\epsilon}.
		\]
		Clearly the definitions of $\lambda_{r,\epsilon}$ and $\mu_{r,\epsilon}$
		are independent from the representative $(e,\tau)$ and by hypothesis
		depend only on the choice of $r$ and $\epsilon$. By hypothesis it also holds
		that $\mu_{r,\epsilon}$ is positive for all values of
		$\{ \beta_R \}_{R\in \sigma}\in [0,\infty)^{|\sigma|}$.
		 Furthermore, 
		$\mu_{r,\epsilon}$ belongs to $M$.\par
		First we are going to show that for any
		$(k,r)$-pattern $\epsilon$ it holds that
	 	\[
	 	\Ln E[X_{n,\epsilon}] = \mu_{r,\epsilon}.
	 	\]
	 	This step is not necessary for the proof as a whole but serves as a 
	 	simple example that showcases the methods used.		
		\par
		Fix a $(k,r)$-pattern $\epsilon$. 
		By definition $X_{n,\epsilon}$ counts
		the colored edges $(e,\tau)\in Copies(\epsilon,[n],v)$ 
		such that $e$ is an initial edge in $T_n$ satisfying that
		for any $u\in V(e)$ with $u\neq v$, it holds
		$Tr(T_n(v),u)\in \tau(u)$. Thus,
		\[
		\mathrm{E}\big[ X_{n,\epsilon} \big]
		= \sum_{(e,\tau)\in Copies(\epsilon,[n];(v,\rho))} \Pr
		\left(e\in E(T_n)
		\bigwedge_{\substack{u\in V(e)\\ u\neq v} } Tr(T_{n}, v; u)\in 
		\tau(u) \right).
		\]
		Because of the symmetry of our random model the probability 
		in the RHS of last equation is the same for all 
		$(e,\tau)\in Copies(\epsilon,[n];(v,\rho))$. Let
		$(e,\tau)\in Copies(\epsilon,\N;(v,\rho))$ be fixed. 
		Using that 
		$|Copies(\epsilon,[n];(v,\rho))|=\frac{(n)_{|e|-1}}
		{\aut(\epsilon)}$ we obtain
		\[
		\mathrm{E}\big[ X_{n,\epsilon} \big]
		=
		\frac{(n)_{|e|-1}}{|
			Aut(\epsilon)|} \Pr
		\left(e\in E(T_{n})
		\bigwedge_{\substack{u\in V(e)\\ u\neq v} } 
		Tr(T_n, v; u)\in \tau(u) \right).
		\]
		Also, it is satisfied
		\begin{align*}
		&\Pr
		\left( e\in E(T_{n})
		\bigwedge_{\substack{u\in V(e)\\ u\neq v} } Tr(T_n, v; u)\in 
		\tau(u) \right)=\\
		&\PR{ e\in E(G_n)} \cdot
		\Pr
		\left( e\in E(T_{n})
		\bigwedge_{\substack{u\in V(e)\\ u\neq v} } 
		Tr(T_{n}, v; u)\in \tau(u) \, \Bigg| \,  e\in E(G_n) \right)
		\end{align*}
		Because of \cref{lem:far_away}, a.a.s $T_n=N(v;r)$ so a.a.s if
		$e\in E(G_n)$
		and $v\in V(e)$, then $e\in E(T_n)$. Also,
		$\PR{e\in E(G_n)}=\frac{\beta_{R(e)}}{n^{|e|-1}}$. In consequence
		the RHS of last equation is asymptotically equivalent
		to
		\[
		\frac{\beta_{R(e)}}{n^{|e|-1}} \cdot
		\Pr	\left(
		\bigwedge_{\substack{u\in V(e)\\ u\neq v} } Tr(T_{n}, v; u)\in \tau(u) \, \Bigg| \,  e\in E(G_n) \right)
		\]
		Fix $\overline{u}\in (\N)_*$ a list that contains exactly the vertices
		in $e$. Notice that the event $e\in E(G_n)$ clearly can be described
		by an edge sentence
		over the vertices in $\overline{u}$.
		Because of \cref{lem:far_away},
		\[
		\Ln \Pr\Big(
		N^{G_n}(v;r) \text{ is a tree } \Big| \, \, e\in E(G_n) \Big)=1.
		\] 
		Set $G^\prime_n:=G_n \setminus e$.
		In the case that $T_n= N(v;r)$ and $e\in E(G_n)$ then
		the following chain of equalities holds for all 
		$u\in \overline{u}$ different from $v$:
		\[
		Tr(T_{n}, v, u)=N^{G^\prime_n}(u;\,r-1)=Tr(G_n,\overline{u};\, u;\, r-1).
		\] 
		Thus,
		\begin{align}
		\nonumber
		&\Pr	\left(
		\bigwedge_{\substack{u\in V(e)\\ u\neq v} } Tr(T_{n}, v; u)\in \tau(u) 
		\, \Bigg| \,  e\in E(G_n) \right)
		\sim\\ \label{eqn:equaltrees}
		&
		\Pr \left(
		\bigwedge_{\substack{u\in V(e)\\ u\neq v} }
		Tr(G_n,\overline{u};\, u;\, r-1)
		\in \tau(u) \, \Bigg| \,  e\in E(G_n) \right)
		\end{align}
	
		By hypothesis, the RHS of last equality
		is asymptotically equivalent to
		$\prod_{\substack{u\in V(e)\\ u\neq v} } \Pr\big[
		r-1, \tau(u)\big]=\lambda_{r,\epsilon}$.
		Finally, joining everything we obtain
		\[
		\Ln
		\mathrm{E}\big[ X_{n,\epsilon} \big]
		= \Ln \frac{(n)_{|e|-1}}{\aut(\epsilon)}
		\cdot  \frac{\beta_{R(e)}}{n^{|e|-1}}
		\prod_{\substack{u\in V(e)\\ u\neq v} }
		\Pr\big[
		r-1, \tau(u)\big] =
		\frac{\beta_{R(e)}}{\aut(\epsilon)} 
		\cdot \lambda_{r,\epsilon}= \mu_{r,\epsilon},
		\]
		as we wanted.\par
		Now we are going to prove that the variables $X_{n,\epsilon}$
		converge in distribution to independent Poisson variables 
		with mean values $\mu_{r,\epsilon}$ respectively. For
		each $\epsilon\in P(k,r)$ let $b_\epsilon\in \N$. We want 
		to show that
		\begin{equation} \label{eqn:binomexpedges}
		\Ln
		\mathrm{E}
		\left[
		\prod_{\epsilon\in P(k,r)} \binom{X_{n,\epsilon}}{b_\epsilon}	
		\right]
		= \prod_{\epsilon\in P(k,r)} \frac{(\mu_{r,\epsilon})^
		{b_\epsilon}}{b_\epsilon!}.	
		\end{equation}
		For each $n\in \N$ define
		\[
		\Omega_n:=\left\{
		(E_\epsilon)_{\epsilon\in P(k,r)} \quad 
		\Big | \quad \forall \epsilon\in P(k,r) \quad
		E_\epsilon\subset Copies(\epsilon,[n],(v,\rho)), 
		\quad |E_\epsilon|=b_\epsilon	
		\right\}		
		\]
		We also define $\Omega_\N$ substituting writing $\N$
		instead of $[n]$ in the definition of $\Omega_n$. 
		Informally, elements of $\Omega_n$ represent choices of 
		$b_\epsilon$ possible initial edges of $T_n$ whose $k$-
		pattern is $\epsilon$ for all $(k,r)$-patterns $\epsilon$. 	
		Using \cref{obs:binomialmean} we obtain
		\[
		\mathrm{E}
		\left[
		\prod_{\epsilon\in P(k,r)} \binom{X_{n,\epsilon}}{b_\epsilon}	
		\right]
		= 
		\sum_{
		(E_\epsilon)_{\epsilon\in P(k,r)}
		\in \Omega_n}
		\Pr\left(
		\bigwedge_{\substack{
		\epsilon\in P(k,r)\\
		(e, \tau)\in E_{\epsilon}
		}} \left(
		e\in E(T_n) \bigwedge_{\substack{
		u\in V(e)\\
		u\neq v}} Tr(T_n,v;u)\in \tau(u)		
		\right)
		\right). 		
		\]
		We say that an element $(E_\epsilon)_{\epsilon\in P(k,r)}$
		of $\Omega_n$ is \textbf{disjoint} each vertex 
		$w\in [n]\setminus\{v\}$ belongs to at most one edge 
		$(e,\tau)\in\bigcup_{\epsilon\in P(k,r)} E_\epsilon$.
		Notice that if we want the probability in the last sum to be greater 
		than $0$ for a particular $(E_\epsilon)_{\epsilon\in P(k,r)}
		\in \Omega_n$ then necessarily $(E_\epsilon)_{\epsilon\in P(k,r)}$
		is disjoint. 
		Indeed, suppose a vertex $w\in [n]\setminus \{v\}$ belongs to two different 
		edges $(e_1, \tau_1),$ $(e_2,\tau_2)\in \bigcup_{\epsilon\in P(k,r)} E_\epsilon$. 
		In consequence $e_1$ and $e_2$ form a cycle of diameter $1$, as they 
		both contain $v$ and $w$. This implies that $e_1, e_2 \notin E(T_n)$.
		\par
		For each $n\in \N$ let $\Omega_n^\prime\subset \Omega_n$ be the set
		of disjoint elements in $\Omega_n$. Then
		\[
		\mathrm{E}
		\left[
		\prod_{\epsilon\in P(k,r)} \binom{X_{n,\epsilon}}{b_\epsilon}	
		\right]
		= 
		\sum_{
			(E_\epsilon)_{\epsilon\in P(k,r)}
			\in \Omega_n^\prime}
		\Pr\left(
		\bigwedge_{\substack{
				\epsilon\in P(k,r)\\
				(e, \tau)\in E_{\epsilon}
		}} \left(
		e\in E(T_n) \bigwedge_{\substack{
				u\in V(e)\\
				u\neq v}} Tr(T_n,v;u)\in \tau(u)		
		\right)
		\right). 		
		\] 
		Also, because of the symmetry of the random model, 
		for all disjoint elements 
		$(E_\epsilon)_{\epsilon\in P(k,r)}$ the probability 
		in last sum is the same. In consequence, if we fix
		$(E_\epsilon)_{\epsilon\in P(k,r)}\in \Omega^\prime_\N$
		we obtain
		\begin{equation} \label{eqn:aux2}
		\mathrm{E}
		\left[
		\prod_{\epsilon\in P(k,r)} \binom{X_{n,\epsilon}}{b_\epsilon}	
		\right]
		= 
		|\Omega_n^\prime|\cdot
		\Pr\left(
		\bigwedge_{\substack{
				\epsilon\in P(k,r)\\
				(e, \tau)\in E_{\epsilon}
		}} \left(
		e\in E(T_n) \bigwedge_{\substack{
				u\in V(e)\\
				u\neq v}} Tr(T_n,v;u)\in \tau(u)		
		\right)
		\right). 		
		\end{equation}
	
		Counting vertices and automorphisms we get that
		\begin{equation} \label{eqn:aux3}
		|\Omega_n^\prime|= (n)_{\sum_{\epsilon\in P(k,r)} 
			(|\epsilon|-1)\cdot b_\epsilon}
		\prod_{\epsilon\in P(k,r)}
		\frac{1}{b_\epsilon!} \cdot
		\left( \frac{1}{\aut(\epsilon)} \right)^
		{b_\epsilon} .
		\end{equation}
		Because of \cref{lem:far_away} a.a.s if $e\in E(G_n)$
		and $v\in V(e)$, then $e\in E(T_n)$. In consequence:
		\begin{align} \label{eqn:aux4}
		&\Pr\left(
		\bigwedge_{\substack{
				\epsilon\in P(k,r)\\
				(e, \tau)\in E_{\epsilon}
		}} \left(
		e\in E(T_n) \bigwedge_{\substack{
				u\in V(e)\\
				u\neq v}} Tr(T_n;\, u)
			\in \tau(u)		
		\right)
		\right)\sim\\ \nonumber
		&\Pr\left(
		\bigwedge_{\substack{
				\epsilon\in P(k,r)\\
				(e, \tau)\in E_{\epsilon}
		}} e\in E(G_n)
		\right)\cdot 		
		\Pr\left(
		\bigwedge_{\substack{
				\epsilon\in P(k,r)\\
				(e, \tau)\in E_{\epsilon}\\
				u\in V(e)\\
				u\neq v
		}}Tr(T_n;\, u)\in \tau(u)	
		\, \Bigg| \, 
		\bigwedge_{\substack{
		\epsilon\in P(k,r)\\
		(e, \tau)\in E_{\epsilon}
		}} e\in E(G_n)
		\right).
		\end{align}
	Let $\overline{w}\in (\N)_*$ be a list containing exactly
	the vertices $u\in V(e)$ for all $e\in 
	\bigcup_{\epsilon\in P(k,r)} E_\epsilon$. 
	The event 
	$ \bigwedge_{\substack{
			\epsilon\in P(k,r)\\
			(e, \tau)\in E_{\epsilon}
	}} e\in E(G_n)$ clearly can be described via an edge sentence
	whose variables are interpreted as vertices in $\overline{w}$.
	Thus, analogously to \cref{eqn:equaltrees}
	we obtain 
	\begin{align} \label{eqn:aux5}
	&	\Pr\left(
	\bigwedge_{\substack{
			\epsilon\in P(k,r)\\
			(e, \tau)\in E_{\epsilon}
	}} \left(
	e\in E(T_n) \bigwedge_{\substack{
			u\in V(e)\\
			u\neq v}} Tr(T_n,v;u)\in \tau(u)		
	\right) \, \, \Big| \, \,
	\bigwedge_{\substack{
			\epsilon\in P(k,r)\\
			(e, \tau)\in E_{\epsilon}
	}} e\in E(G_n)\,	
	\right)\sim \\ \nonumber
	&
	\Pr\left(
	\bigwedge_{\substack{
			\epsilon\in P(k,r)\\
			(e, \tau)\in E_{\epsilon}
	}} \left(
	e\in E(T_n) \bigwedge_{\substack{
	u\in V(e)\\
	u\neq v}}
	Tr(G_n,\overline{w};\, u;\, r-1)\in \tau(u)		
	\right)\, \, \Big| 
	\, \,
	\bigwedge_{\substack{
	\epsilon\in P(k,r)\\
	(e, \tau)\in E_{\epsilon}
	}} e\in E(G_n)\,
	\right)
	\end{align}
	By hypothesis last probability is asymptotically equivalent
	to
	\[
	\prod_{\epsilon\in P(k,r)} (\lambda_{r,\epsilon})^{b_\epsilon}.
	\]
	Joining this with \cref{eqn:aux2},
	\cref{eqn:aux3}, \cref{eqn:aux4} and \cref{eqn:aux4} we obtain 
	\begin{align*} 
	\Ln
	\mathrm{E}
	\left[
	\prod_{\epsilon\in P(k,r)} \binom{X_{n,\epsilon}}{b_\epsilon}	
	\right]
	= \Ln  
	\frac{(n)_{\sum_{\epsilon\in P(k,r)}(|\epsilon|-1)\cdot b_\epsilon}}
	{n^{\sum_{\epsilon\in P(k,r)}(|\epsilon|-1)\cdot b_\epsilon}}
	\cdot
	\prod_{\epsilon\in P(k,r)}
	\frac{1}{b_\epsilon!} \cdot
	\left( \frac{\beta_{R(\epsilon)}}{\aut(\epsilon)} \right)^
	{b_\epsilon} \cdot 
	(\lambda_{r,\epsilon})^{b_\epsilon}&\\
	= \prod_{\epsilon\in P(k,r)}
	\frac{\left( \mu_{r,\epsilon} \right)^
	{b_\epsilon}}{b_\epsilon!}
	,&
	\end{align*}
	as we wanted. In consequence, by \cref{thm:BrunSieve},
	given $a_{\epsilon}\in \N$ for all $\epsilon\in P(k,r)$
	it holds
	\[
	\Ln
	\Pr\left( 
	\bigwedge_{\epsilon \in P(k,r)} X_{n,\epsilon}=a_\epsilon
	\right)=
	\prod_{\epsilon\in P(k,r)} e^{-\mu_{r,\epsilon}}
	\frac{(\mu_{r,\epsilon})^{a_\epsilon}}{a_\epsilon!}.
	\]

	Finally, using \cref{obs:equivalenttrees} we get that for some
	partition $E^1_\mathcal{T}, E^2_\mathcal{T}$ of $P(k,r)$ and 
	some natural numbers $a_\epsilon<k$ for each 
	$\epsilon\in E^2_\mathcal{T}$ it holds that
	\[ \PR[r,\mathcal{T}]=
	\Ln \PR{T_n\in \mathcal{T}}= 
	\left(
	\prod_{\epsilon\in E^1_\mathcal{T}} \mathrm{Poiss}_{\geq k}(\mu_{r,\epsilon}) 
	\right)\cdot
	\left(
	\prod_{\epsilon\in E^2_\mathcal{T}}
	\mathrm{Poiss}_{a_\epsilon}(\mu_{r,\epsilon}) 
	\right).
	\]
	And last expression belongs to $\Lambda$ as we wanted to prove. Furthermore,
	as the $\mu_{r,\epsilon}$'s are positive, this expression is also 
	positive for all values of
	$\{ \beta_R \}_{R\in \sigma}\in [0,\infty)^{|\sigma|}$.
 	\end{proof}
\sep


	In order for \cref{thm:BigTrees} to be completely proven the following result is needed
	
	\begin{lemma}
		Let $r\in \N$ be a positive number.
		Let $\overline{u}\in (\N)_*$ be a list of different fixed 
		vertices, and let $\pi(\overline{x})\in FO[\sigma]$ be an
		edge sentence such that 
		$len(\overline{x})=len(\overline{u})$.
		Let $\overline{v}\in (\N)_*$ be vertices contained
		in $\overline{u}$. For each $v\in \overline{v}$
		let $\mathcal{T}_v$ be a $k$-equivalence class
		of trees with radii	at most $r$. Suppose
		\cref{thm:BigTrees} holds for $r-1$. Then
		\[
		\Ln \mathrm{Pr}\big( \bigwedge_{v\in \overline{v}} 
		Tr\big(G_n, \overline{u};\,\,v;\,\,r\big)\in \mathcal{T}_v 
		\, | \, \pi(\overline{u})
		\big)= \prod_{v\in \overline{v}} \mathrm{Pr}[r,\mathcal{T}_v]. \]	 
	\end{lemma}
	\begin{proof}[Sketch of the proof.]
		The proof is completely analogous to the one 
		of the previous lemma but now with more 
		random variables. For each $v\in \overline{v}$ we define
		$T_{n,v}:=Tr\big(G_n, \overline{u};\,\,v;\,\,r\big)$.
		Given a $(k,r)$-pattern $\epsilon\in P(k,r)$ and a vertex
		$v\in \overline{v}$ we define the random variable 
		$X_{n,v,\epsilon}$ as the one that counts the number of initial
		edges $e\in E(T_{n,v})$ whose pattern is $\epsilon$. Similarly to last lemma
		one can show that the $X_{n,v,\epsilon}$ are asymptotically distributed
		like independent Poisson variables whose respective means are the
		$\mu_{r,\epsilon}$ defined in the previous lemma. As before, this is done using 
		\cref{thm:BrunSieve}, by computing the binomial moments of the $X_{n,v,\epsilon}$'s.
		Once the asymptotic distribution of those variables is computed the identity
		\[\Ln \mathrm{Pr}\big( \bigwedge_{v\in \overline{v}} 
		Tr\big(G_n, \overline{u};\,\,v;\,\,r\big)\in \mathcal{T}_v 
		\, | \, \pi(\overline{u})
		\big)= \prod_{v\in \overline{v}} \mathrm{Pr}[r,\mathcal{T}_v]
		\]
		follows easily using the definition of $\mathrm{Pr}[r,\mathcal{T}_v]$ 
		provided at the end of last proof. 
	\end{proof}

	\subsection{Almost all graphs are (k,r)-rich}
	
	\begin{theorem}\label{thm:rich}
		Let $r\in \N$. Then a.a.s $G_n$ is $(k,r)$-rich.
	\end{theorem}
	\begin{proof}
		Let $\Sigma$ be the set of all $\sim_k$-classes of rooted trees
		with radii at most $r$. For each $\mathcal{T}\in \Sigma$
		let $\overline{v}(\mathcal{T})\in (\N)_k$ be a $k$-tuple of 
		vertices such that all the $\overline{v}(\mathcal{T})$'s are
		disjoint. Given any $\overline{v}\in (N)_*$
		event $\phi_n\big(\overline{v}(\mathcal{T})\big)$ as the event that
		for any $v\in \overline{v}$, $N(v;r)\cup Core(G_n;r)\emptyset$ 
		(thus $N(v;r)$ is a tree),
		and that for any two $v_1,v_2\in \overline{v}$, 
		$d^{G_n}(v_1,v_2)>2r+1$. For each $n\in \N$ let
		\[
		A_n:=\bigwedge_{\mathcal{T}\in \Sigma} 
		\exists\,\, \overline{v}(\mathcal{T}) \in ([n])_k\,\,
		 \left(
		\phi_n(\overline{v}(\mathcal{T})) \bigwedge_{
		v\in \overline{v}(\mathcal{T})} 
		Tr\big(
		G_n;\,\,v;\,\,r
		\big)\in \mathcal{T}
		\right).
		\]
		Then the event $A_n$ that $G_n$ is $(k,r)$-rich.
		Indeed, suppose $G_n$ satisfies
		$A_n$. Let $\mathcal{T}\in \Sigma$ be a $\sim_k$ class and let 
		$v_1,\dots,v_{k-1}\in [n]$ be any vertices in $G_n$. Then because of
		$\phi_n(\overline{v}(\mathcal{T}))$ there is at least
		one vertex $v\in \overline{v}(\mathcal{T})$ such that $d(v,v_i)\geq 2r+1$
		for all $v_i$'s. It also holds that $T:=N(v;r)$ is a tree, and because of
		$A_n$, $(T,v)\in \mathcal{T}$. In consequence $G_n$ is $(k,r)$-rich.\par
		Now we show that $\Ln \PR{A_n}=1$. For that we will prove that for
		each $\mathcal{T}\in \Sigma$, a.a.s. the following event holds:
		\[
		B_{n,\mathcal{T}}:=
		\exists\,\, \overline{v}(\mathcal{T}) \in ([n])_k\,\,
		\left(
		\phi_n(\overline{v}(\mathcal{T})) \bigwedge_{
			v\in \overline{v}(\mathcal{T})} 
		Tr\big(
		G_n;\,\,v;\,\,r
		\big)\in \mathcal{T}
		\right).
		\]
		Fix $\mathcal{T}\in \Sigma$ and fix $\epsilon>0$ an arbitrarily 
		small real number. Let $m\in \N$,
		and let $\overline{v}\in (\N)_m$ be a $m$-tuple of vertices. Let 
		$X_{n,\overline{v}}$ be the random variable that counts the number of 
		vertices $v\in \overline{v}$ such that $\mathrm{Tr}(G_n;\,\,v;\,\,r)$
		belongs to $\mathcal{T}$. Because of \cref{thm:BigTrees}, $X_{n,\mathcal{v}}$
		converges in distribution to a binomial variable with parameters $m$ and
		$\Pr[r, \mathcal{T}]$. That is, for each $l\in [m]$,
		\[
		\Ln \PR{X_{n,\overline{v}}=l}= \binom{m}{l} \Pr[r, \mathcal{T}]^l \cdot \left(
		1-\Pr[r,\mathcal{T}]\right)^{m-l}.
		\]		    
		In particular, as $\Pr[r, \mathcal{T}]$ is greater than zero, for $m$ sufficiently
		big it holds that
		\[
		\Ln \PR{X_{n,\overline{v}}\geq k}> 1- \epsilon.
		\]
		Also, because of \cref{lem:far_away} we have that $\Ln \phi_n(\overline{v})=1$,
		and in consequence for $m$ sufficiently big
		\[
		\Ln \Pr\left(
		(X_{n,\overline{v}}\geq k)
		\wedge \phi(\overline{v})
		\right)> 1- \epsilon.
		\]
		As $(X_{n,\overline{v}}\geq k)
		\wedge \phi(\overline{v})$ implies $B_{n,\mathcal{T}}$ we have that
		$\Ln \PR{B_{n,\mathcal{T}}}=1$. As this holds for all $\mathcal{T}\in \Sigma$
		and $\Sigma$ is a finite set, then $\Ln \PR{A_n}=1$ as well and the result follows. 
	\end{proof}
		
	\subsection{Probabilities of cycles}
	
\begin{definition}
	We define $\Gamma$ and $\Upsilon$ as the minimal families of expressions with arguments
	$\{\beta_R\}_{R\in \sigma}$ that satisfies the conditions:
	\begin{itemize}
	\item[(1)] For each $R\in \sigma$ let $a_R\in \N$ be any natural number. Then, given any positive number $b\in \N$
	and any $\lambda\in \Lambda$ the expression $\frac{\lambda}{b}\cdot \prod_{R\in \sigma} \beta_R^{a_R}$
	belongs to $\Gamma$.
	\item[(2)] Given any $\gamma\in \Gamma$ and any $a\in \N$, the expressions $\mathrm{Poiss}_a(\gamma)$
	and $\mathrm{Poiss}_{\geq a}(\gamma)$ both belong to $\Upsilon$.
	\item[(3)] If $\upsilon_1, \upsilon_2\in \Upsilon$ then 
	$\upsilon_1\cdot \upsilon_2 \in \Upsilon$ as well. 
	\end{itemize}
\end{definition}

\begin{theorem}\label{thm:agreeabilityprobabilities}
	Let $\mathcal{O}$ be a simple $k$-agreeability class
	of hypergraphs. Then 
	$\Ln \PR{G_n\in \mathcal{O}}$ exists and is an expression
	in $\Upsilon$. 
\end{theorem}
\begin{proof}
	Define $r:=3^k$. 
	For each $O\in C(k,r)$ let $X_{n,O}$ be the random variable
	that counts the number of cycles in $Core(G_n;r)$ 
	whose $k$-type is $O$. Fix $O\in C(k,r)$.
	For any $O\in C(k,r)$ we define $\lambda_O$ and 
	$\gamma_O$ in the following way. Let $(H,\tau)$ be
	a representative of $O$. Then
	\[ 
	\lambda_O:
	=\prod_{v\in V(H)}
	\Pr \big[ r, \tau(v) \big],
	\]
	and
	\[
	\gamma_O:=
	\frac{\prod_{R\in \sigma} \beta_R^{|E_R(H)|}}
	{\aut(H,\tau)}\cdot\lambda_O.
	\]	
	As in the proof of 
	\cref{lem:singletreeprob} one can show that
	\[
	\Ln \mathrm{E}\big[ X_{n,O}\big]
	= \gamma_O
	\].
	Notice that the expression $\gamma_O$ both belongs to $\Gamma$ and
	does only depend on the $(k,r)$-cycle $O$. \par
%	Because of the symmetry of the random model last probability is the 
%	same for all $(H,\tau)\in Copies(O,[n])$. Fix $(H,\tau)\in
%	Copies(O,\N )$. Then
%	\begin{align*}
%	\mathrm{E}\big[ X_{n,O}\big]
%	&=\frac{(n)_{|V(H)|}}{\aut(H,\tau)} 
%	\cdot \Pr\left(
%	H\subset G_n \bigwedge_{v\in V(H)} 
%	Tree(G_n, v;r)\in \tau(v)
%	\right)	\\
%	&=\frac{(n)_{|V(H)|}}{\aut(H,\tau)}\cdot
%	\frac{\prod_{R\in \sigma} \beta_R^{|E_R(H)|}}{n^{|V(H)|}}\cdot
%	\Pr\left(\bigwedge_{v\in V(H)} 
%	Tr(G_n, v;r)\in \tau(v)
%	\quad  \Big| \quad H\subset G_n
%	\right)\\
%	&\sim 
%	\frac{\prod_{R\in \sigma} \beta_R^{|E_R(H)|}}
%	{\aut(H,\tau)}\cdot
%	\Pr\left(\bigwedge_{v\in V(H)} 
%	Tr(G_n, v;r)\in \tau(v)
%	\quad  \Big| \quad H\subset G_n
%	\right)
%	\end{align*}
%	Let $\overline{v}\in (\N)_*$ be a list containing 
%	exactly the vertices in $V(H)$. If $H\subset G_n$
%	then $Tr(G_n, v;r)=Tr(G_n,\overline{v},v;r)$. Also,
%	the event $H\subset G_n$ clearly can be described via
%	an edge sentence concerning the vertices in $\overline{v}$.
%	In consequence, using \cref{thm:BigTrees}, last expression
%	is asymptotically equivalent to
%	\[
%	\frac{\prod_{R\in \sigma} \beta_R^{|E_R(H)|}}
%	{\aut(H,\tau)}\cdot
%	\prod_{v\in V(H)}
%	\Pr \big[ r, \tau(v) \big].
%	\]
	
	We are going to prove that the variables $X_{n,O}$ converge
	in distribution as $n$ tends to infinity to 
	independent Poisson variables whose respective means are
	the $\gamma_O$. For that we are going to use again the factorial 
	moments method. For each $O\in C(k,r)$ fix a number $b_{O}\in\N$.
	We want to prove 
	\[
	\Ln 
	\mathrm{E}\left[
	\prod_{O\in C(k,r)}
	\binom{X_{n,O}}{b_O}
	\right]= \prod_{O\in C(k,r)} 
	\frac{(\gamma_O)^{b_O}}{b_O!}.
	\]
	For each $n\in \N$ we define
	\[
	\Omega_n:=\left\{
	(F_O)_{O\in C(k,r)} \quad \Big|
	\quad \forall O\in C(k,r) \quad
	F_O\subset Copies(O,[n]), \quad
	|F_O|=b_O	
	\right\}.
	\]
	We also define $\Omega_\N$ by substituting $[n]$ for $\N$ in
	the definition of $\Omega_n$. Informally, an element of $\Omega_n$ 
	represents a choice of an unordered $b_O$-tuple
	of possible cycles over $[n]$ whose $(k,r)$-type is $O$, for each
	$(k,r)$-cycle $O$. Using observation \cref{obs:binomialmean} we obtain
	\[
	\mathrm{E}\left[
	\prod_{O\in C(k,r)}
	\binom{X_{n,O}}{b_O}
	\right]=
	\sum_{(F_O)_{O\in C(k,r)}\in \Omega_n}
	\Pr\left(
	\bigwedge_{
	\substack{
	O\in C(k,r)\\
	(H,\tau)\in F_O
	}}
	\left(
	H\subset G_n
	\bigwedge_{v\in V(H)}
	Tr(G_n,v;r)\in \tau(v)
	\right)
	\right).
	\]
	Consider the subset $\Omega_n^\prime\subset \Omega_n$ that contains
	the elements $(F_O)_{O\in C(k,r)}\in \Omega_n$ such that there exists
	some vertex $v\in [n]$ contained in two graphs
	$(H_1,\tau_1),(H_2,\tau_2)\in \bigcup_{O\in C(k,r)} F_O$. We want to argue 
	that
	\begin{equation}\label{eqn:denseconfigurations}
	\Ln
	\sum_{(F_O)_{O\in C(k,r)}\in \Omega_n^\prime}
	\Pr\left(
	\bigwedge_{
		\substack{
			O\in C(k,r)\\
			(H,\tau)\in F_O
	}}
	\left(
	H\subset G_n
	\bigwedge_{v\in V(H)}
	Tr(G_n,v;r)\in \tau(v)
	\right)
	\right)=0.
	\end{equation}
	Given an element $(F_O)_{O\in C(k,r)}\in \Omega_n$ we 
	define the hypergraph $G\Big((F_O)_{O\in C(k,r)}\Big)$ as
	follows:
	\[
	G\Big((F_O)_{O\in C(k,r)}\Big):=
	\bigcup_{H\in F} H, 
	\]
	where
	\[
	F:=\left\{
	H \quad \Big| \quad 
	(H,\tau)\in \bigcup_{O\in C(k,r)} F_O	
	\right\}.
	\]
	That is, $G\Big((F_O)_{O\in C(k,r)}\Big)$
	is the union of all hypergraphs chosen in 
	$(F_O)_{O\in C(k,r)}$. Then, for all 
	$(F_O)_{O\in C(k,r)}\in \Omega_n$ it is 
	satisfied
	\begin{align*}
	&\Pr\left(
	\bigwedge_{
		\substack{
			O\in C(k,r)\\
			(H,\tau)\in F_O
	}}
	\left(
	H\subset G_n
	\bigwedge_{v\in V(H)}
	Tr(G_n,v;r)\in \tau(v)
	\right)
	\right)
	\leq 
	\Pr\left(
	\bigwedge_{
		\substack{
			O\in C(k,r)\\
			(H,\tau)\in F_O
	}}
	H\subset G_n
	\right) =
	\\&
	\Pr \left(
	G\Big((F_O)_{O\in C(k,r)}\Big) \subset G_n
	\right).
	\end{align*}
	Let 
	\[
	t = \sum_{O\in C(k,r)} |V(O)|\cdot b_O.
	\]
	Then $V\Big(
	G\Big((F_O)_{O\in C(k,r)}\Big)
	\Big) \leq t$
	for any $(F_O)_{O\in C(k,r)}\in \Omega_n$.\par
	Consider the following facts
	\begin{itemize}
		\item[(1)] If $(F_O)_{O\in C(k,r)}\in \Omega_n^\prime$ then
		$G\big((F_O)_{O\in C(k,r)}\big)$ is dense.
		 
		\item[(2)] Given an hypergraph $H$ with $V(H)\subset \N$,
		the number of elements  $(F_O)_{O\in C(k,r)}\in \Omega_\N^\prime$
		such that $H=G\big((F_O)_{O\in C(k,r)}\big)$ is finite and it is 
		the same for all $H^\prime \simeq H$ with $V(H^\prime) \subset \N$.
		
		\item[(3)] There is a finite amount of unlabeled dense
		hypergraphs with size bounded by $t$.
	\end{itemize}
	Then it follows that
	\begin{align*}
	\sum_{(F_O)_{O\in C(k,r)}\in \Omega^\prime_n}
	\Pr \Big(
	G\Big(&(F_O)_{O\in C(k,r)}\Big) \subset G_n
	 \Big) \\
	& = 
	\mathrm{O} \left(
	\mathrm{E}\left[
	\# \text{ of dense subgraphs in $G_n$ with size bounded by
	 $t$}\right] \right).	
	\end{align*}
	And this, together with \cref{lem:nocopiesdense} proves
	\cref{eqn:denseconfigurations}.\par
	For all $n$ define $\Omega_n^{\prime\prime}=
	\Omega_n\setminus \Omega_n^\prime$. That is, 
	$\Omega_n^{\prime\prime}$ contains the elements
	$(F_O)_{O\in C(k,r)}$ in $\Omega_n$ such that all vertices
	$v\in [n]$ belong to at most one hypergraph
	$(H,\tau)\in \bigcup_{O\in C(k,r)} F_O$. We also
	define $\Omega_\N^{\prime\prime}$. Because of 
	\cref{eqn:denseconfigurations} we have
	\[
	\mathrm{E}\left[
	\prod_{O\in C(k,r)}
	\binom{X_{n,O}}{b_O}
	\right]=
	\sum_{(F_O)_{O\in C(k,r)}\in \Omega_n^{\prime\prime}}
	\Pr\left(
	\bigwedge_{
	\substack{
		O\in C(k,r)\\
		(H,\tau)\in F_O
	}}
	\left(
	H\subset G_n
	\bigwedge_{v\in V(H)}
	Tr(G_n,v;r)\in \tau(v)
	\right)
	\right)+o(1).
	\]
	
	Because of the symmetry of the model the probability inside of last sum 
	is the same for all elements $(F_O)_{O\in C(k,r)}\in \Omega_n^{\prime\prime}$.
	Also, counting all different vertices and automorphisms we obtain that
	\[
	|\Omega_n^{\prime\prime}|=
	\frac{(n)_{\sum_{O\in C(k,r)} |V(O)|\cdot b_O}}
	{\prod_{O\in C(k,r)} b_O!\cdot \aut(O)^{b_O}}.
	\]
	Fix $(F_O)_{O\in C(k,r)}\in \Omega_\N^{\prime\prime}$. Then 
	\begin{align*}
	&\Ln \mathrm{E}\left[
	\prod_{O\in C(k,r)}
	\binom{X_{n,O}}{b_O}
	\right]=\\
	&
	\Ln
	\frac{(n)_{\sum_{O\in C(k,r)} |V(O)|\cdot b_O}}
	{\prod_{O\in C(k,r)} b_O!\cdot \aut(O)^{b_O}} \cdot
	\Pr\left(
	\bigwedge_{
		\substack{
			O\in C(k,r)\\
			(H,\tau)\in F_O
	}}
	\left(
	H\subset G_n
	\bigwedge_{v\in V(H)}
	Tr(G_n,v;r)\in \tau(v)
	\right)
	\right).
	\end{align*}
	It holds that the probability in last expression equals
	\[
	\prod_{O\in C(k,r)}
	\left( 
	\frac{\prod_{R\in \sigma} \beta_R^{|E_R(O)|}}{n^{|V(O)|}}
	\right)^{b_O} \cdot
	\Pr\left(
	\bigwedge_{
		\substack{
			O\in C(k,r)\\
			(H,\tau)\in F_O\\
			v\in V(H)
	}}
	Tr(G_n,v;r)\in \tau(v) \,
	\Bigg|  \,
	\bigwedge_{
		\substack{
			O\in C(k,r)\\
			(H,\tau)\in F_O\\
		}}
	H\subset G_n
	\right).
	\]
	Let $\overline{v}\in (\N)_*$ be 
	a list that contains exactly the vertices in $G\big(
	(F_O)_{O\in C(k,r)}	\big)$. Then the event
	\[
	A_n:=
		\bigwedge_{
		\substack{
			O\in C(k,r)\\
			(H,\tau)\in F_O\\
	}}
	H\subset G_n
	\]
	can be written as an edge sentence concerning the vertices in $\overline{w}$.
	Also, if $A_n$ holds then all vertices in $\overline{w}$ belong to 
	$Core(G_n;r)$. Thus, for all $v\in \overline{v}$,
	$Tr(G_n,v;r)=Tr(G_n,\overline{w};r)$ and using \cref{thm:BigTrees}
	we obtain
	\[
	\Pr\left(
	\bigwedge_{
		\substack{
			O\in C(k,r)\\
			(H,\tau)\in F_O\\
			v\in V(H)
	}}
	Tr(G_n,v;r)\in \tau(v) \,
	\Bigg|  \,
	\bigwedge_{
		\substack{
			O\in C(k,r)\\
			(H,\tau)\in F_O\\
	}}
	H\subset G_n
	\right) \sim
	\prod_{O\in C(k,r)} (\lambda_O)^{b_O}.	
	\]
	Joining everything together we obtain
	\begin{align*}
&\Ln \mathrm{E}\left[
\prod_{O\in C(k,r)}
\binom{X_{n,O}}{b_O}
\right]=\\
&
\Ln
\frac{(n)_{\sum_{O\in C(k,r)} |V(O)|\cdot b_O}}
{\prod_{O\in C(k,r)} b_O!\cdot \aut(O)^{b_O}} \cdot
\prod_{O\in C(k,r)}
\left( 
\frac{\lambda_O \cdot \prod_{R\in \sigma} \beta_R^{|E_R(O)|}}{n^{|V(O)|}}
\right)^{b_O} = \\
&\prod_{O\in C(k,r)} \frac{1}{b_O!}\left(\frac{\lambda_O  
\cdot \prod_{R\in \sigma} \beta_R^{|E_R(O)|} }{\aut(O)}
\right)^{b_O}= \prod_{O\in C(k,r)} 
\frac{(\gamma_O)^{b_O}}{b_O!},
\end{align*}
as we wanted. With this, because of \cref{thm:BrunSieve}, it 
is proven that when $n$ tends to infinity
the $X_{n,O}$'s are asymptotically distributed 
like independent
Poisson variables with the $\gamma_O$'s as their respective means.\par
Fix a $(k,r)$-agreeability class of $r$-simple hypergraphs
$\mathcal{O}$. 
Because \cref{obs:agreeablecores} it holds that
there is a partition $C_1, C_2\subset C(k,r)$, $C_1\cup C_2=C(k,r)$ 
and natural numbers $a_O\leq k-1$ for any $O\in C_2$ such that
$C_1, C_2, (a_O)_{O\in C_2}$ depend only on $\mathcal{O}$ and
\begin{align*}
 & \Ln
 \PR{G_n \in \mathcal{O}}=\\
 &\Ln
 \Pr \left(
 G_n \text{ is $r$-sparse } \wedge
 \left(
 \bigwedge_{O\in C_1}
 X_{n,O}\geq k
 \right)
 \wedge
 \left(
 \bigwedge_{O\in C_1}
 X_{n,O}=a_O.
 \right)
 \right).
\end{align*}
Because of \cref{thm:sparse}, last limit equals
\begin{align*}
&\Ln
\Pr \left(
\left(
\bigwedge_{O\in C_1}
X_{n,O}\geq k
\right)
\wedge
\left(
\bigwedge_{O\in C_1}
X_{n,O}=a_O
\right)
\right)
=\\
&
\left(
\prod_{O\in C_1}
\mathrm{Poiss}_{\geq k}(\gamma_O)
\right)
\cdot
\left(
\prod_{O\in C_2}
\mathrm{Poiss}_{a_O}(\gamma_O)
\right)
.
\end{align*}
This last expression belongs to $\Upsilon$, so the theorem is proven. 
\end{proof}

\section{Proof of the main theorem} \label{sect:main}

\begin{theorem}
	Let $\phi\in FO[\sigma]$. Then the function 
	$F_\phi: [O,\infty)^{|\sigma|}\rightarrow [0,1]$
	given by
	\[
	\{\beta_R\}_{R\in \sigma} \mapsto
	\Ln \Pr \left(
	G_n\left(\{\beta_R\}_{R\in \sigma}\right) \models \phi
	\right)
	\]
	is well defined and it is given by a finite sum of expressions
	in $\Theta$.
\end{theorem}
\begin{proof}
	Let $k$ be the quantifier rank of $\phi$ and
	let $r=3^k$. Let 
	$G_n:=G_n\left(\{\beta_R\}_{R\in \sigma}\right)$.
	Using \cref{cor:simple} we obtain
	\[
	\Ln \Pr \left(
	G_n \models \phi
	\right)=
	\Ln \Pr \left(
	G_n\models \phi
	\, \Big|
	\, G_n
	\text{ is $r$-sparse}
	\right).
	\]
	Let $\Sigma$ be the set of $(k,r)$-agreeability classes of 
	$(k,r)$-simple hypergraphs. 
	Then
	\begin{equation} \label{eq:aux1}
	\Ln \Pr \left(
	G_n \models \phi
	\right)=
	\Ln
	\sum_{\mathcal{O}\in \Sigma} \Pr\left(
	G_n\in \mathcal{O}
	\right) \cdot 
	\Pr\left(
	G_n\models \phi \,
	\Big| \,
	G_n\in \mathcal{O}
	\right).
	\end{equation}
	Notice that, because the set $\Sigma$ is finite, 
	this is the limit of a finite sum and we can exchange
	summation and limit. Also, 
	using \cref{thm:rich}, we obtain that for any $\mathcal{O}\in \Sigma$
	it holds
	\[
	\Ln \Pr\left(
	G_n\models \phi \,
	\Big| \,
	G_n\in \mathcal{O}
	\right) = 
		\Ln \Pr\left(
	G_n\models \phi \,
	\Big| \,
	G_n\in \mathcal{O}
 	\text{ and } G_n \text{ 
	is $(k,r)$-rich} 	\right).
	\]
	Because \cref{thm:Duplicatorwins} we have that given any two hypergraphs
	$H_1$ and $H_2$ such that $H_1$ and $H_2$ are $(k,r)$-agreeable and they
	are both $(k,r)$-rich then they both satisfy the same first order sentences
	with quantifier rank at most $k$. Then the LHS of last equation always equals
	either zero or one. Let $\Sigma^\prime\subset \Sigma$ be the set of classes 
	$\mathcal{O}$ for which last limit equals one. Then
	\[
	\Ln \Pr \left(
	G_n \models \phi
	\right)=
	\sum_{\mathcal{O}\in \Sigma^\prime}
	\Ln \Pr\left(
	G_n \in \mathcal{O}
	\right).		
	\]
	Because of \cref{thm:agreeabilityprobabilities} we know that
	each of the limits inside last sum exists and is given by
	an expression that belongs to $\Theta$. As a consequence the
	theorem follows. 	
\end{proof}

\section{Application to random SAT}

We will define a binomial model of random CNF formulas, 
in analogy with the one in \cite{chvatal1992mick},
but clearly the generality
in \cref{thm:main} allows for many modifications. \par

\begin{definition}
Given a variable $x$ both expressions
$x$ and $\neg x$ are called \textbf{literals}. A \textbf{clause}
is a set of literals. A clause $C$ is called \textbf{ordinary}
if no variable $x$ satisfies that both $x$ and $\neg x$
belong to $C$. An \textbf{assignment} over a set of variables $X$ is a 
map $f$ that assigns $0$ or $1$ to each variable of $X$. A clause $C$
is \textbf{satisfied} by an assignment $f$ if either there is some variable $x$
such that $x\in C$ and $f(x)=1$ or there is some variable $x$ such that
$\neg x\in C$ and $f(x)=0$. 
Given a natural number $l\in\N$
a $l$-\textbf{CNF formula} is a set of  ordinary clauses 
that contain exactly $l$ literals. 
We say that a formula $F$
over the variables $x_1,\dots, x_n$ is \textbf{satisfiable} if there is an
assignment $f:\{x_1,\dots, x_n\}\rightarrow \{0,1\}$ that satisfies all clauses
for any clause $C\in F$. 
\end{definition}

Given $n, l \in \N$ and a real number $0\leq p \leq 1$ we define
the random model $F(l,n,p)$ as the discrete probability space that
assigns to each $l$-CNF formula $F$ formed of clauses over
the variables $\{x_i\}_{i\in [n]}$
 the probability
\[
\PR{F}= p^{|F|}\cdot (1-p)^{2^l\binom{n}{l}-|F|}.
\] 
Equivalently, a random formula in $F(l,n,p)$ is obtained
by choosing each one of the $2^l\binom{n}{l}$ normal 
clauses of size $l$ over the variables $\{x_i\}_{i\in [n]}$
with probability $p$ independently. \par	
We can model $l$-CNF formulas, as we have defined them, as relational 
structures with a language $\sigma$ consisting of $l+1$ relation symbols
$R_0,\dots, R_l$ with arity $l$. We do that in such a way that the expression
$R_j(x_{i_1},\dots,x_{i_l})$ means that our formula contains the clause
consisting of $\neg x_{i_1}, \dots, \neg x_{i_j}$ and $x_{i_{j+1}},\dots
x_{i_l}$. In consequence we need $R_1,\dots, R_l$ to satisfy the 
following additional axioms:
\begin{itemize}
	\item For each $0\leq j \leq l$ and 
	for any variables $y_1,\dots,y_j$, $y_{j+1},\dots, y_l$ it holds that
	 $R_j(y_1,\dots y_j, y_{j+1},\dots, y_l)$ if and only if it still holds after 
	 applying any permutations on the variables $y_1,\dots,y_j$ and the variables
	 $y_{j+1},\dots,y_l$.
	 \item For each $0\leq j \leq l$ and 
	 for any variables $y_1,\dots, y_l$ it holds that
	 $R_j(y_1,\dots, y_l)$ only if all the $y_i$'s are different. 
\end{itemize}
Call $\mathcal{C}$ to the family of $\sigma$-structures satisfying these last two axioms.
Then, there is a clear correspondence between $l$-CNF formulas over the variables
$\{x_i\}_{i\in [n]}$ and the structures in $\mathcal{C}$ whose universe is 
$\{x_i\}_{i\in [n]}$. \par
The language $\sigma$ and the family $\mathcal{C}$ satisfy the conditions in 
\cref{sect:structures}. The random model $F_l(n,p)$ coincides with the model
$G(n,\{p_R\}_{R\in \sigma})$ of random $\mathcal{C}$-hypergraphs described in 
\cref{sect:random} when all the $p_R$'s are equal. In consequence the following theorem 
holds
\begin{theorem} \label{thm:mainsat}
	Let $l>1$ be a natural number.
	For each $n\in \N$ let $F_n(\beta)$ be a random formula from
	$F(l,n,\beta/n^{l-1})$. Then for each sentence $\Phi\in FO[\sigma]$ 
	it is satisfied
	that the map $f_\Phi: [0,\infty) \rightarrow \R$ given by
	\[
	\beta \mapsto \PR{ F_n(\beta)\models \Phi}
	\]
	is well defined and analytic. 
\end{theorem}
%As in last theorem, let $F_n(\beta)$ denote a random formula from
%$F(l,n,\beta/n^{l-1})$.
A different model of random CNF formulas is studied in \cite{chvatal1992mick}. There 
formulas are viewed as families of non necessary different clauses rather than sets.
They define a random formula with $m$ clauses of size $l$ over $n$ variables as
a sequence of independent 
random clauses $C_1,\dots, C_m$ where each $C_i$ is chosen uniformly at random among
the $2^l \binom{n}{l}$ ordinary clauses of size $l$ over $n$ variables. The 
following holds
\begin{theorem} 
Let $l\geq 2$ be a natural number, and let $c\in [0,\infty)$ 
be an arbitrary real number. 
Let $m:\N\rightarrow \N$ be a map such that
$m(n)=(c+o(1))n$. For each $n$ let $C_{n,1},\dots, C_{n,m(n)}$
be clauses chosen uniformly at random independently among the 
$2^l \binom{n}{l}$ ordinary clauses of size $l$ over the
variables $x_1,\dots, x_n$. For each $n$, let $UNSAT_n$
denote the event that there is no assignment of the variables
$x_1,\dots,x_n$ that satisfies all clauses $C_{n,1},\dots, C_{n,m(n)}$. 
Then there are two real constants $0<c_1<c_2$, independent from such that 
\[
\Ln \Pr\big( UNSAT_n \big)=0
\]
if $c< c_1$, and 
\[
\Ln \Pr\big( UNSAT_n \big)=1
\]
if $c> c_2$. 
\end{theorem}
The existence of $c_1$ is proven in theorem 1 of \cite{chvatal1992mick}. 
The existence of the other constant $c_2$ follows from a simple application of
the first order method that also appears in \cite{chvatal1992mick}, as well as 
\cite{franco1983probabilistic}, \cite{chvatal1988many}, \cite{simon1986etude}
and possibly others. We want to show that this ``phase transition" also happens in
our model $F(l,n,p)$ when $p\sim \beta/n^{l-1}$. We start by showing the following
\begin{corollary}
	Let $l\geq 2$ be a natural number. Let $c\in [0,\infty)$ be an arbitrary
	real number and let $m:\N\rightarrow \N$ satisfy $m(n)=(c+o(1))n$. 
	For each $n\in \N$ let $F_{n,m(n)}$ be a random formula chosen uniformly at
	random among all the sets of $m(n)$ ordinary clauses of size $l$ over
	the variables $x_1,\dots, x_n$. Then there
	are two real positive constants $0<c_1<c_2$ such that
	\[
	\Ln \Pr \big( F_{n,m(n)} \text{ is unsatisfiable } \big)=0
	\]
	if $c< c_1$, and
	\[
	\Ln \Pr \big( F_{n,m(n)} \text{ is unsatisfiable } \big)=1
	\]
	if $c> c_2$.
\end{corollary}
\begin{proof}
One can consider $F_{n,m(n)}$ to be the result of `selecting clauses $C_{n,1},\dots, C_{n,m(n)}$
uniformly at random independently among all possible clauses' as in the previous theorem, but only in the case that `no two clauses $C_{n,i}, C_{n,j}$ are equal'. In consequence 
\[
\Pr \big( F_{n,m(n)} \text{ is unsatisfiable } \big) 
=
\Pr \big(UNSAT_n \, \big| \, \text{ all the $C_{n,i}$'s are different }\,\big),
\] 
where the event $UNSAT_n$ is defined as in the previous theorem. An application of the first
order method yields that for $l>3$ a.a.s the number of unordered pairs $\{i,j\}$ such that
$C_{n,i}=C_{n,j}$ is zero. For the case of $l=2$ an application of the factorial moments
method proves that the number of such pairs $\{i,j\}$ converges in distribution to 
a Poisson variable of positive mean. In either case we have
\[
\Ln \Pr \big(\, \text{ all the $C_{n,i}$'s are different }\,\big)>0.
\]
In consequence the constants $c_1$ and $c_2$ from the previous theorem satisfy
the statement of this corollary. 
\end{proof}
\sep


Let $F_{n,m(n)}$ be as in last theorem. Notice that because the symmetry in 
the random model $F(l,n,p(n))$ one can consider $F_{n,m(n)}$ to be a random
sample of the space $F(l,n,p(n))$ conditioned to the event that the number of clauses
is $m(n)$. Using this observation we can prove the following result:
  
\begin{theorem} \label{thm:phasetransition}
	Let $l\geq 2$ be a natural number.
	For each $n\in \N$ let $F_n(\beta)$ be a random formula from
	$F(l,n,\beta/n^{l-1})$. Then there are real positive values 
	$\beta_1 < \beta_2$ such that for any $0<\beta<\beta$
	it holds
	\[
	\Ln \PR{F_n(\beta) \text{ is unsatisfiable }  }=0,
	\]
	and for any $\beta>\beta_2$
	it holds
	\[
	\Ln \PR{F_n(\beta) \text{ is unsatisfiable }  }=1.
	\]
\end{theorem}
\begin{proof}
	For each $n\in \N$ let $X_{n}(\beta)$ be the random variable
	that counts the clauses in $F_n(\beta)$. It is satisfied that
	$\mathrm{E}[X_n(\beta)]\sim \frac{\beta\cdot 2^l}{l!}n$. Let $c_1,
	c_2$ be as in last corollary. Define $\beta_1:= \frac{c_1\cdot l!}{2^l}$
	and $\beta_2:= \frac{c_2 \cdot l!}{2^l}$. Fix $\beta\in \R$ satisfying
	$0<\beta<\beta_1$. Let $\epsilon>0$ be a real number such that
	$ \frac{\beta\cdot 2^l}{l!} +\epsilon< c_1$. For each $n\in \N$
	set $\delta_1(n):=\floor*{ \left(\frac{\beta\cdot 2^l}{l!}-\epsilon\right)n}$
	and $\delta_2(m):=\floor*{ \left(\frac{\beta\cdot 2^l}{l!}+\epsilon\right)n}$.
	Because of the Central Limit
	Theorem it holds
	\begin{equation}\label{eqn:aux6}
	\Ln \Pr\left( \delta_1(n) \leq X_n(\beta)  
	\leq \delta_2(n)    \right) = 1.
	\end{equation}
	Denote by $dp$ the probability density function of the variable $X_n(\beta)$.
	That is $dp(m)=\Pr(X_n(\beta)=m)$. Then, because of the previous equation it holds
	\[
	\Pr\left( F_n(\beta) \text{ is unsatisfiable }\right)\sim
	\int_{\delta_1(n)}^{\delta_2(n)}
	dp(m) \cdot \Pr \left(
	F_n(\beta) \text{ is unsatisfiable } \Big|
	X_n(\beta)=m	
	\right).
	\]
	Notice that the property of being unsatisfiable is monotonous. That is,
	for any two natural numbers $m_1<m_2$ it holds
	\[
	 \Pr \left(
	F_n(\beta) \text{ is unsatisfiable } \Big|
	X_n(\beta)=m_2	
	\right) <  \Pr \left(
	F_n(\beta) \text{ is unsatisfiable } \Big|
	X_n(\beta)=m_2	
	\right).
	\]
	In consequence,
	\begin{align*}
	&\int_{\delta_1(n)}^{\delta_2(n)}
	dp(m) \cdot \Pr \left(
	F_n(\beta) \text{ is unsatisfiable } \Big|
	X_n(\beta)=m	
	\right)\leq \\ & 
	\Pr \left(
	F_n(\beta) \text{ is unsatisfiable } \Big|
	X_n(\beta)=\delta_2(n) \right) \cdot 
	\Pr\big( \delta_1(n) \leq X_n(\beta)  
	\leq \delta_2(n)  \big).   
	\end{align*}
	And because of \cref{eqn:aux6}, 
	\begin{align*}
	&\Pr \left(
	F_n(\beta) \text{ is unsatisfiable } \Big|
	X_n(\beta)=\delta_2(n) \right) \cdot 
	\Pr\left( \delta_1(n) \leq X_n(\beta)  
	\leq \delta_2(n)  \right)\sim\\&
	\Pr \left(
	F_n(\beta) \text{ is unsatisfiable } \Big|
	X_n(\beta)=\delta_2(n) \right).
	\end{align*}
	Finally, as $\delta_2(n)< c_2n$, because of the previous
	corollary
	\[
	\Ln \Pr \left(
	F_n(\beta) \text{ is unsatisfiable } \Big|
	X_n(\beta)=\delta_2(n) \right)= 0.
	\]  
	Thus, joining everything, we have proven that for any 
	$\beta < \beta_1$, it holds that
	$F_n(\beta)$ a.a.s is satisfiable, as we wanted.
	Showing that for any $\beta > \beta_2$, 
	a.a.s $F_n(\beta)$ is unsatisfiable is analogous. We
	fix $\epsilon>0$ such that $\frac{\beta\cdot 2^l}{l!}-\epsilon>c_2$.
	We define $\delta_1(n)$ and $\delta_2(n)$ as before. Then similarly to before 
	using the Central Limit Theorem and the fact that
	the property of being unsatisfiable is monotonous one can prove the bound
	\[
	\Ln \Pr \left(
	F_n(\beta) \text{ is unsatisfiable } \right) \geq
	\Ln \Pr \left(
	F_n(\beta) \text{ is unsatisfiable } \Big|
	X_n(\beta)=\delta_1(n) \right).
	\]
	And using the previous corollary we obtain that last limit equals one, 
	and the result follows. 
	
\end{proof}
\sep

A direct consequence of last theorem, due to Albert Atserias, is the following

\begin{theorem} \label{thm:satapplication}
	Let $l>1$ be a natural number.
	For each $n\in \N$ let $F_n(\beta)$ be a random formula from
	$F(l,n,\beta/n^{l-1})$. Let $\Phi\in FO[\sigma]$ be a first order
	sentence that implies unsatisfiability.  Then for all $\beta\in [0,\infty)$
	it holds
	\[
	\Ln \PR{F_n(\beta)\models \Phi  }=0.
	\]
\end{theorem}
\begin{proof}
	Let $\beta_1$ and $\beta_2$ be as in \cref{thm:phasetransition}. 
	As $\Phi$ implies unsatisfiability it holds
	$\PR{F_n(\beta)\models \Phi  }\leq  
	\PR{F_n(\beta) \text{ is unsatisfiable }  }$. Thus, using \cref{thm:phasetransition}
	we get that for all $\beta\in [0,\beta_1]$
	\[
	\Ln \PR{F_n(\beta)\models \Phi  }=0.
	\]
	Because \cref{thm:mainsat} last limit varies analytically with $\beta$, so
	if it vanishes in a proper interval $[0,\beta_1]$ then it has to vanish
	in the whole $[0,\infty)$ by the principle of analytic continuation, and the result 
	holds. 
\end{proof}


\section*{Conclusions}


\section*{Acknowledgments}
This work was supported by the European Research Council (ERC) under the European Union's Horizon 2020 research and innovation programme (grant agreement ERC-2014-CoG 648276 AUTAR).\par
\todo{Albert and Marc Noy}
%I wish to thank Albert Atserias and Marc Noy for their help. Albert Atserias 
%presented to me the problem of generalizing the
%convergence law found by Lynch in \cite{lynch1992probabilities} with the application
%to RANDOM SAT given in \cref{thm:satapplication} in mind. 

\pagebreak
\bibliography{biblio}
\bibliographystyle{unsrt}	
\end{document}