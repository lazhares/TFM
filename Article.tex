\documentclass[11pt,notitlepage,a4paper]{article}

\usepackage[left=3cm,right=3cm,top=3cm,bottom=3cm]{geometry}
\usepackage{graphicx}
\usepackage{amssymb,mathtools, amsmath, amsfonts, amsthm}
%\usepackage{color}
\usepackage{float}
\usepackage{hyperref}
\usepackage{enumerate}
\usepackage{enumitem}
\usepackage{chngcntr}
\usepackage{cleveref}
\usepackage{pdfpages}

\counterwithout{equation}{section}

\newlength{\margen}
\setlength{\margen}{\paperwidth}
\addtolength{\margen}{-\textwidth}
\addtolength{\skip\footins}{0.7 cm}
\setlength{\margen}{0.5\margen}
\addtolength{\margen}{-1in}
\setlength{\oddsidemargin}{\margen}
\setlength{\evensidemargin}{\margen}
\setlength{\abovedisplayskip}{3pt}
\setlength{\belowdisplayskip}{3pt}
%%%% Small setup %%%%
\hypersetup{
	colorlinks=false,
	pdfborder={1 1 0.0005},
}
\setlength{\parskip}{0.2cm}
%%%%%%%%%%%%%%%
\usepackage{tikz-cd}
\usetikzlibrary{cd}
\usepackage[english]{babel}
\usepackage{todonotes}
\usepackage{cleveref}
\usepackage{caption}
\usepackage{subcaption}
\usepackage{bbding}
\usepackage{tcolorbox}
\usepackage{natbib}


\newtheorem{proposition}{Proposition}[section]
\newtheorem{fact}{Fact}[section]
\newtheorem{theorem}{Theorem}[section]
\newtheorem{lemma}{Lemma}[section]
\newtheorem{corollary}{Corollary}[section]
\theoremstyle{definition}
\newtheorem{definition}{Definition}[section]
\newtheorem{propdef}{Proposition / Definition}[section]
\newtheorem{remark}{Remark}[section]


\newcommand{\cc}{\mathfrak{c}}
\newcommand{\Z}{\mathbb{Z}}
\newcommand{\CC}{\mathbb{C}}
\newcommand{\Q}{\mathbb{Q}}
\newcommand{\R}{\mathbb{R}}
\newcommand{\N}{\mathbb{N}}
\newcommand{\Hc}{\mathcal{H}}
\newcommand{\Lan}{\mathcal{L}}
\newcommand{\Ln}{\lim\limits_{n\to \infty}}
\newcommand{\clist}{\mathfrak{c}_{1}, \cdots, \mathfrak{c}_m}
\newcommand{\morph}[1]{\stackrel{#1}{\simeq}}
\newcommand{\vlst}[2]{#1_1,\dots, #1_{#2}}
\newcommand{\gnp}{G(n,\beta_1/n^{a_1-1}, \dots,\beta_l/n^{a_l-1})}

\begin{document}
\begin{abstract}
	We consider a finite relational vocabulary $\sigma$ with
    a first order theory $T$ written in $\sigma$ 
    composed of symmetry and 
    anti-reflexivity axioms. We define a binomial random model of finite 
    $\sigma$-structures that satisfy $T$ and show that first order properties have well defined asymptotic probabilities in the sparse 
    case. It is also shown that those limit probabilities are well-behaved with
    respect to some parameters that represent edge densities. 
    An application of these results to the problem of random Boolean 
    satisfability is presented afterwards. 
    We show that there is no first order property of $k$-CNF formulas
    that implies unsatisfability and holds for almost all typical 
    unsatisfable formulas when the number of clauses is linear. 
   
\end{abstract}
\clearpage

\section*{Introduction}
.
.
.
.



\setcounter{section}{-1}

\section{Preliminaries}



A (relational) vocabulary $\sigma$ is a collection of
relation symbols $(R_1,\dots, R_m,\dots)$
where each relation symbol has associated a natural number called its arity.
.\par
 
A ($\sigma$-)structure $\mathfrak{A}$ is composed of (1)
a set $A$, called the universe of $\mathfrak{A}$, and sets of tuples 
$R_i^{\mathfrak{A}}\subseteq A^{a_i}$ for all relation symbols 
$R_i$ in $\sigma$, where $a_i$ is the arity of $R_i$.  
A structure is called finite if its universe is a finite set. 

The first order language with signature $\sigma$ deals with strings
of symbols taken from the alphabet consisting of
 variable symbols $x_1,\dots,x_m,\dots$, the symbols in $\sigma$,
the logical connectives $\neg, \wedge, \vee$,
the universal $\forall$ and existential $\exists$ quantifiers,
the equality symbol, and the parentheses $),($.
A first order formula is a string obtained after applying the
following set of rules a finite number of times: 
\begin{enumerate}[label=(\Roman*),itemsep=0pt, topsep=0pt]
	\item If $x_1,x_2$ are variables then $x_1=x_2$ is a formula.
	\item If $R_i$ is a relation symbol in $\sigma$ with arity $a_i$,
	and $t_1,\dots, t_{a_i}$ are terms then 
	$R_i(t_1,\dots,t_{a_i})$ is a formula.
	\item If $\varphi$ and $\psi$ are formulas, then both
	$(\varphi \wedge \psi)$ and $(\varphi \vee \psi)$ are formulas.
	\item If $\varphi$ is a formula then $\neq \varphi$ is also 
	a formula.
	\item If $\varphi$ is a formula and $x$ is a variable then both
	$\forall x \varphi$ and $\exists x\varphi$ are formulas. 
\end{enumerate} 

Let $\varphi$ be a first order formula. An occurrence of a variable 
in $\varphi$ is called bounded if it is within the scope of
a quantifier binding it, and is called free otherwise. We
will assume that the occurrences of any given variable in a first
order sentence are either all free or all bounded. 
We will use the notation $\varphi(x_1,\dots, x_n)$ to indicate
that $x_1, \dots, x_n$ are distinct and they are the free variables
(i.e. the variables whose occurrences are all free) 
in $\varphi$. A formula with no free variables is called a sentence,
and a formula whose variables are all free is called open.

Let $\mathfrak{A}$ be a $\sigma$-structure and $\varphi(x_1,\dots,x_n)$ 
be a first order sentence. Given a map $\alpha$ from the set
of free variables $\{x_1,\dots,x_n\}$ to the universe $A$ 
of $\mathfrak{A}$ we define the relation 
$\mathfrak{A} \models \varphi[\alpha]$, "$\alpha$ satisfies $\phi$ in $\mathfrak{A}$",
in the following way:
\begin{enumerate}[label=(\Roman*),itemsep=0pt, topsep=0pt]
	\item 
	\item If $R_i$ is a relation symbol in $\sigma$ with arity $a_i$,
	and $t_1,\dots, t_{a_i}$ are terms then 
	$R_i(t_1,\dots,t_{a_i})$ is a formula.
	\item If $\varphi$ and $\psi$ are formulas, then both
	$(\varphi \wedge \psi)$ and $(\varphi \vee \psi)$ are formulas.
	\item If $\varphi$ is a formula then $\neq \varphi$ is also 
	a formula.
	\item If $\varphi$ is a formula and $x$ is a variable then both
	$\forall x \varphi$ and $\exists x\varphi$ are formulas. 
\end{enumerate} 




 
%
%\section{Random relational structures}
%
%Given a natural number $n$, we will use the notation
%$[n]:= \{1,\dots, n\}$. We will denote by $S_n$
%the symmetric group on $[n]$, and by $\Delta_n$ the 
%diagonal set $\{(a,a)\in [n]^2 \}$. \par
%Given a set $X$, then $S_n$ acts on $X^n$ in an evident way. That is, 
%given $g\in S_n$ and $(x_1, \dots, x_n)$ one can define
%\[ g \cdot (x_1,\dots,x_n) =(y_1,\dots,y_n), \]
%where $y_{g(i)}=x_i$ for all $1\leq i\leq n$.\par
%Given $\Phi$ a subgroup of $S_n$ we will denote by $X^n/\Phi$
%the orbit set associated to the action of $\Phi$ over $X^n$.\par
%We will use the notation $[x_1,\dots,x_n]$ to refer to the 
%equivalence class of the $n$-tuple $(x_1,\dots, x_n)$ in 
%any sort of quotient $X^n/\Phi$. That is, while the notation
%$(x_1,\dots, x_n)$ will be reserved to ordered $n$-tuples, 
%$[x_1,\dots,x_n]$ will denote an ordered $n$-tuple modulo the
%action of some arbitrary group of permutations. Which group is this 
%will depend solely on the ambient set where $[x_1,\dots,x_n]$ is
%considered.
%\par
%
%
%\begin{definition}
%	Let $n,a \in \N$, let $\Phi$ be a subgroup of $S_a$, and let
%	$A$ be a subset 
%	\[ A\subseteq [a]^2 \setminus \delta.\]
%	The total edge set $\mathcal{H}_{(a,\Phi, A)}(n)$ of size $a$, 
%	symmetry group $\Phi$ and
%	restrictions $R$ on $n$ elements is the set:
%	\[  \mathcal{H}_{(a,\Phi, R)}(n)= ([n]^a/\Phi) \, \,
%	\setminus \{\,  [x_1, \dots,x_a] \in [n]^a/\Phi  \, \, 
%	| \, \, x_i=x_j \, \text{for some } (i,j)\in R \} \]
%\end{definition}
%
%\begin{definition}
%	An (hyper)-graph $([n], H_1,\dots, H_c)$ with edge colors 
%	$1,\dots, c$, sizes $a_1,\dots,a_c$, 
%	symmetry groups $\Phi_1,\dots,\Phi_c$ and 
%	restrictions $A_1,\dots,A_c$ consists of 
%	\begin{itemize}
%		\item The set $[n]$ for some natural number $n$.
%		\item For $i=1,\dots,c$, a colored edge set $H_i\subseteq \mathcal{H}_{(a_i,\Phi_i,A_i)}(n)$ whose elements 
%		have color $i$.
%	\end{itemize}
%\end{definition}
%
%\begin{definition} 
%	Let $p=(p_1,\dots, p_c)$, where all $p_i$'s are real numbers
%	between $0$ and $1$. 
%	The random model $HG(n,p)$ with edge colors $1,\dots,c$,
%	sizes $a_1,\dots,a_c$, 
%	symmetry groups $\Phi_1,dots,\Phi_c$ 
%	and restrictions $A_1,\dots,A_c$, is the one that
%	assigns to each graph $G=([n], H_1,\dots, H_c)$
%	probability
%	\[ \mathrm{Pr}(G)=\prod_{i=1}^{c} p_i^{|H_i|}(1-p_i)^{|\mathcal{H}_{(a_i,\Phi_i,A_i)}(n)|-|H_i|}.	
%	\]
%	Equivalently, this is the probability space obtained by 
%	assigning to each colored edge 
%	$e\in \mathcal{H}_{(a_i,\Phi_i,A_i)}(n)$ probability $p_i$ independently. 
%\end{definition}
%
%For the rest of the work we will consider 
%\begin{itemize}
%	\item the total number of colors $c$,	
%	\item the sizes $a_1,\dots,a_c$,
%	\item the symmetry group $\Phi_1,\dots, \Phi_c$ and,
%	\item the restrictions $A_1,\dots,A_l$
%\end{itemize}
%fixed. When we say ``graph'' from now on what
%we will mean is ``hiper-graph with edge colors 
%$1,\dots, c$, sizes $a_1,\dots,a_c$, 
%symmetry groups $\Phi_1,\dots,\Phi_c$ and 
%restrictions $R_1,\dots,R_c$''. \par
%
%Given a graph $G=([n], H_1,\dots, H_c)$ we will denote by
%$H_i(G)$ the edge set $H_i$, and by $V[G]$ the vertex set
%$[n]$. Also, we will write $H(G)$ to denote the 
%disjoint union of colored sets $\cup_{i=1}^c H_i$.
%This way, an edge $e\in H(G)$ with color $i$ is an element
%$[x_1,\dots,x_{a_i}]\in H_i(G)$, and the $x_i$'s are vertices
%belonging to $V(G)$. \par
%Given a set of vertices, $X\subseteq V(G)$, we will denote
%the by $G[X]$ the induced sub-graph on $X$. \par
%As usual, we will sometimes treat edges as sets of vertices
%rather than ``tuples modulo the action of some permutation group''. 
%This way, expressions like $e_1\cup e_2$ for $e_1, e_2\in H(G)$
%will make sense and mean 
%``the set of vertices that occupy some place in $e_1$ 
%and in $e_2$".\par
%Some other times we will treat edges $e\in H(G)$
%as sub-graphs of $G$ in the evident way. That is, the subgraph
%denoted by $e$ is the one whose vertex set is $e$-i.e., the vertices in $e$-
%and whose only edge is $e$. 
%This way, when we have some edges $e_1,\dots, e_l\in H(G)$ 
%it will make sense to talk
%about the subgraph $\cup_{i=1}^l e_i$, which is the graph
%whose vertex set is the set of vertices belonging to the $e_i$'s, 
%and whose edges are exactly the $e_i$'s. In spite of these
%abuses of notation the ``type'' of any ``term'' involving edges 
%should be derivable from the context. \par
%Another usual abuse of notation we will make is to sometimes
%treat graphs as their underlying vertex sets. Hence,
%expressions defined for sets of vertices will also be defined 
%for graphs. 
%	
	
	
\end{document}