\documentclass[11pt,notitlepage]{report}

\usepackage[left=4cm,right=4cm,top=3cm,bottom=3cm]{geometry}
\usepackage{graphicx}
\usepackage{amssymb,mathtools, amsmath, amsfonts, amsthm}
%\usepackage{color}
\usepackage{float}
\usepackage{hyperref}
\usepackage{enumerate}
\usepackage{enumitem}

\newlength{\margen}
\setlength{\margen}{\paperwidth}
\addtolength{\margen}{-\textwidth}
\addtolength{\skip\footins}{0.7 cm}
\setlength{\margen}{0.5\margen}
\addtolength{\margen}{-1in}
\setlength{\oddsidemargin}{\margen}
\setlength{\evensidemargin}{\margen}
\setlength{\abovedisplayskip}{3pt}
\setlength{\belowdisplayskip}{3pt}
%%%% Small setup %%%%
\hypersetup{
	colorlinks=false,
	pdfborder={1 1 0.0005},
}
\setlength{\parskip}{0.2cm}
%%%%%%%%%%%%%%%
\usepackage{tikz-cd}
\usetikzlibrary{cd}
\usepackage[english]{babel}
\usepackage{todonotes}
\usepackage{cleveref}
\usepackage{caption}
\usepackage{subcaption}
\usepackage{bbding}
\usepackage{tcolorbox}
\usepackage{natbib}

\DeclareMathOperator{\incl}{incl}

\newtheorem{proposition}{Proposition}[chapter]
\newtheorem{fact}{Fact}[chapter]
\newtheorem{theorem}{Theorem}[chapter]
\newtheorem{lemma}{Lemma}[chapter]
\newtheorem{corollary}{Corollary}[chapter]
\theoremstyle{definition}
\newtheorem{definition}{Definition}[chapter]
\newtheorem{propdef}{Proposition / Definition}[chapter]

\newtheorem{inneraxiom}{Axiom}
\newenvironment{axiom}[1]
{\renewcommand\theinneraxiom{#1}\inneraxiom}
{\endinneraxiom}

\newcommand{\Z}{\mathbb{Z}}
\newcommand{\CC}{\mathbb{C}}
\newcommand{\Q}{\mathbb{Q}}
\newcommand{\R}{\mathbb{R}}
\newcommand{\N}{\mathbb{N}}
\newcommand{\Hc}{\mathcal{H}}
\newcommand{\Lan}{\mathcal{L}}
\newcommand{\Ln}{\lim\limits_{n\to \infty}}
\newcommand{\clist}{c_{i_1}, \cdots, c_{i_m}}
\newcommand{\morph}[1]{\stackrel{#1}{\simeq}}
\newcommand{\vlst}[2]{#1_1,\dots, #1_{#2}}
\newcommand{\gnp}{G(n,\beta_1/n^{a_1-1}, \dots,\beta_l/n^{a_l-1})}



\begin{document}
\tableofcontents

\chapter*{Introduction}

\chapter*{Notation}



\chapter{Preliminaries}


\section{Models of Random Graphs}

\section{First Order Logic}

\section{Ehrenfeucht Fraisse Games and the Logic of Random Graphs}



\chapter{Probabilities of Sentences about Very Sparse Random Graphs}


In this chapter we will review the results obtained
 in the paper with the same name by James F. Lynch \cite{lynch1992probabilities}.
In there, limit probabilities of sentences in the first order language of graphs $\mathcal{L}$
are discussed for the binomial model $G(n,p)$ in the cases $p=\beta/n$ and
$p=\beta n^{-\alpha}$ with $\alpha=(l+1)/l$ . \par

More precisely, it is proven that in those cases the 
probability of every sentence converges and it is shown
that for any of those sentences, its limit probability 
is among the values taken by some analytic formulas
with parameter $\beta$. \par

We are interested in the case $p=\frac{\beta}{n}$, which is 
the one discussed more extensively in \cite{lynch1992probabilities}. 
According to the author, the relevant theorems for 
the other case can be proven analogously. From now on we
will only refer as random graphs to the ones in $G(n,\beta/n)$\par

From now on we will denote by $Poi_\lambda$ the probability
function of the Poisson distribution with mean $\lambda$.
That is, the one given by $Poi_\lambda(n)=e^{-\lambda}\lambda^n/n!$ 
for any $n\in \N$.
Also, we define $Poi_\lambda(\leq n)$ and $Poi_\lambda(>n)$ as 
$\sum_{i=0}^n Poi_\lambda(n)$ and $1-Poi_\lambda(\leq n)$
respectively. Notice that for a fixed $n$, both $Poi_\lambda(\leq n)$ 
and $Poi_\lambda(>n)$ can be considered real functions of parameter $\lambda$. \par

We define the following sets of functions. Let $\Lambda$ be the smallest 
set of expressions with parameter $\beta$ such that:

\begin{itemize}[noitemsep, topsep=0pt]
	\item $1\in \Lambda$,
	\item For any $\lambda \in \Lambda$ and $i\in \N$, 
	both $Poi_{\beta\lambda}(n)$ and $Poi_{\beta\lambda}(> n)$ are in
	$\Lambda$.
	\item For any $\lambda_1,\lambda_2 \in \Lambda$,
	$\lambda_1 \lambda_2$ belongs to $\Lambda$ as well.
\end{itemize}

And let $\Theta$ be the smallest set of functions with parameter $\beta$
such that:
\begin{itemize}[noitemsep, topsep=0pt]
	\item For any $\lambda \in \Lambda$ 
	and $n,a,i\in \N$, with $i\geq 3$, 
	both $Poi_{\beta^i\lambda/a}(\leq n)$ and $Poi_{\beta^i\lambda/a}(> n)$
	are in $\Theta$.
\end{itemize} 

The main result is the following:

\begin{theorem}[Lynch, 1992] \label{thrm:main}
	Let $\phi$ be a sentence in the first order theory of graphs. Then the limit
	$\lim\limits_{n\to \infty} P(G(n,\beta/n) \models \phi )$ exists for all positive real numbers
	$\beta$, and it is a finite sum of expressions in $\Theta$.
\end{theorem}

We show now an outline of the proof. \par 
We show that for any quantifier rank $k$ there are some classes of graphs 
$C^k_1,\dots, C^k_{n_k}$ such that
\begin{itemize}[noitemsep, topsep=0pt]
	\item[(1)] a.a.s the rank $k$ type of any two graphs in the same class coincide, 
	\item[(2)] a.a.s. any random graph belongs to some of them, and
	\item[(3)] the limit probability of random graph belonging to any of them is an expression in $\Theta$. 
\end{itemize}

After this is archived the theorem follows easily. Indeed, let $\phi$ be a sentence 
in the first order language $\Lan$ of graphs whose quantifier rank is $k$, and denote by
$G$ a random graph in $G(n,\beta/n)$. We define the events $E_1,\dots,E_{n_k}$ as
\[E_i:= (G \models \phi) \wedge (G \in C_i),\]
and the  event $F$ as
\[F:= (G \models \phi) \bigwedge_{i=1}^{n_k} (G \notin C_i).\]
Then, for any $n\in \N$
\begin{equation} \label{eqn:sumevents}
P(G\models \phi) = \sum_{i=1}^{n_k} P(E_i)  + P(F),
\end{equation}
as the events $E_i$ together with $F$ form a partition of all the cases where $G$ satisfies $\phi$. \par

Fix and index $i\in \{1, \dots, n_k\}$. From the property (1) of the classes $C_1,\dots, C_{n_k}$ 
it follows that if $G$ and $H$ are random graphs, then
\[\Ln P((G\models \phi) \wedge \neg(H\models \phi) \, | \, G\in C_i \wedge H\in C_i \, ) = 0.\]
This is because $G$ and $H$ share a.a.s the same rank $k$ type if they both belong to $C_i$. 
In consequence the limit
\[ \Ln P(G\models \phi \, | \, G\in C_i )\] 
is either zero or one, and 
\begin{equation}\label{eqn:property1}
\Ln P(E_i)= \Ln P(G\in C_i)\cdot P(G\models \phi \, | \, G\in C_i)= \text{ either } 0 \text{ or } \Ln P(G\in C_i).
\end{equation} 
\par 
Also, as a consequence of property $(2)$ we obtain
\[\Ln P(\bigwedge_{i=1}^{n_k} G \notin C_i)=0, \] 
so
\begin{equation}\label{eqn:property2}
\Ln P(F)= \Ln P(\bigwedge_{i=1}^{n_k} G \notin C_i)\cdot P(G\models \phi \, | \, \bigwedge_{i=1}^{n_k} G \notin C_i)=0.  
\end{equation}
\par
Taking limits in equation \ref{eqn:sumevents} and using equations \ref{eqn:property1} and \ref{eqn:property2} 
we get
\[ \Ln P(G\models \phi) = \sum\limits_{C_i\in \mathcal{C}} \Ln P(G \in C_i) ,\]
where $\mathcal{C}$ is a (possibly empty) subset of $\{C_1,\dots, C_{n_k}\}$.
Finally, because of property (3) for each $i$ the limit $\Ln P(G \in C_i)$ is 
an expression in $\Theta$. Thus $\Ln P(G\models \phi)$ is a finite sum of expressions in $\Theta$
and the theorem follows. \par

The objective of next sections will be to define the classes $C_1,\dots, C_{n_k}$ and to show
that they satisfy properties (1), (2) and (3). Later we will prove a stronger result, so we will
allow ourselves to just sketch some of the proofs during this chapter. 

\section{Agreability Classes}


It is known that $n^{-v/e}$ is the t
hreshold probability for the appearance 
of a balanced graph of density $v/e$. 
In our case $v/e=1$, so in consequence any connected graph $H$
with $e(H)< v(H)$ a.a.s will not appear 
as a subgraphs of $G(n,\beta/n)$. It can be easily 
shown that such graphs $H$ are exactly 
the ones containing more than one cycle. \par

If $H$ is a connected graph with $v=e$, then $H$ is an uni-cyclic graph.
In this case, the number $X_H$ of copies of $H$ 
in $G(n,\beta/n)$ will asymptotically have non-zero bounded expectancy $m$. 
It does not take much work to prove, using Brun's sieve, 
that $X_H$ converges in distribution to a Poisson 
random variable with mean $m$ as $n$ goes to infinity.  \par

Finally, if $H$ is a connected graph 
with $v>e$ then it must be a tree. Here 
the expected number of copies of $H$ 
in $G(n,\beta/n)$ diverges asymptotically. 
Informally, trees of any kind will occur arbitrarily often. \par

This all means, in a sense, that a.a.s the only 
difference between large graphs in $G(n,\beta/n)$
lies in their uni-cyclic subgraphs. More precisely,
because of the ``locality" of first order logic of quantifier 
rank $k$ we will only be interested in the 
``small`` neighborhoods of the ``short" cycles.  
Thus, our goal will be to classify uni-cyclic graphs
in a way that respects equivalence under first 
order logic of quantifier rank $k$. \par
~\par
To make our classification suitable for proofs 
involving E.F. games we need to work graphs to which we ``attach" labels. 
We define the set of symbols $Const=\{c_i\}_{i\in \N}$ 
as the set of constants. Also, we will denote by
$Const_n$ the set $\{c_1,\dots, c_n\}$.
 
\begin{definition} 
	A \textbf{co-labeling} of a graph $G=(V,E)$ is a map 
	$\sigma: D\rightarrow V$, where $D\subset C$ is a 
	finite set of constant symbols. Given $c_i\in D$, 
	we will say that the vertex $\sigma(c_i)$ is labeled $c_i$.
	Equivalently, we can denote a labeling $\sigma$ as a tuple
	$(c_{i_1}[x_1],\dots, c_{i_m}[x_m])$ where each $c_{i_j}$
	 is a constant symbol, and $x_j$ is the vertex
	in $V$ labeled $c_{i_j}$.
\end{definition}

\begin{definition} 
	A \textbf{graph with constants}\footnote{
		Compare with \cite{lynch1992probabilities}, where they are called ``rooted graphs". 
		}
	$G(c_{i_1}[x_1],\dots, c_{i_m}[x_m])$ 
	is a graph $G$ together with a co-labeling 
	$(c_{i_1}[x_1],\dots, c_{i_m}[x_m])$. 
\end{definition}

To keep our notation compact we will often drop 
he $x_i$'s and say $G(c_{i_1},\dots, c_{i_m})$. \par

\begin{definition}
Let $G$ be a graph with constants. A subgraph $H$ of $G$ is
a graph with constants such that $V(H)\subseteq V(G)$, $E(H)\subseteq E(G)$ and 
all vertices in $V(H)$ have the same labels in $H$ and $G$. 
\end{definition}


An important abuse of notation we are going to make
will be to identify the constants $c_i$ with their labeled
vertices $\sigma(c_i)$. This way things like $c_i\sim c_j$ 
will make sense. In this context, notice that the expression
$c_i=c_j$ is ambiguous because the vertices labeled $c_i$ and $c_j$ 
may be the same for some $i\neq j$, but 
the constant symbols $c_i$ and $c_j$ will be equal only if $i=j$. 
We will make sure to leave no room for ambiguity in this situations. \par


\begin{propdef}
	Let $G=(V,E,\clist)$ be a connected graph with constants. Then it has a unique minimal
	connected subgraph $H$ containing all its constants and cycles.
	We will call the \textbf{center} of $G$ to such subgraph and denote it by $Center(G)$.
	If $\bar{G}$ is an arbitrary graph with constants, then its center $Center{\bar{G}}$ will
	be the union of the centers of its connected components. 
\end{propdef}
\begin{proof}
	TO DO
\end{proof}

For an arbitrary graph with constants we define the metric $d(\cdot,\cdot)$ on $V(G)$ 
as the one such that $d(x,y)$ is the minimum length of a path connecting
$x$ and $y$ in $G$ or $\infty$ if such path does not exist. 
For any vertex $x\in V(G)$ and $r\in \N$ we define the co-labeled
subgraph $N(x;r)$ as the ball of radius $r$ centered at $v$. 
That is, the induced subgraph with vertex set
\[ V(N(x;r))= \{\, y\in V \, | \, d(x,y)\leq  r \,	\}.\] 
In a similar vein, given $X\subseteq V(G)$ we define its neighborhood of radius $r$ as
the induced co-labeled subgraph $N(X;r)$ whose vertex set is
\[ V(N(X;r))= \{\, y\in V \, | \, \forall x\in X: \, \,  d(x,y)\leq  r \,	\}.\]  
\par

Let $G=(V,E)$, and $V^\prime\subseteq V$. Another important abuse 
of notation we will make is writing $H=(V^\prime, E)$ 
for a subgraph $H$ to mean that the edge set of $E(H)$ is the one
induced by $E(G)$ on $V^\prime$.

\begin{definition} 
	A \textbf{rooted tree} $T=(V,E,x)$ is a tree 
	$(V,E)$ with distinguished vertex $x\in V$ with we will call 
	\textbf{root} of the tree.
\end{definition}

\begin{propdef}
	Let $G=(V,E,\clist)$ be a connected graph and $x\in V$. 
	Then define $Tree(x,G)$	as the rooted tree
	\[Tree(x,G) = (V_x,E,x),\] where
	\[V_x= \{\, y\in V \, | \,\,d(Center(G),y) = d(Center(G),x) + d(x,y) \}.\]

\end{propdef}
\begin{proof}
	TO DO
\end{proof}



The radius $r(T)$ of a rooted tree $T=(V,E,x)$
is the maximum distance between its root $x$ and any other of
its vertices. The branches of $T$ are the rooted trees of the form
$Tree(y,T)$, where $y\sim x$. We will denote by $Br(T)$ the set of
branches of $T$. \par

We begin by defining an equivalence relation between rooted 
trees for each quantifier rank $k$.

\begin{definition} 
	Let $k\in \N$ with $k\geq 1$. The \textbf{k-morphism} equivalence relation
	$\morph{k}$ between
	graph with constantss is the one inductively defined as follows:
	\begin{itemize}
		\item If $T_1, T_2$ are rooted trees of radius $0$ -i.e., they
		consist only of their roots- they are $k$-morphic. 
		\item Let $T_1, T_2$ be rooted trees of radius $r$ whose 
		rots have the same label. Then $T_1 \morph{k} T_2$ if 
		for any $k$-morphism class $C$ of trees with
		radius less than $r$ and root either
		\begin{center}
			\vspace{-0.2cm}
			``$T_1$ and $T_2$ have the same number of branches of type $C$"
		\end{center}
		\vspace{-0.3cm}
		\[ |Br(T_1)\cap C| = |Br(T_2)\cap C|,\]
		or 
		\begin{center}
			\vspace{-0.2cm}
			``$T_1$ and $T_2$ have both more than $k$ branches of type $C$"
		\end{center}
	\vspace{-0.3cm}
		\[ |Br(T_i)\cap C| \geq k+1 \text{ for } i=1,2. \]
	\end{itemize} 
\end{definition}

It follows from the definition that $k$-morphic trees have the same radius. 
It is also easy to check that the $k$-morphism relation is indeed an equivalence one. 

\begin{proposition}
	For all $k,r\in N$ and with $k\geq 1$, the set of classes of $k$-morphic trees
	with radius lesser or equal than $r$ is finite.
\end{proposition}
\begin{proof}
TO DO
\end{proof}

We define now the $k$-morphism relation for arbitrary graph with constantss. 
%If $G^1=(V^1,E^1,\clist)$ and $G^2=(V^2,E^2,\clist)$ are graph with constantss with
%the same constants, to avoid confusion we will denote by $c_{i_j}^1$ and $c_{i_j}^2$
%the vertices labeled $c_{i_j}$ in $G^1$ and $G^2$ respectively. 

\begin{definition}
	Let $G^1=(V^1,E^1,c_{i_1}[x^1_1],\dots, c_{i_m}[x_m^1])$ and  $G^2=(V^2,E^2,c_{i_1}[x^1_2],$ 
	$\dots, c_{i_m}[x_m^2])$ be graph with constantss with the same constant symbols. 
	We will say that they are $k$-morphic (denoted by $G^1 \morph{k}G^2$) if there is
	a bijection $f: V(Center(G^1))\rightarrow V(Center(G^2))$ such that
	\begin{itemize}
		\item ``$f$ preserves edges"
		\[\forall x,y\in V(Center(G^1)): \quad  x\sim y \iff f(x)\sim f(y). \]
		\item ``$f$ preserves labels"
		\[\forall j\in \{1,\dots,m\}: \quad f(x^1_j) = x^2_j.\]
		\item ``$f$ preserves $k$-morphism classes of trees"
		\[\forall x\in V(Center(G^1)): \quad  Tree(x,G^1)\morph{k} Tree(f(x),G^2).\]
	\end{itemize}
	In this case we will say that $G^1 \morph{k} G^2$ via $f$. 	
\end{definition}

We are going to show that the rank $k$ type of a random graph a.a.s only 
depends on the neighborhoods of its small cycles. In consequence the following definition
is motivated:

\begin{definition} 
	Let $G=(V,E,\clist)$ be a graph with constants. Then its core of radius $r$, $Core(G,r)$
	is the co-labeled subgraph $N(X;r)$, where $X$ is the union of the (vertex sets of the)
	cycles in $G$ with size at most $2r+1$ and all of the labeled vertices in $G$. 
\end{definition}



\chapter{First Order Logic of Sparse Random General Hypergraphs}

\section{General Hypergraphs}

$\forall n\in N: \quad [n]:= \{1,\dots, n\}$\par
$S_n:$ symmetric group on $n$ elements. \par
Let $X$ be a set, then:
\begin{itemize}
	\item $\Delta\subset X^2:$ $\Delta$ is the diagonal subset. $\Delta: \{(a,a)\in X^2 \}$.
	\item $S_n$ acts on $X^n$ in the obvious way. 
	\item When the subgroup $\phi$ is understood or (is not relevant) $\langle x_1,\dots,x_n\rangle$
	denotes the equivalence class of $(\vlst{x}{n})$ in $X^n/\phi$.
\end{itemize}
\par


\begin{definition}
	Let $V$ be a finite set. Given a natural number $a\in \N$, a group $\phi\leq S_a$ and a subset
	$R\subseteq [a]^2 \setminus \delta$, the total (hiper-)edge set of size $a$, symmetry group $\phi$ and
	restrictions $R$ is the set:
	\[ \mathcal{H}_{V,a}^{\phi,R}= V^n/\phi \, \, \setminus \{\,  \langle 
	x_1, \dots,x_n\rangle
	 \in V^n/\phi  \, \, | \, \, x_i=x_j \, \text{for some } (i,j)\in R \} \]
	
\end{definition}

\begin{definition}
	An hypergraph $(V,H_1,\dots, H_l)$ with edge sizes $a_1,\dots,a_l$, 
	symmetry groups $\phi_1,dots,\phi_1$ and restrictions $R_1,\dots,R_l$ consists of
	\begin{itemize}
		\item A finite vertex set $V$.
		\item For $i=1,\dots,l$, an edge set $H_i\subseteq \mathcal{H}_{V,a_i}^{\phi_i,R_i}$.
	\end{itemize}
\end{definition}

\begin{definition} 
	Random model $G(n, p_1,\dots, p_l)$ with edge sizes $a_1,\dots,a_l$, 
	symmetry groups $\phi_1,dots,\phi_1$ and restrictions $R_1,\dots,R_l$:
	\begin{itemize}
		\item We obtain a hypergraph $(V,H_1,\dots, H_l)$ with $V=[n]$
		\item The probability of each edge $e\in \mathcal{H}_{V,a_i}^{\phi_i,R_i}$
		is $p_i$ independently. 
	\end{itemize}
\end{definition}

From now on we fix the edge sizes $a_1,\dots,a_l$, 
symmetry groups $\phi_1,dots,\phi_1$ and restrictions $R_1,\dots,R_l$. \par

In a graph $G(V,E_1,\dots, E_l)$, the type of an edge $e$ is the index
$i$ such that $e\in E_i$.  

The language is the first order relational 
language with relations $R_1,..., R_l$ of arities
$a_1,\dots, a_l$ respectively. The interpretation on 
$(V,H_1,\dots, H_l)$ is the evident one. \par

\begin{definition}
	The characteristic $ch(G)$ of an hypergraph $G=(V,E_1,\dots,E_l)$ is
	\[ \left( \sum_{i=1}^l |E_i|(a_i-1)\right) -	|V|  .\]
	An hypergraph is c-balanced if it contains no subgraph of
	greater characteristic than itself. 
\end{definition}

\begin{proposition}
	A connected graph cannot have characteristic lesser than $-1$.
\end{proposition}

\begin{definition}
	A unicycle is a connected c-balanced graph with characteristic $0$.
\end{definition}

\begin{definition} A cluster is a minimal c-balanced graph.
\end{definition}

Remark: Note that clusters have to be connected.
 
\begin{definition}
	A cycle is a minimal unicycle, or equivalently a characteristic $0$ cluster.  
\end{definition}

\begin{proposition}
	Any unicycle contains exactly one cycle. 
\end{proposition}

\begin{definition} 
	A (hiper)forest is a graph with no clusters. A (hiper)tree is a connected forest.
\end{definition}

\begin{definition} 
	A path between $x$ and $y$ is a connected graph containing both $x$ and $y$
	that is minimal among the ones with those properties. 
\end{definition}

\begin{proposition} There is a unique path between any two vertices of a tree. 
\end{proposition}


We define the constant set $Const:=\{c_i\}_{i\in \N}$.

\begin{definition} 
	As before, a (hiper)graph with constants $G(\clist)$ is a hypergraph 
	$G$ together with a co-labeling. 
\end{definition}

\begin{definition}
	The center of a connected graph with constants $G(\clist)$ is its minimal 
	connected subgraph containing all the constants and clusters. If $G(\clist)$
	is not connected then its center is the union of the centers of its connected
	components.   
\end{definition}

We define a distance over $G(\clist)$.
\[ d(x,y)= (\text{minimum size of a connected graph containing }x \text{ and } y) - 1 \]
If such graph does not exist we define $d(x,y)=\infty$.\par

Given $x\in V(G)$ and $X\subseteq V(G)$ we define $N(x;r)$ and $N(X;r)$ as
\[ N(x;r)= G[S], \quad \text{where } S=\{\, y\in V(G) \, | dist(x,y)\leq r \, \} \]
\[ N(X;r)= G[S], \quad \text{where } S=\{\, y\in V(G) \, | dist(X,y)\leq r \, \}. \]

\begin{definition}
	 A rooted (hiper)tree is a tree with a distinguished vertex called root.
	 The radius of the tree is the maximum distance between its root and any
	 of its vertices.  
\end{definition}

In a rooted tree $(T,x)$ all the (hiper)edges can be rooted in a canonical way. 
The root of an edge $e$ is the vertex $y\in e$ such that $dist(x,y)=dist(x,e)$.

\begin{definition}
	Let $G(\clist)$ be a graph with constants and $x\in V(G)$. Then
	$Tree(x,G)=(G[V_x],x)$, where
	\[ V_x= \{\, y\in V \, | \, dist(Center(G),y)= dist(Center(G),x)+ dist(x,y) \,	 \}. \]
\end{definition}

\begin{definition} 
	The $k$-morphism equivalence relation $\morph{k}$ between rooted trees,
	and the $\stackrel{k}{\sim}$ relation for rooted (hyper)edges are defined inductively as follows:
	\begin{itemize}
		\item If $T_1$ and $T_2$ are radius $0$ trees then they consist only of
		their roots and $T_1\morph{k} T_2$. 
		\item 
		If $e_1$ and $e_2$ are edges in rooted trees $T_1$,$T_2$ of radii at most $r$,
		then $e_1\stackrel{k}{\sim} e_2$ if $e_1$ and $e_2$ are edges of the same 
		type and $e_1=[x_1,\dots, x_m]$, $e_2=[y_1,\dots,y_m]$ in a way that
		\begin{itemize}
			\item $x_i$ is the root of $e_1$ if and only if $y_i$ is the root of $e_2$.
			\item If $x_j$ is not the root of $e_1$, then
			$Tree(x_j, T_1)\morph{k} Tree(y_j, T_2)$.
		\end{itemize}
		\item If $T_1$ and $T_2$ are trees of radii at most $r$ and roots $x$ and $y$,
		then $T_1\morph{k} T_2$ means that for any $\stackrel{k}{\sim}$ class
		of edges $C$ either
		\begin{itemize}
			\item the number of edges in $T_1$ of type $C$ containing $x$ is the same
			as the number of edges in $T_2$ of tipe $C$ containing $y$, or
			\item both $T_1$ and $T_2$ contain no less than $k+1$ edges of type $C$
			containing $x$ and $y$ respectively. 
		\end{itemize}
		 
	\end{itemize}
	
\end{definition}

\begin{definition} 
	Let $G_1(\clist)$ and $G_2(\clist)$ be graphs with the same constants. Then
	we say $G_1(\clist)\morph{k}G_2(\clist)$ if there is a bijection 
	$f:V(Center(G_1))\rightarrow V(Center(G_2))$ such that
	\begin{itemize}
		\item $f(c^1_i)=c^2_i$ for all constants, and
		\item $Tree(x,G_1)\morph{k} Tree(f(x),G_2)$ for all $x\in V(Center(G_1))$.
	\end{itemize}
\end{definition}

The diameter of a (hiper)graph is the maximum distance between any two of its vertices. 

\begin{definition} 
	Let $G(\clist)$ be a graph with constants. We define $Core(G,r)$ to be
	$N(X,r)$ where $X$ is the union of constants and clusters of diameter at most $r$. 
\end{definition}


\begin{definition} 
	A (hyper)graph $G$ is $r$-simple if $Core(G,r)$ is a union of unicycles. 
\end{definition}

\begin{definition}
	A (hyper)graph $G$ is $j,r$-rich for some $j,r\in \N$, if for any rooted tree $T$
	of radius at most $r$ there are $x_1,\dots,x_j\in V(G)$ such that
	\begin{itemize}
		\item The $N(x_l;r)$'s are disjoint.
		\item The $N(x_l;r)$'s do not intersect $Core(G,r)$
		\item $N(x_l;r)(x_l) \morph{j} T$.		
	\end{itemize} 
\end{definition}


\begin{theorem}
	If $G_1(c_1,\dots,c_m)\morph{k} G_2(c_1,\dots,c_m)$, then for any $x\in V(G_1)$ such 
	that $d(x,Center(G_1))\neq \infty$ there is a $y\in V(G_2)$ such that
	$G_1(c_1,\dots,c_m,c_{m+1}[x])\morph{k-1} G_2(c_1,\dots,c_m,c_{m+1}[y])$.
\end{theorem}

\begin{theorem}
	If $Core(G_1,r)\morph{k} Core(G_2,r)$ then for any $s<r$, $Core(G_1,s)\morph{k} Core(G_2,s)$.
\end{theorem}


\begin{definition} 
	Two graphs with constants $G_1$ and $G_2$ are $k$-agreeable if for each $k$-morphism 
	equivalence class $C$ they either have the same number of connected components of type
	$C$ or they both have no less than $k+1$ components of type $C$. 
\end{definition}


\begin{theorem} 
	Let $G_1, G_2$ be graphs with constants satisfying:
	\begin{itemize}
		\item They are both $(k, k, 3^k)$-rich.
		\item $Core(G_1,3^k)$ is $k$-agreeable with $Core(G_2,3^k)$.
	\end{itemize} 
	Then the $G_1$ and $G_2$ share the same rank $k$ type.
\end{theorem}

From now on fix $\beta_1,\dots,\beta_l$ positive real numbers. 

\begin{lemma}
	Let $H$ be a cluster with $ch(H)>0$, and let $X_n$ be the
	random variable that counts the number of times that $H$ appears
	as a subgraph of $G(n,p(n))$. Then
	\[\Ln Pr(X_n>0)=0. \]
\end{lemma}
\begin{proof}
	Let $v=|V(H)|$ and $e_i=|E_i(H)|$ for $i=1,\dots, l$. 
	The criticality condition on $H$ means 
	\[ \langle e, a-1 \rangle - v > 0 .\]
	Chose a ordering of the vertices in $H$. 
	For any ordered sequence of vertices $S=(x_1, \dots, x_v)$, 
	let $X_{n,S}$ be the indicator variable that equals $1$ if $H$
	is a subgraph of $G[S]$ (in a way that respects the ordering) and
	$0$ otherwise. Clearly $X_n$ is the sum of all the $X_{n,S}$'s, so
	\[ E(X_n)= \frac{n(n-1)\cdots (n-v+1)}{b} \prod_{i=1}^{l}
	\left( \frac{c_i}{n^{a_1-1}}\right) ^{e_i}, \] 
	where $b$ is the carnality of $H$'s group of isomorphisms. 
	Then because of the criticality hypothesis
	\[ \Ln E(X_n)=0, \]
	and using the first moment method 
	the result follows. 
\end{proof}

\begin{lemma}
	Let $H$ be a critical graph of characteristic $r$.
	Then $H$ contains a critical subgraph with size no greater than
	$(a+2)(r+1)+2a$, where $a$ is the largest edge size in $H$. 
\end{lemma}
\begin{proof}
	Choose $x\in V(H)$. Successively remove from $H$ edges $e$
	such that $dist(x, e)$ is maximum until the resulting graph 
	$H^\prime$ has characteristic no greater than $0$. We have two cases:
	\begin{itemize}
		\item $ch(H^\prime)=1$. Let $e=[x_1, \dots, x_b]$ be
		the last removed edge and
		$e\cap H^\prime=\{ x_{i_1}, \dots, x_{i_c}\}$.
		For any $j=1,\dots, c$ choose 
		$P_j$ a path of size no greater than $r+1$ joining
		$x$ and $x_{i_j}$ in $H^\prime$.   
		Then $P_1\cup \dots P_c \cup e$ is a critical subgraph of $H$
		of size less than $a(r+1) + a< (a+2)(r+1) + 2a$.
		\item $ch(H^\prime)=0$. Let $e_1=[x_1, \dots, x_{b_1}]$ be the
		last removed edge. Continue removing the edges of $H^\prime$ 
		that are at maximum distance from $x$ until you obtain 
		$H^{\prime \prime}$ with $ch(H^\prime)=-1$. Let 
		$e_2=[y_1, \dots, y_{b_1}]$ be the last removed edge.
		As before, let $e_1\cap H^\prime=\{ x_{i_1}, \dots, x_{i_c}\}$
		and for $j=1,\dots, c$ let $P_j$ a path of size no greater than $r+1$ 
		joining	$x$ and $x_{i_j}$ in $H^\prime$
		Then $e_2 \cup H^{\prime \prime}=\{ y_{i_1}, y_{i_2}  \}$.
		Let $Q_1, Q_2$ be paths size no greater than $r+1$ from
		$x$ to $y_{i_1}$ and $y_{i_2}$ in $H^{\prime \prime}$.
		Then $Q_1 \cup Q_2 \cup e_2$ is a graph of characteristic $0$ 
		and size less than $2r+2 + a$,
		and $Q_1\cup Q_2\cup P_1 \cup \dots \cup P_c \cup e_1 \cup e_2$ 
		is a critical graph	with size less than $(2+a)(r+1) + 2a$	
	\end{itemize} 
\end{proof}


\begin{corollary}
	Let $A_n$ be the event that $G(n,p(n))$ contains critical 
	subgraph with diameter no greater than $r$. Then
	\[ \Ln Pr(A_n)=0. \]
\end{corollary}
\begin{proof}
	If a random graph contains $G$ such critical graph, then by the previous
	lemma it has to contain a critical graph of size less than some constant $M$.
	The number of critical graphs of such size is finite and the probability 
	that any one of those appears as a subgraph of $G$ is asymptotically zero.	 
\end{proof}


\begin{theorem} For any $r$,
	\[\Ln Pr(G(n,\beta_1/n^{a_1-1}, \dots,\beta_l/n^{a_l-1} ) \text{ is }r \text{-simple})=1 .\]
\end{theorem}

	
\begin{lemma} 
	Let $(T_1,x)$ and $(T_2,y)$ be rooted trees such that for some $k>0$ 
	$T_1 \morph{k} T_2$. Then, for any $0\leq j < k$, it is satisfied
	$T_1 \morph{j} T_2$.
\end{lemma}
\begin{proof}
	Fix $j<k$. The proof is by induction on the radius of $T_1$ and $T_2$.
	\begin{itemize}
		\item If they have radius at most $0$ they are $j$-morphic by definition. 
		\item Suppose now that they have radii at most $r$. Let $e_1$ be an edge with root
		$x$ in $T_1$ and $e_2$ be an edge of root $y$ in $T_2$ such that $e_1 \morph{k} e_2$.
		Then $e_1=[x_1,\dots,x_a]$ and $e_2=[y_1,\dots,y_a]$ for some vertices such that for all
		$i=1,\dots, a$ either 
		\begin{itemize}
			\item $Tree(x_i)\morph{k} Tree(y_i)$, and because of the induction hypothesis 
			$Tree(x_i)\morph{j} Tree(y_i)$. 
			\item Or $x_i=x$ and $y_i=y$.
		\end{itemize}
		In consequence $e_1\morph{j} e_2$. Then, any class $C$ 
		of $j$-morphic edges in trees of radius at most
		$r$ is union of classes $D_1,\dots, D_s$ of $k$-morphic edges in 
		trees of radius at most $r$. Then
		\[(T_1,C)=\sum_{i=1}^s (T_1, D_i) \text{, and }	(T_2,C)=\sum_{i=1}^s (T_2, D_i).\]	
	 	Both of the above sums coincide if all the $(T_1, D_i)$'s are not greater than $k$ or
	 	both are greater than $k$ otherwise. As $j\leq k$ this implies
	 	\[\text{Either }(T_1,C)=(T_2,C) \text{ or } (T_1,C),(T_2,C)\geq j+1 ,\]
	 	and $T_1\morph{j}T_2$. 		
	\end{itemize}
\end{proof}	


\begin{corollary}
	Let $G_1(c_1,\dots, c_t)$, and $G_2(c_1,\dots,c_t)$ be graphs
	with constants such that $G_1(c_1,\dots, c_t)\morph{k}G_2(c_1,\dots,c_t)$.
	Then, for any $j<k$, $G_1(c_1,\dots, c_t)\morph{j}G_2(c_1,\dots,c_t)$.
\end{corollary}



\begin{lemma}
	Let $(T_1,x)$ and $(T_2,y)$ be rooted trees such that for some $k>0$ 
	$T_1 \morph{k} T_2$. Then, for any $r\geq 0$, $N(x;r)\morph{k} N(y;r)$.
\end{lemma}
\begin{proof}
	The proof is by induction on the radius of $T_1$ and $T_2$.
	\begin{itemize}
		\item If they have at most radius $0$ the statement is vacuously true. 
		\item Suppose now that they have radii at most $r$. Let $e_1$ be an edge with root
		$x$ in $N(x;r)$ and $e_2$ be an edge of root $y$ in $N(y;r)$ such that $e_1 \morph{k} e_2$
		when both edges are considered in $T_1$ and $T_2$ respectively. We want to prove that
		$e_1 \morph{k} e_2$ as well when they are considered as edges in $N(x;r)$ and $N(y;r)$.
		By definition we can write $e_1=[x_1,\dots,x_a]$ and 
		$e_2=[y_1,\dots,y_a]$ for some vertices such that for all
		$i=1,\dots, a$ either 
		\begin{itemize}
			\item $Tree(x_i)\morph{k} Tree(y_i)$, and because of the induction hypothesis 
			\[Tree(x_i)\cap N(x_i;r-a) \morph{k} Tree(y_i)\cap N(y_i;r-a).\] 
			\item Or $x_i=x$ and $y_i=y$.
		\end{itemize}
		In consequence $e_1 \morph{k} e_2$ as edges in $N(x;r)$ and $N(y;r)$. 
		This implies that for any $k$-morphism class of rooted edges $C$ of trees with
		radii less than $r$ there are $k$-morphism classes $D_1,\dots,D_c$ of rooted edges
		of trees with radii at most $r$ such that
		\[ (N(x;r),C) = (T_1, D_1)\cup \dots \cup (T_1,D_a), \]
		and 
		\[(N(y;r),C) = (T_2, D_1)\cup \dots \cup (T_2,D_a).\]
		And we can easily obtain that either
		$|(N(x;r),C)|=|(N(y;r),C)|$ or $|(N(x;r),C)|,$ $
		|(N(y;r),C)|\geq k+1$. 	Thus $N(x;r)\morph{k} N(y;r)$.
	\end{itemize}	
\end{proof}

\begin{lemma}
	Let $(T_1,x),(T^\prime_1,x), (T_2, y)$ and $(T^\prime_2,x)$ 
	be rooted trees satisfying $T_1\morph{k} T_2$.
	$T_1^\prime\morph{k} T_2^\prime$ for some 
	$k\geq 0$ and $V(T_1)\cap V(T_1^\prime)=x$, 
	$V(T_2)\cap V(T_2^\prime)=y$. Then
	$T_1\cup T_1^\prime \morph{k} T_2\cup T_2^\prime$.
\end{lemma}
\begin{proof}
	Let $C$ be a $k$-morphism class of rooted edges. Then
	\[|(T_1\cup T_1^\prime,C)|= |(T_1,C)|+ |(T_1^\prime,C)|, \text{and }\]
	\[|(T_2\cup T_1^\prime,C)|= |(T_2,C)|+ |(T_2^\prime,C)|.\]
	And it follows easily that either
	\[|(T_1\cup T_1^\prime,C)|=|(T_2\cup T_2^\prime,C)|,\]
	or both quantities are greater than $k$. 
\end{proof}


\begin{lemma} 
	Let $(T_1,x)$, $(T_2, y)$ be rooted trees and let 
	$(T^\prime_1,x)$ and $(T^\prime_2,y)$ be sub-trees of 
	$T_1$ and $T_2$ respectively. Let $k\geq 0$.
	If there is an isomorphism 
	$f:T^\prime_1 \rightarrow T^\prime_2$ such that $f(x)=y$ and
	for all $x_1\in V(T^\prime_1)$
	\[ Tree(x_1,T^\prime_1)\morph{k} Tree(f(x_1),T^\prime_2),\]
	then $T_1\morph{k}T_2$. 
\end{lemma}
\begin{proof}
	The proof is, again, by induction on the radius of $T^\prime_1$ and $T^\prime_2$.
	\begin{itemize}
		\item If $T^\prime_1$ and $T^\prime_2$ have radius zero, then they consist only of
		$x$ and $y$ respectively. 
		\item Suppose that $T^\prime_1$ and $T^\prime_2$ have at most $r$. Let $x_1\neq x$ be
		a vertex in an edge $e$ rooted at $x$ such that $e\in T^\prime_1$.
		Then $Tree(x_1, T_1)$ and $Tree(x_1,T^\prime_1)$ satisfy the hypothesis of the lemma,
		and the radius of $Tree(x_1,T^\prime_1)$ is strictly less than $r$. Then by the induction
		hypothesis $Tree(x_1, T_1)\morph{k} Tree(f(x_1),T_2)$. In consequence $e\morph{k}f(e)$
		and using the previous lemma successively we get $T_1\morph{k} T_2$.
	\end{itemize}
\end{proof}

\begin{theorem}
	Let $G_1(c_1,\dots, c_t)$, and $G_2(c_1,\dots,c_t)$ be graphs
	with constants such that for some $r\in N$
	\[ Core(G_1(c_1,\dots, c_t);r)\morph{k} Core(G_2(c_1,\dots,c_t);r)\]
	through $f$. 
	Then, for any $s<r$
	\[ Core(G_1(c_1,\dots, c_t);s)\morph{k} Core(G_2(c_1,\dots,c_t);s).\]
\end{theorem}
\begin{proof}
	Fix $s<r$. Let us introduce some notation
	\[ H_i:= Core(G_i(c_1,\dots, c_t);r) \quad \text{for }i=1,2.\]
	\[ H^\prime_i:= Core(G_i(c_1,\dots, c_t);s) \quad \text{for }i=1,2.\]
	One can check $H^\prime_i\subseteq H_i$ for $i=1,2$, and that the isomorphism 
	$f:Center(H_1)\rightarrow Center(H_2)$ restricts to one between
	$Center(H^\prime_1)$ and $Center(H^\prime_2)$. Let $v\in Center(H^\prime_1)$.
	Let $T=Tree(v,H^\prime_1)\cap Center(H_1)$. It is not hard to see that $T$ is connected 
	and in consequence is a tree. The following identity holds:
	\[ Tree(v,H^\prime_1)= (Tree(v,H_1)\cap N(v;s)) \cup Tree(T,H^\prime_1) \]
	Also, for any $w\in V(T)$
	\[Tree(w,Tree(v,H^\prime_1),T) = Tree(w,H_1)\cap N(w; s-dist(Center(H^\prime_1),w)).\]
	So, by lemma REF 
	\[Tree(w,Tree(v,H^\prime_1),T)\morph{k}Tree(f(w),Tree(f(v),H^\prime_2),f(T)).\]
	In consequence, by lemma REF,
	\[Tree(T,H^\prime_1)\morph{k} Tree(f(T),H^\prime_2).\]
	Using that 
	\[Tree(v,H_1)\cap N(v;s)\morph{k} \morph{k} Tree(v,H_1)\cap N(v;s),\]
	and lemma REF we obtain
	\[Tree(v,H^\prime_1)\morph{k} Tree(f(v),H^\prime_2) .\]
	Hence, $H^\prime_1\morph{k} H^\prime_2$, as desired. 
\end{proof}

\begin{lemma}
	Let $(T_1,x)$, $(T_2, y)$ be rooted trees such that for some $k\geq 0$
	$T_1\morph{k} T_2$. Let $e_1$ and $e_2$ be initial edges of $T_1$ and $T_2$
	such that $e_1\morph{k} e_2$. Then $T_1\setminus e_1\morph{k-1} T_2\setminus e_2$.
\end{lemma}

\begin{proof}
	For any $k$-morphism class $C$ of rooted edges clearly either
	\[|(T_1\setminus e_1,C)|=|(T_2\setminus e_2,C)|,\]
	or both quantities are greater than $k-1$.
	Now, using that any $k-1$ morphism class of rooted edges is union of
	$k$ morphism classes the result follows. 	
\end{proof}

\begin{lemma}
	Let $(T_1,x_1)$, $(T_2,x_2)$ be rooted trees such that for some $k\geq 0$
	$T_1\morph{k} T_2$. For $i=1,2$, given a vertex $v\in V(T_i)$ let us denote by
	$P(v)$ the unique path between $v$ and $x_i$.
	Then for any $v\in V(T_1)$ there is a vertex $w\in V(T_2)$ and an isomorphism
	$f:P(v)\rightarrow P(w)$ such that
	\begin{itemize}
		\item[(1)]  $f(x_1)=x_2$ and $f(v)=w$.
		\item[(2)] For any edge $e\in E(P(v))$, $e\morph{k} f(e)$.
		\item[(3)] For any vertex $y\in V(P(v))$, $Tree(y,T_1)\morph{k} Tree(f(y),T_2)$.
	\end{itemize}
\end{lemma}
\begin{proof}
	The proof is by induction on $d(x_1, v)$.
	\begin{itemize}
		\item If $d(x_1,v)=0$ then $x_1=v$ and the statement is true taking $w=x_2$.
		\item Suppose now that $d(x_1,v)=r$. Then one can write the path $P(v)$ as a
		succession of edges $e_1,e_2,\dots, e_s$, where $v\in e_s$. Let $v^\prime$ be 
		the root of $e_s$. Then $d(x_1,v^\prime)=r-a(e_s)+1$. Thus, by the
		induction hypothesis there exists $w^\prime$ such that there is an isomorphism
		$f: P(v^\prime)\rightarrow P(w^\prime)$ with the required properties. 
		In particular, $Tree(v^\prime,T_1)\morph{k}Tree(w^\prime,T_2)$, so there is an
		edge $e^\prime$ rooted at $w^\prime$ with the same $k$-type as $e_s$. Then one can write
		\[e_s=[v_1,\dots,v_a], \quad  e^\prime=[w_1,\dots,w_a], \]
		in a such a way that for some $i$, $v_i=v^\prime$ and $w_i=w^\prime$ and for all $j\neq i$
		$Tree(v_j,T_1)\morph{k} Tree(w_j,T_2)$. Let $j$ be such that $v=v_j$. Then we can take
		$w=w_j$, and extend the isomorphism $f$ in the evident way. 
	\end{itemize}
\end{proof}



\begin{theorem} 
	Let $G_1(c_1,\dots, c_t)$, and $G_2(c_1,\dots,c_t)$ be graphs
	with constants such that 
	\[ Core(G_1(c_1,\dots, c_t);r)\morph{k} Core(G_2(c_1,\dots,c_t);r)\]
	by means of $f$. 
	Then, for any vertex $x\in Core(G_1(c_1,\dots, c_t);r)$ there is a vertex
	$y\in Core(G_2(c_1,\dots, c_t);r)$ such that 
	\[Core(G_1(c_1,\dots, c_t,c_{t+1}[v_1]);r)\morph{k-1} Core(G_2(c_1,\dots,c_t,c_{t+1}[v_2]);r)\]
\end{theorem}
\begin{proof}
	Let us introduce some notation:
	\[H_i:=Core(G_i(c_1,\dots, c_t);r)\quad \text{for }i=1,2. \]
	\[H^\prime_1:=Core(G_1(c_1,\dots, c_t,c_{t+1}[v_1]);r).\]
	
	The vertex $v_1$ belongs to $Tree(x_1,H_1)$
	for a unique $x_1\in Center(H_1)$. By the previous lemma there exist a vertex $v_2$ in 
	$Tree(f(x_1),H_2)$ such that the path $P(v_1)$, 
	joining $v_1$ and $x_1$, is isomorphic to the path
	$P(v_2)$, joining $v_2$ and $f(x_1)$, through an isomorphism $f^\prime: P(v_1)
	\rightarrow P(v_2)$ satisfying properties (1), (2) and (3). 
	Let 
	\[H^\prime_2:=Core(G_1(c_1,\dots, c_t,c_{t+1}[v_2]);r).\]
	We are going to show that $H^\prime_1\morph{k-1}H^\prime_2$.
	Clearly $Center(H^\prime_i)=Center(H_i)\cup P(v_1)$ for $i=1,2$ , so we can glue
	$f$ and $f^\prime$ into an isomorphism $g:Center(H^\prime_1)\rightarrow Center(H^\prime_2)$.
	Let $w_1\in H^\prime$. Then either
	\begin{itemize}
		\item $Tree(w_1,H_1)$ contains no edges in $P(v_1)$. 
		In this case $Tree(w_1,H^\prime_1)=Tree(w_1,H_1)$.
		Thus, $Tree(w_1,H^\prime_1)\morph{k} Tree(g(w_1),H^\prime_1)$
		and $Tree(w_1,H^\prime_1)\morph{k-1} Tree(g(w_1),H^\prime_1)$
		in consequence by lemma REF.
		\item $Tree(w_1,H_1)$ contains edges from $P(v_1)$.
		In this case, $Tree(w_1,H_1)$ contains exactly one initial edge
		$e_1$ in $P(v_1)$, and $Tree(w_1,H^\prime_1)=Tree(w_1,H_1)\setminus e_1$.
		One can check that $Tree(g(w_1),H_2)$ contains exactly one edge
		in $P(v_2)$, namely $g(e_1)$, and 
		$Tree(g(w_1),H^\prime_2)=Tree(g(w_1),H_2)\setminus g(e_1)$.
		We had $Tree(w_1,H_1)\morph{k}Tree(g(w_1),H_2)$, so by lemma REF
		$Tree(w_1,H^\prime_1)\morph{k-1}Tree($ $g(w_1),H^\prime_2)$.	
	\end{itemize}
\end{proof}


\begin{definition}
	Let $G_1,G_2$ be graphs with constants. We say that $G_1$ and $G_2$ are
	$k$-agreeable if for each $\morph{k}$ class $C$ of connected graphs either
	\begin{itemize}
		\item $G_1$ and $G_2$ have the same number of connected components of type $C$, or
		\item Both $G_1$ and $G_2$ have no less than $k$ connected components of type $C$.
	\end{itemize} 
\end{definition}


\begin{theorem} 
	Let $G_1, G_2$ be $k,3^k$-rich graphs such that $Core(G_1;3^k)$ is $k$-agreeable
	with $Core(G_2;3^k)$. Then Duplicator has a winning strategy in 
	the $E.F$ game on $G_1$ and $G_2$. 
\end{theorem}
\begin{proof}
	We will show, by induction on $i$, that Duplicator can play in such a way
	that in the $i$-th round $Core(G_1(c_1[x_1],\dots,c_i[x_i]);3^{k-i})$
	is $k-i$-agreeable with  $Core(G_2(c_1[y_1],\dots,c_i[y_i]);3^{k-i})$,
	where for each $j=1,\dots,i$,0 $x_j,y_j$ are the $j$-th 
	chosen vertex in the graphs $G_1$ and $G_2$	respectively. After this the theorem
	follows because
	$Core(G_1(c_1[x_1],\dots,c_i[x_k]);1)$ being $0$-agreeable with
	$Core(G_2(c_1[y_1],\dots,c_i[y_k]);1)$ implies that
	the map given by $x_i\mapsto y_i$ defines a partial isomorphism. \par
	\begin{itemize}[leftmargin=*]
		\item For $i=0$ the statement is true by hypothesis. 
		\item Assume now that $Core(G_1(c_1[x_1],\dots,c_i[x_{i-1}]);3^{k-i+1})$
		is $k-i+1$-agreeable with $Core(G_2(c_1[y_1],\dots, c_i[y_{i-1}]); 3^{k-i+1})$.
		Without loss of generality we can suppose that Spoiler chooses a vertex 
		$x_i$ in $G_1$ in the $i$-th round. We have two cases.
		\begin{itemize}[leftmargin=*]
			\item[Case 1.] $N(x_i;3^{k-i})$ is contained in 
			$Core(G_1(c_1[x_1],\dots,c_i[x_{i-1}]);3^{k-i+1})$.
			Let $H_1$ be the connected component of $Core(G_1(c_1[x_1],\dots,c_i[x_{i-1}]);3^{k-i+1})$
			containing $N(x_i;3^{k-i})$. Then there is a connected component $H_2$
			in $Core(G_2(c_1[y_1],\dots,$ $c_i[y_{i-1}]);3^{k-i+1})$ such that
			$H_1 \morph{k-i+1} H_2$. Applying theorems REF and REF successively Duplicator can choose
			$y_i\in V(H_2)$ such that 
			\[ Core(H_1(c_i[x_i]); 3^{k-1})\morph{k-i} Core(H_2(c_i[y_i]); 3^{k-1}).\]
			Using Theorem Ref and counting the connected components in each $k-i$-morphism class
			one can check that now $Core(G_1(c_1[x_1],\dots,c_i[x_i]);3^{k-i})$
			is $k-i$-agreeable with  $Core(G_2(c_1[y_1],\dots,c_i[y_i]);3^{k-i})$.
			\item[Case 2.] $N(x_i;3^{k-i})$ is contained in 
			$Core(G_1(c_1[x_1],\dots,c_i[x_{i-1}]);3^{k-i+1})$. Then
			$N(x_i;3^{k-i})$ is disjoint from $Core(G_1(c_1[x_1],\dots,c_i[x_{i-1}]);3^{k-i})$
			and in particular, $N(x_i;3^{k-i})$ is a tree. As $G_2$ was originally $k,3^k$-rich
			Duplicator can choose $y_i$ in $G_2$ such that $N(y_i;3^{k-i})$ is disjoint from 
			$Core(G_2(c_1[y_1],\dots,c_i[y_{i-1}]);3^{k-i})$ and the tree $N(x_i;3^{k-i})$
			rooted at $x_i$ is $k-i$-morphic to the tree $N(y_i;3^{k-i})$ rooted at $y_i$.
			Counting connected components of each type we can conclude that
			$Core(G_1(c_1[x_1],\dots,c_i[x_i]);3^{k-i})$
			is $k-i$-agreeable with  $Core(G_2(c_1[y_1],\dots,c_i[y_i]);3^{k-i})$.
		\end{itemize}  		
	\end{itemize}
	
\end{proof}

Define the families of expressions with arguments $\beta_1,\dots, \beta_l$: 
$\Lambda$, $\mathcal{E}$ and $\Theta$. \par
For any sentence $\phi$ we use the notation 
$Pr_n(\phi):=Pr(G\models \phi)$. \par
For any F.O. formula $\phi(x_1,\dots,x_l)$ with free variables 
$x_1, \dots, x_l$, we define $Pr_n(\phi(x_1,\dots,x_l))=
\sum_{G\models \phi(a_1,\dots,a_l)} Pr_n(G) $, 
where $a_1, \dots, a_l$ are fixed natural numbers in $[n]$.



\begin{theorem}
	
\end{theorem}

\begin{theorem} 
	Let $\phi(x_1,\dots, x_l)$ be a consistent open formula, and let $k,r,s\in \N$ $s\leq l$. Then for any $k$-morphism classes of trees with radii
	at most $r$ the limit
	\[ \Ln Pr(\bigwedge_{i=i}^s Tree_r(x_i,G)\in C_i \, | \, \phi(x_1,\dots, x_l) \,)\]
	exists and belongs to $\Lambda$.	
\end{theorem}


 



\bibliography{biblio}
\bibliographystyle{unsrt}
\end{document}
